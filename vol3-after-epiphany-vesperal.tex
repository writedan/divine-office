\documentclass[14pt,twoside]{extarticle}
\usepackage{moresize}
\usepackage{fontspec}
\usepackage{aeguill}
\usepackage{titlesec}
\usepackage[maxlevel=3]{csquotes}
\usepackage{lettrine}
\usepackage{xifthen}
\usepackage{yfonts}
\usepackage{parskip}
\usepackage[dvipsnames]{xcolor}
\usepackage{fancyhdr}
\usepackage{hanging}
\usepackage[autocompile,allowdeprecated=false]{gregoriotex}
\usepackage[twoside]{geometry}
\usepackage{multicol}
\usepackage{paracol}
\usepackage[latin,english]{babel}
\usepackage{enumitem}
\usepackage{tabto}
\usepackage{titlesec}
\usepackage[savepos]{zref}
\usepackage{setspace}
\usepackage{xstring}
\def\ReplaceStr#1{%
	\IfSubStr{#1}{\newline}{%
		\StrSubstitute{#1}{\newline}{ }}{#1}}
%\usepackage{libertine}

\setlength{\columnseprule}{0.025pt}
\setlength{\columnsep}{1.5em}
\setlength{\topskip}{7pt}

\def\vehymn{-0.50}

\makeatletter
\patchcmd{\pcol@buildcolseprule}%
{\hrule\@height\@tempdima\@width\columnseprule}%
{\hrule\@height\dimexpr\@tempdima-0.5ex\@width\columnseprule\@depth-1pt}{}{}
\makeatother

\gredefsizedsymbol{GreCross}{greextra}{Cross}
\gredefsizedsymbol{GreDagger}{greextra}{Dagger}
\gredefsizedsymbol{GreCrossAlt}{greextra}{Cross.alt}
\gredefsizedsymbol{GreStarHeight}{greextra}{StarHeight}
\gredefsizedsymbol{greABar}{greextra}{ABar}
\gredefsizedsymbol{greVBar}{greextra}{VBar}
\gredefsizedsymbol{greRBar}{greextra}{RBar}

\gresetspecial{r}{\textcolor{red}{\Rbar{}.}}
\gresetspecial{v}{\textcolor{red}{\Vbar{}.}}
\gresetspecial{a}{\textcolor{red}{\Abar{}.}}
\gresetspecial{+}{\textcolor{red}{\GreCross{17}}}
\gresetspecial{1}{\textcolor{red}{\GreDagger{17}}}
\gresetspecial{2}{\textcolor{red}{\GreCrossAlt{17}}}
\gresetspecial{*}{\textcolor{red}{\GreStarHeight{17}}}
%%% for spacing in chant %%%
\gresetspecial{ }{\textcolor{white}{x}}
\gresetspecial{  }{\textcolor{white}{xx}}
\gresetspecial{   }{\textcolor{white}{xxx}}
\gresetspecial{    }{\textcolor{white}{xxxx}}

\grechangestaffsize{30}
%\grechangedim{abovelinestextheight}{0.25cm}{fixed}
%\grechangedim{spacebeneathtext}{0.5cm}{fixed}
%\grechangedim{annotationseparation}{0.25cm}{fixed}


\newcommand{\skiplines}[1]{\pagebreak[1]\vspace*{#1\baselineskip}}

\makeatletter
\newcommand{\veformat}[1]{\textcolor{red}{\textbf{#1}}}
\newcommand{\veant}[2]{\greannotation{#1. \veformat{Ant.}}\greannotation{\veformat{#2.}}}
\newcommand{\instruct}[1]{\textcolor{red}{\textit{#1}}}

\newcommand{\whiteribbon}[0]{\textcolor{black}{\textbf{white ribbon}} }%
\newcommand{\greenribbon}[0]{\textcolor{ForestGreen}{\textbf{green ribbon}} }%
\newcommand{\redribbon}[0]{\textcolor{red}{\textbf{red ribbon}} }%
\newcommand{\blueribbon}[0]{\textcolor{blue}{\textbf{blue ribbon}} }%
\makeatother

%%% HEADERS %%%
\makeatletter
\newcommand{\vepart}[1]{%
	\cleardoublepage%
	\vspace*{\fill}%
	\begin{center}%
		\setstretch{2}
		{\Huge \selectfont \uppercase{ \textbf{#1}}}\par%
	\end{center}%
	\vspace*{\fill}%
	\fancyhead{}
}%
\makeatother

\newcommand{\vechapter}[1]{%
	\cleardoublepage%
	\fancyhead[RO,RE]{text}
	\StrSubstitute{#1}{\\}{\space}[\vetempmacrolol]%
	\fancyhead[LO,LE]{\vetempmacrolol}%
	\begin{center}
		\setstretch{1.5}
		{\LARGE \selectfont \color{red} \uppercase{ \textbf{#1}}}\par%
	\end{center}
	\skiplines{1}
}%
\makeatother

\newcommand{\vesectionp}[1]{%
	\skiplines{1}
	\fancyhead[RO,RE]{#1}%
	\begin{center}
		\setstretch{1.5}
		{\Large \selectfont \uppercase{ \textbf{#1}}}\par%
	\end{center}
	\skiplines{1}
}%
\makeatother

\newcommand{\vesection}[1]{%
	\clearpage\skiplines{-2}\vesectionp{#1}%
}%
\makeatother

\makeatletter
\newcommand{\veheading}[1]{%
	\skiplines{1}%
	\begin{center}
		\setstretch{1.5}
		{\selectfont \textit{#1}}\par%
	\end{center}
	\skiplines{1}
}%
\makeatother

\makeatletter
\newcommand{\vecomment}[1]{%
	\hfill{\small\selectfont \textit{#1}.}%
}%
\makeatother

%%% OTHER FORMATS %%%

\makeatletter
\newcommand{\red}[1]{%
	{\color{red} #1}%
}%
\makeatother

\makeatletter
\newcommand{\chapter}[1]{%
	\gresetinitiallines{1}%
	\greannotation{\veformat{Chap.}}%
	\greannotation{\veformat{Dir.}}%
	\gabcsnippet{#1}%
}%
\makeatother

\makeatletter
\newcommand{\resp}[2]{%
	\vecomment{#1}
	
	\gresetinitiallines{1}%
	\greannotation{\veformat{Resp.}}%
	\greannotation{\veformat{#2.}}%
}%
\makeatother

\makeatletter
\newcommand{\tone}[2]{%
	\greannotation{\veformat{Ps.}}
	\greannotation{\veformat{#1.}}
	
	\gabcsnippet{#2}
}%
\makeatother

\makeatletter
\newcommand{\repant}[1]{%
	\gresetspecial{*}{}
	\nopagebreak[4]\skiplines{1}\nopagebreak[4]\gresetinitiallines{0}\gregorioscore{#1}\gresetinitiallines{1}
	\gresetspecial{*}{\textcolor{red}{\GreStarHeight{17}}}
}%
\makeatother



%%% FANCY %%%
\makeatletter  
\newcounter{score}
\newcounter{tabstop}[score]
\newcommand{\grealign}{%
	\@bsphack%
	\ifgre@boxing\else%
	\kern\gre@dimen@begindifference%
	\stepcounter{tabstop}%
	\expandafter\zsavepos{stop-\thescore-\thetabstop}%
	\kern-\gre@dimen@begindifference%
	\fi%
	\@esphack%
}

\newcommand{\setstops}{%
	\gdef\nstabbing@stops{%
		\hspace*{-\oddsidemargin}\hspace{-1in}%
		\hspace*{\zposx{stop-\thescore-1} sp}\=%
	}%
	\count@=\@ne
	\loop\ifnum\count@<\value{tabstop}%
	\begingroup\edef\x{\endgroup
		\noexpand\g@addto@macro\noexpand\nstabbing@stops{%
			\noexpand\hspace{-\noexpand\zposx{stop-\thescore-\the\count@} sp}%
			\noexpand\hspace{\noexpand\zposx{stop-\thescore-\the\numexpr\count@+1} sp}\noexpand\=%
		}%
	}\x
	\advance\count@\@ne
	\repeat
	\nstabbing@stops\kill
}
\makeatother

\newenvironment{nstabbing}
{\setlength{\topsep}{0pt}%
	\setlength{\partopsep}{0pt}%
	\tabbing%
	\setstops}
{\endtabbing\stepcounter{score}}

\begin{document}
	\newgeometry{margin=0pt}
	\begin{titlepage}
		\vspace*{\fill}
		
		\begin{center}
			
			\textbf{\Huge\color{red} VESPERAL}
			
			ACCORDING TO THE RITE OF THE
			
			\textbf{\LARGE HOLY ROMAN CHURCH}
			
			\vspace*{\fill}
			
			\textbf{\Large\color{red} VOLUME III}
			
			Second Sunday after Epiphany\\Until the First Sunday in Lent
			
		\end{center}
		
		\vspace*{\fill}
	\end{titlepage}
	
	\newgeometry{top=0.25in,bottom=0.5in,inner=0.75in,outer=0.75in,includehead,nofoot}
	\pagestyle{fancy}
	
	\pagenumbering{arabic}
	\gresetinitiallines{0}
	
	\raggedbottom
	
	%\tableofcontents
	
	
	%%% COPY ABOVE %%%
	
	\vechapter{Liturgical Calendar}
	\skiplines{-1}\vesectionp{January}
		\begin{enumerate}[noitemsep, nolistsep]
			\setcounter{enumi}{13}
			\item % 14
			\item % 15
			\item % 16
			\item % 17
			\item % 18
			\item % 19
			\item % 20
			\item Saint Agnes, Virgin-Martyr. \red{9 lect. duplex} % 21
			\item % 22
			\item % 23
			\item % 24
			\item \red{Conversion of Saint Paul, Apostle.} Minor feast % 25
			\item % 26
			\item % 27
			\item % 28
			\item % 29
			\item % 30
			\item % 31
		\end{enumerate}
	
	\vesectionp{February}
		\begin{enumerate}[noitemsep, nolistsep]
			\item % 1
			\item \red{Purification of the Blessed Virgin Mary.} Principle feast. % 2
			\item % 3
			\item % 4
			\item Saint Agatha, Virgin-Martyr. \red{9 lect. duplex.} % 5
			\item % 6
			\item % 7
			\item % 8
			\item % 9
			\item % 10
			\item % 11
			\item % 12
			\item % 13
			\item Saint Valentine, Priest and Martyr. \red{3 lect. simplex.} % 14
			\item % 15
			\item % 16
			\item % 17
			\item % 18
			\item % 19
			\item % 20
			\item % 21
			\item % 22
			\item % 23
			\item % 24
			\item % 25
			\item % 26
			\item % 27
			\item % 28
			\item % 29
		\end{enumerate}
	
	\vesectionp{March}
		\begin{enumerate}[noitemsep, nolistsep]
			\item % 1
			\item % 2
			\item % 3
			\item % 4
			\item % 5
			\item % 6
			\item % 7
			\item % 8
			\item % 9
			\item % 10
			\item % 11
			\item % 12
			\item % 13
			\item % 14
		\end{enumerate}
	
	\vepart{Ordinary}
	
		\vechapter{The Order of Vespers}
		
			\vesectionp{Introduction}\label{vespers:introduction}

	\instruct{Before the service begins, kneel and make these prayers quietly.}
	
	Our Father, who art in heaven, hallowed be thy name. Thy kingdom come. Thy will be done on earth as it is in heaven. Give us this day our daily bread and forgive us our trespasses as we forgive those who trespass against us. And lead us not into temptation but deliver us from evil.
	
	Hail Mary, full of grace, the Lord is with thee. Blessed art thou among women and blessed is the fruit of thy womb, Jesus. Holy Mary, Mother of God, pray for us sinners now and at the hour of our death. Amen.
	
	\instruct{Stand as the minister begins from his place:}
	
	\gresetinitiallines{0}
	\skiplines{1}\gabcsnippet{(c3) <sp>v</sp> O(fv) God,(hv) make(hv) speed(iv) to(h) save(g) me.(hv) <sp>r</sp>(::) O(hv) Lord,(hv) make(hv) haste(hv) to(hv) help(g) me.(hv)}
	
	\instruct{The congregation continues together, turning toward the altar, making a profound bow, and crossing themselves at the invocation of the Holy Trinity here and whenever else:}
	
	\skiplines{1}\gabcsnippet{(c3) Glor(h)y(h) be(h) to(h) the(h) Fa(h)ther,(h) and(h) to(h) the(h) Son,(h) (,) and(h) to(h) the(h) Ho(h)ly(h) Spir(g)it.(hv) (:) As(h) it(h) was(h) in(h) the(h) be(h)gin(h)ning,(h) is(h) now(h) and(h) ev(h)er(h) shall(g) be:(hv) (,) world(hv) with(hv)out(hv) end.(hv) A(g)men.(hv) (::) A(hv)le(hi)lu(h)ia.(hg)}
	
\vesectionp{Psalmody}\label{vespers:psalmody}

	\instruct{Turn to the \greenribbon and sing the psalms with their antiphons as appointed there.}
	
\vesectionp{Chapter}\label{vespers:chapter}

	\instruct{The lector then proceeds to stand before the altar and reads the chapter. The congregation responds:}
	
	\skiplines{1}\gabcsnippet{(c3) Thanks(h) be(h) to(h) God.(h)}
	
	\instruct{Turn to the \redribbon and sing the responsory if any is appointed or prescribed there, and the hymn and versicle.}
	
\vesection{Magnificat}\label{vespers:magnificat}

	\instruct{The Magnificat is sung in like manner to the psalmody. The minister incenses the altar during its singing.}

\vesectionp{Conclusion}\label{vespers:conclusion}

	\instruct{All kneel as the cantors turn to the \greenribbon and sing the Kyrie as appointed there. The minister proceeds to incense the whole church and congregation.}
	
	\instruct{This being accomplished, all continue kneeling and pray quietly.}
	
	Our Father, who art in heaven, hallowed be thy name. Thy kingdom come. Thy will be done on earth as it is in heaven. Give us this day our daily bread and forgive us our trespasses as we forgive those who trespass against us. And lead us not into temptation but deliver us from evil.

	\instruct{All stand as the minister begins.}
	
	\skiplines{1}\gabcsnippet{(c3) <sp>v</sp> O(h) Lord,(h) hear(h) my(f) prayer.(g) <sp>r</sp>(::) And(h) let(h) my(h) cry(h) come(h) un(h)to(f) thee.(f)}
	
	\instruct{The minister will then proclaim the collect of the day. The congregation responds:}
	
	\skiplines{1}\gabcsnippet{(c3) A(gh)men.(h)}
	
	\instruct{This petition concludes the office.}
	
	\skiplines{1}\gregorioscore{common/conclusion.gabc}
	
	\instruct{All then kneel and make this prayer quietly.}
	
	Our Father, who art in heaven, hallowed be thy name. Thy kingdom come. Thy will be done on earth as it is in heaven. Give us this day our daily bread and forgive us our trespasses as we forgive those who trespass against us. And lead us not into temptation but deliver us from evil.
			
		\vechapter{The Order of Compline}
		
			\vesectionp{Introduction}\label{compline:introduction}

	\instruct{Before the service begins, kneel and make these prayers quietly.}
	
	Our Father, who art in heaven, hallowed be thy name. Thy kingdom come. Thy will be done on earth as it is in heaven. Give us this day our daily bread and forgive us our trespasses as we forgive those who trespass against us. And lead us not into temptation but deliver us from evil.
	
	Hail Mary, full of grace, the Lord is with thee. Blessed art thou among women and blessed is the fruit of thy womb, Jesus. Holy Mary, Mother of God, pray for us sinners now and at the hour of our death. Amen.
	
	\instruct{Continue kneeling as the minister begins in a low voice.}
	
	\gresetinitiallines{0}
	\skiplines{1}\gabcsnippet{
		(f3) <sp>v</sp> Turn(f) us(hv) then,(h) O(h) God(h) our(h) Sav(g)ior.(hv)
		<sp>r</sp>(::) And(hv) let(h) thine(h) an(h)ger(h) cease(h) from(g) us.(hv)
	}

	\instruct{Stand as the minister begins from his place:}
	
	\skiplines{1}\gabcsnippet{(c3) <sp>v</sp> O(f) God,(hv) make(h) speed(iv) to(h) save(g) me.(hv) <sp>r</sp>(::) O(hv) Lord,(h) make(h) haste(h) to(h) help(g) me.(hv)}
	
	\instruct{The congregation continues together, turning toward the altar, making a profound bow, and crossing themselves at the invocation of the Holy Trinity here and whenever else:}
	
	\skiplines{1}\gabcsnippet{(c3) <sp>a</sp> Glor(h)y(h) be(h) to(h) the(h) Fa(h)ther,(h) and(h) to(h) the(h) Son,(h) (,) and(h) to(h) the(h) Ho(h)ly(h) Spir(g)it.(h) (:) As(h) it(h) was(h) in(h) the(h) be(h)gin(h)ning,(h) is(h) now(h) and(h) ev(h)er(h) shall(g) be:(h) (,) world(h) with(h)out(h) end.(h) A(g)men.(h) (::) A(h)le(hi)lu(h)ia.(hg)}
	
\vesectionp{Psalmody and Propers}

	\instruct{Turn to the \greenribbon and sing the psalms with their antiphons appointed there.}

	\instruct{Once the psalmody has been completed, the lector should proceed to stand before the altar and from there proclaim the chapter from the capitulary. The congregation responds.}

	\skiplines{1}\gabcsnippet{(c3) <sp>r</sp> Thanks(h) be(h) to(h) God.(h)}

	\instruct{The lector will then intone the responsory from his place. Once it is finished, he will return to his place.}

	\instruct{Turn to the \blueribbon and sing the responsory, hymn, versicle, and canticle there appointed until returning here.}
	
\vesectionp{Conclusion}\label{compline:conclusion}

	\instruct{All then kneel as the cantors turn back to the \redribbon and sing the kyrie as there appointed. The minister should incense the whole church and congregation while it is sung, proceeding from down the north, up the nave, down the south, and up the nave again.}
	
	\instruct{Once the minister returns to his place, all continue kneeling and quietly make this prayer.}
	
	Our Father, who art in heaven, hallowed be thy name. Thy kingdom come. Thy will be done on earth as it is in heaven. Give us this day our daily bread and forgive us our trespasses as we forgive those who trespass against us. And lead us not into temptation but deliver us from evil.
	
	\instruct{All then say together.}
	
	I confess to almighty God and to all the Saints that I have sinned exceedingly in thought, word, and deed: through my fault, through my fault, through my most grievous fault. Therefore I beseech blessed Mary ever-Virgin, blessed Michael the Archangel, blessed John the Baptist, the holy apostles Peter and Paul, all the Saints, and you, brethren, to pray for me to the Lord our God.
	
	\instruct{The minister alone continues.}
	
	May almighty God \GreCross{17} have mercy upon us and forgive us all our sins, deliver us from all evil, and bring us to everlasting life. Amen.
	
	\instruct{All arise as the minister begins.}
	
	\skiplines{1}\gabcsnippet{
		(c3) <sp>v</sp> O(h) Lord,(h) hear(h) my(f) prayer.(g)
		<sp>r</sp>(::) And(h) let(h) my(h) cry(h) come(h) un(h)to(f) thee.(f)
		(::) <sp>v</sp> Let(h) us(g) pray.(h)
	}
	
	\instruct{The minister will then pray the collect from the collectary. This completed, all respond.}
	
	\skiplines{1}\gabcsnippet{<sp>r</sp> A(gh)men.(h)}
	
	\instruct{Then is said this petition.}
	
	\skiplines{1}\gregorioscore{common/compline_conclusion.gabc}
	
	\instruct{All turn back to the \blueribbon and sing the Marian anthem there appointed. The minister proceeds from his place from the north to the Marian altar if any be present, else he remains. He is to carry incense in the procession and incense her altar upon arrival.}
	
	\instruct{The versicle prescribed at the \blueribbon is then made; the minister says}
	
	\skiplines{1}\gabcsnippet{(c3) Let(h) us(g) pray.(h)}
	
	\instruct{He then prays the prescribed collect from the collectary. All respond.}
	
	\skiplines{1}\gabcsnippet{<sp>r</sp> A(gh)men.(h)}
	
	\instruct{All then kneel and make these prayers quietly.}
	
	Our Father, who art in heaven, hallowed be thy name. Thy kingdom come. Thy will be done on earth as it is in heaven. Give us this day our daily bread and forgive us our trespasses as we forgive those who trespass against us. And lead us not into temptation but deliver us from evil.
	
	Hail Mary, full of grace, the Lord is with thee. Blessed art thou among women and blessed is the fruit of thy womb, Jesus. Holy Mary, Mother of God, pray for us sinners now and at the hour of our death. Amen.
	
	I believe in God the Father almighty, maker of heaven and earth. And in Jesus Christ his only Son our Lord: who was conceived by the Holy Spirit, born of the Virgin Mary, suffered under Pontius Pilate, was crucified, died, and buried. He descended into hell. The third day he rose again from the dead; he ascended into heaven, and sitteth on the right hand of God the Father almighty. From whence he shall come to judge the living and the dead. I believe in the Holy Spirit, the holy catholic Church, the communion of the Saints, the remission of sins, the resurrection of the body, and the life everlasting. Amen.
			
	\gresetinitiallines{1}
			
	\vepart{Psalter}
	
		\vechapter{Sunday at Vespers}
		
			\veant{1}{i. v}
\gregorioscore{ant/sede_a_dextris.gabc}

\greannotation{\veformat{Ps. 109}}
\greannotation{\veformat{i. v.}}
\skiplines{1}\gabcsnippet{
	(f3) The(hv) Lord(iv) (:?) said(jvr) unto(jv) <b>my</b>(i) Lord(jv) <sp>*</sp>(:) Sit(jrv) thou() on() my() right() hand,() until() I() make() thine() en(jv)e(jv)<b>mies</b>(i) thy(h) foot(irv)stool.(gf)
}

\begin{multicols}{2}
	The Lord shall send the rod of thy power out of \textbf{Zi}on * be thou ruler, even in the midst a\textbf{mong} thine enemies.
	
	In the day of thy power shall the people offer thee free-will offerings with a holy \textbf{wor}ship * the dew of thy birth is of the womb \textbf{of} the morning.
	
	The Lord sware, and will not \textbf{re}pent: * Thou art a priest forever after the order \textbf{of} Melchizedek.
	
	The Lord upon thy \textbf{right} hand * shall wound even kings in the \textbf{day} of his wrath.
	
	He shall judge among the heathen; he shall fill the places with the dead \textbf{bod}ies * and smite in sunder the heads over \textbf{di}verse countries.
	
	He shall drink of the brook in \textbf{the} way * therefore shall \textbf{he} lift up his head.
	
	Glory be to the Father, and to \textbf{the} Son * and to the \textbf{Ho}ly Spirit.
	
	As it was in the beginning is now and ever \textbf{shall} be: * world with\textbf{out} end. Amen.
\end{multicols}

\repant{ant/sede_a_dextris.gabc}


\pagebreak[1]


\veant{2}{iv. vi}
\skiplines{0}\gregorioscore{ant/fidelia_omnia.gabc}

\greannotation{\veformat{Ps. 110}}
\greannotation{\veformat{iv. vi}}
\skiplines{1}\gabcsnippet{
	(f3) I(jv) will(i) (:?) give(jrv) thanks(jv) unto() the() Lord(jv) <b>with</b>(i) my(jv) whole(kv) heart(jr) <sp>*</sp>(:) se(jr)cretly(j) among() the() faithful() and() in() the() congrega(j)<b>tion.</b>(i)
}

\begin{multicols}{2}
	The works \textbf{of} the Lord are great * sought out of all them that \textbf{have} pleasure therein.
	
	His work is worthy to be praised, and \textbf{had} in honor * and his righteousness en\textbf{dur}eth forever.
	
	The merciful and gracious Lord hath so done his \textbf{mar}velous works * that they ought to be \textbf{had} in remembrance.
	
	He hath given meat unto \textbf{them} who fear him * he shall ever be mind\textbf{ful} of his covenant.
	
	He hath shewed his people the \textbf{pow}er of his works * that he may give them the herit\textbf{age} of the heathen.
	
	The works of his hands are veri\textbf{ty} and judgment * all his \textbf{com}mandments are true.
	
	They stand fast fore\textbf{ver} and ever * and are done \textbf{in} truth and equity.
	
	He sent redemption un\textbf{to} his people * he hath commanded his covenant forever; holy \textbf{and} reverend is his Name.
	
	The fear of the Lord is the begin\textbf{ning} of wisdom * a good understanding have all they who do thereafter; his praise en\textbf{dur}eth forever.
	
	Glory be to the Fath\textbf{er}, and to the Son * and to \textbf{the} Holy Spirit.
	
	As it was in the beginning, is now and \textbf{ev}er shall be * world \textbf{with}out end. Amen.
\end{multicols}

\skiplines{-1}\repant{ant/fidelia_omnia.gabc}


\pagebreak[1]

\veant{3}{iv. vi}
\skiplines{1}\gregorioscore{ant/in_mandatis_ejus.gabc}

\greannotation{\veformat{Ps. 111}}
\greannotation{\veformat{iv. vi}}
\skiplines{1}\gabcsnippet{
	(f3) Bles(jv)ed(i) (:?) is(jrv) the(jv) man() that(jv) <b>fear</b>(i)eth(jv) the(kv) Lord(jr) <sp>*</sp>(:) he(jr) hath() great() delight() in() his() com(j)mand(j)<b>ments.</b>(i)
}

\begin{multicols}{2}
	His seed shall be migh\textbf{ty} upon earth * the generation of the faithful shall be bles\textbf{sed}.
	
	Riches and plenteousness \textbf{shall} be in his house * and his righteousness endureth fore\textbf{ver}.
	
	Unto the godly there ariseth up light \textbf{in} the darkness * he is merciful, loving, and righ\textbf{teous}.
	
	A good man is merci\textbf{ful}, and lendeth * and will guide his words with discre\textbf{tion}.
	
	For he \textbf{shall} never be moved * and the righteous shall be had in everlasting remem\textbf{brance}.
	
	He will not be afraid of any \textbf{ev}il tidings * for his heart standeth fast, and believeth in the \textbf{Lord}.
	
	His heart is esta\textbf{blished}, and will not shrink * until he sees his desire upon his enem\textbf{ies}.
	
	He hath dispersed abroad, and \textbf{giv}en to the poor * and his rightouesness remaineth forever; his horn shall be exalted with hon\textbf{or}.
	
	The ungodly shall see it, and \textbf{it} shall grieve him * he shall gnash with his teeth, and consume away; the desire of the ungodly shall per\textbf{ish}.
	
	Glory be to the Fath\textbf{er}, and to the Son * and to the Holy Spir\textbf{it}.
	
	As it was in the beginning, is now and \textbf{ev}er shall be * world without end. A\textbf{men}.
\end{multicols}

\skiplines{-1}\repant{ant/in_mandatis_ejus.gabc}

\pagebreak[1]


\veant{4}{vii. ii}
\skiplines{1}\gregorioscore{ant/sit_nomen_domini.gabc}

\greannotation{\veformat{Ps. 112}}
\greannotation{\veformat{vii. ii.}}
\skiplines{1}\gabcsnippet{
	(c3) Praise(gv) ye(hv) (:?) the(irv) <b>Lord,</b>(kv) ye(jr) <b>ser</b>(i)vants(jv) <sp>*</sp>(:) O(ir) <b>praise</b>(jv) the(ir) Name(i) <b>of</b>(h) the(hr) Lord.(gh)
}

\begin{multicols}{2}
	Blessed be the \textbf{Name} of \textbf{the} Lord * from this time \textbf{forth} for\textbf{ev}ermore.
	
	The \textbf{Lord's} Name \textbf{is} praised * from the rising up of the sun unto the \textbf{go}ing down \textbf{of} the same.
	
	The Lord is high a\textbf{bove} all \textbf{ hea}then * and his glory a\textbf{bove} the \textbf{heav}ens.
	
	Who is like unto the Lord our God, that hath his \textbf{dwel}ling \textbf{so} high? * and yet humbleth himself to behold the things that are in \textbf{hea}ven \textbf{and} earth?
	
	He taketh up the simple \textbf{out} of \textbf{the} dust * and lifteth the poor \textbf{out} \textbf{of} the mire.
	
	That he may sit him \textbf{with} the \textbf{prin}ces * even with the princes \textbf{of} his \textbf{peo}ple.
	
	He maketh the barren \textbf{wo}man to \textbf{keep} house * and to be the joyful \textbf{mo}ther of \textbf{child}ren.
	
	Glory be to the \textbf{Fa}ther, and to \textbf{the} Son * and to the \textbf{Ho}ly \textbf{Spir}it.
	
	As it was in the beginning is now, \textbf{and} ever \textbf{shall} be * world \textbf{with}out end. \textbf{A}men.
\end{multicols}

\skiplines{-1}\repant{ant/sit_nomen_domini.gabc}


\pagebreak[1]

\veant{5}{T. Per}
\skiplines{0}\gregorioscore{ant/nos_qui_vivimus.gabc}

\skiplines{-0.25}\greannotation{\veformat{Ps. 113}}
\greannotation{\veformat{T. Per.}}
\gregorioscore{psalter/113/peregrine.gabc}

\repant{ant/nos_qui_vivimus.gabc}

\instruct{Turn back to the \whiteribbon for the chapter.}
		
		\vechapter{Monday at Vespers}
		
			\veant{1}{i. iii}
\gregorioscore{ant/inclinavit.gabc}

\greannotation{\veformat{Ps. 114}}
\greannotation{\veformat{i. iii.}}
\skiplines{1}\gabcsnippet{
	(c4) I(hr) am(h) <b>well</b>(g) pleased(hv) <sp>*</sp>(:) that(hrv) the() Lord() hath() heard() the(hv) <b>voice</b>(g) of(f) my(grv) prayer.(ghg)
}

\begin{multicols}{2}
	That he hath inclined his ear un\textbf{to} me * therefore will I call upon him as \textbf{long} as I live.
	
	The snares of death compassed me round \textbf{a}bout * and the pains of hell gat \textbf{hold} upon me.
	
	I shall find trouble and heaviness, and I will call upon the Name of \textbf{the} Lord * O Lord, I beseech thee, de\textbf{li}ver my soul.
	
	Gracious is the Lord, and \textbf{right}eous * truly, our \textbf{God} is merciful.
	
	The Lord preserveth the \textbf{sim}ple * I was in misery, \textbf{and} he helped me.
	
	Turn again then unto thy rest, O \textbf{my} soul * for the Lord \textbf{hath} rewarded thee.
	
	And why? thou hast delivered my soul \textbf{from} death * mine eyes from tears, and my \textbf{feet} from falling.
	
	I will walk before \textbf{the} Lord * in the land \textbf{of} the living.
	
	Glory be to the Father, and to \textbf{the} Son * and to \textbf{the} Holy Spirit.
	
	As it was in the beginning, is now, and ever \textbf{shall} be * world with\textbf{out} end. Amen.
\end{multicols}

\repant{ant/inclinavit.gabc}


\veant{2}{viii. ii}
\skiplines{1}\gregorioscore{ant/credidi_propter.gabc}

\greannotation{\veformat{Ps. 115}}
\greannotation{\veformat{viii. ii.}}
\skiplines{1}\gabcsnippet{
	(c3) I(e) be(f)lieved(h) (:?) and(hr) therefore() will() I() speak;() but() I() was() sore(h) <b>trou</b>(i)bled(hr h) <sp>*</sp>(:) I(hr) said in my haste, All(h) <b>men</b>(f) are(h) li(i)ars.(hr h)
}

\begin{multicols}{2}
	What reward shall I give un\textbf{to} the Lord * for all the benefits that he \textbf{hath} done unto me?
	
	I will receive the cup of sal\textbf{va}tion * and call upon \textbf{the} Name of the Lord.
	
	I will pay my vows now in the presence of all his \textbf{peo}ple * right dear in the sight of the Lord is \textbf{the} death of his saints.
	
	Behold, O Lord, how that I am thy \textbf{ser}vant * I am thy servant, and the son of thine handmaid; thou hast broken my \textbf{bonds} in sunder.
	
	I will offer to thee the sacrifice of thanks\textbf{giv}ing * and will call upon \textbf{the} Name of the Lord.
	
	I will pay my vows unto the Lord, in the sight of all his \textbf{peo}ple * in the courts of the Lord's house, even in the midst of thee, O \textbf{Je}rusalem.
	
	Glory be to the Father, and \textbf{to} the Son * and to the \textbf{Ho}ly Spirit.
	
	As it was in the beginning, is now, and ever \textbf{shall} be * world with\textbf{out} end. Amen.
\end{multicols}

\repant{ant/credidi_propter.gabc}



\veant{3}{iv. 1}
\skiplines{1}\gregorioscore{ant/laudate_dominum.gabc}

\greannotation{\veformat{Ps. 116}}
\greannotation{\veformat{iv. i.}}
\skiplines{1}\gabcsnippet{
	(c4) O(h) praise(g) (:?) the(hr) Lord,(h) <b>all</b>(g) ye(h) hea(i)then(hr h) <sp>*</sp>(:) praise(hr) <b>him</b>,(g) all(h) ye(ih) na(gvFE)tions.(e)
}

\begin{multicols}{2}
	For his merciful kindness is ever more and \textbf{more} toward us * and the truth of the Lord endureth \textbf{for}ever, praise the Lord.
	
	Glory be to the Fa\textbf{ther}, and to the Son * and to \textbf{the} Holy Spirit.
	
	As it was in the beginning, is now, and \textbf{ev}er shall be * world \textbf{with}out end. Amen.
\end{multicols}

\repant{ant/laudate_dominum.gabc}



\veant{4}{iv. vii}
\skiplines{1}\gregorioscore{ant/clamavi.gabc}

\greannotation{\veformat{Ps. 119}}
\greannotation{\veformat{iv. vii.}}
\skiplines{1}\gabcsnippet{
	(c4) When(h) I(g) (:?) was(hr) in() trouble,() I(h) called(g) u(h)pon(i) the(hr) Lord(h) <sp>*</sp>(:) and(hr) he(h) <b>heard</b>(g) me.(h)
}

\begin{multicols}{2}
	Deliver my soul, O \textbf{Lord}, from lying lips * and from a deceit\textbf{ful} tongue.
	
	What reward shall be given or done unto \textbf{thee}, thou false tongue? * even mighty and sharp arrows, with hot burn\textbf{ing} coals.
	
	Woe is me, that I am constrained to \textbf{dwell} with Mesech * and to have my habitation among the tents of \textbf{Ke}dar.
	
	My soul hath long \textbf{dwelt} among them * that are enemies un\textbf{to} peace.
	
	I labor for peace, but when I speak \textbf{un}to them thereof * they make them ready \textbf{to} battle.
	
	Glory be to the Fa\textbf{ther}, and to the Son * and to the Holy \textbf{Spir}it.
	
	As it was in the beginning, is now and \textbf{ev}er shall be. * world without out end. \textbf{A}men.
\end{multicols}

\repant{ant/clamavi.gabc}



\veant{5}{viii. ii}
\skiplines{1}\gregorioscore{ant/auxilium.gabc}

\greannotation{\veformat{Ps. 120}}
\greannotation{\veformat{viii. ii.}}
\skiplines{1}\gabcsnippet{
	(c3) I(e) will(f) (:?) lift(hr) up() mine() eyes() un(h)<b>to</b>(i) the(hr) hills(h) <sp>*</sp>(:) from(hr) whence(h) <b>com</b>(f)eth(h) my(i) help.(hr h)
}

\begin{multicols}{2}
	My help cometh even \textbf{from} the Lord * who hath made \textbf{hea}ven and earth.
	
	He will not suffer thy foot \textbf{to} be moved * and he that keep\textbf{eth} thee will not sleep.
	
	Behold, he that keepeth \textbf{Is}rael * shall neither \textbf{slum}ber nor sleep.
	
	The Lord himself is thy \textbf{keep}er * the Lord is thy defense up\textbf{on} thy right hand.
	
	So that the sun shall not burn \textbf{thee} by day * nei\textbf{ther} the moon by night.
	
	The Lord shall preserve thee from all \textbf{e}vil * truly, it is even he \textbf{that} shall keep thy soul.
	
	The Lord shall preserve thy going out, and thy \textbf{com}ing in * from this time \textbf{forth} forevermore.
	
	Glory be to the Father, and \textbf{to} the Son * and to the \textbf{Ho}ly Spirit.
	
	As it was in the beginning, is now, and ever \textbf{shall} be * world with\textbf{out} end. Amen.
\end{multicols}

\repant{ant/auxilium.gabc}

\instruct{Turn back to the \whiteribbon.}
		
		\vechapter{Tuesday at Vespers}
		
			\veant{1}{iv. vi}
\gregorioscore{ant/in_domum_domini.gabc}

\greannotation{\veformat{Ps. 121}}
\greannotation{\veformat{iv. vi.}}
\skiplines{1}\gabcsnippet{
	(c4) I(h) was(g) (:?) glad(hr) when(h) <b>they</b>(g) said(h) un(i)to(hr) me(h) <sp>*</sp>(:) We(hr) will() go() into() the() house(h) of(h) the(h) Lord.(g) 
}

\begin{multicols}{2}
	Our feet \textbf{shall} stand in thy gates * O Jerusa\textbf{lem}.
	
	Jerusalem is built \textbf{as} a city * that is at unity in it\textbf{self}.
	
	For thither the tribes go up, even \textbf{the} tribes of the Lord * to testify unto Israel, to give thanks unto the Name of the \textbf{Lord}.
	
	For there is the \textbf{seat} of judgment * even the seat of the house of Da\textbf{vid}.
	
	O pray for the peace \textbf{of} Jerusalem * they shall prosper who love \textbf{thee}.
	
	Peace \textbf{be} within thy walls * and plenteousness within thy pala\textbf{ces}.
	
	For my brethren \textbf{and} companions' sakes * I will wish thee prosperi\textbf{ty}.
	
	Yea, because of the house \textbf{of} the Lord our God * I will seek to do thee \textbf{good}.
	
	Glory be to the Fa\textbf{ther}, and to the Son * and to the Holy Spi\textbf{rit}.
	
	As it was in the beginning, is now, and \textbf{ev}er shall be * world without end. A\textbf{men}.
\end{multicols}

\repant{ant/in_domum_domini.gabc}


\veant{2}{viii. i}
\skiplines{1}\gregorioscore{ant/qui_habitas.gabc}

\greannotation{\veformat{Ps. 122}}
\greannotation{\veformat{viii. i.}}
\skiplines{1}\gabcsnippet{
	(c4) Un(g)to(h) (:?) thee(jr) I() lift(j) <b>up</b>(k) mine(jr) eyes(j) <sp>*</sp>(:) O(jr) thou() that() dwellest(j) <b>in</b>(i) the(j) hea(h)vens.(gr g)
}

\begin{multicols}{2}
	Behold, even as the eyes of \textbf{ser}vants * look unto the hand \textbf{of} their masters.
	
	And as the eyes of a maiden unto the hand of her \textbf{mis}tress * even so our eyes wait upon the Lord our God, until he have mer\textbf{cy} upon us.
	
	Have mercy upon us, O Lord, have mercy up\textbf{on} us * for we are ut\textbf{ter}ly despised.
	
	Our soul is filled with the scornful reproof of the \textbf{weal}thy * and with the despite\textbf{ful}ness of the proud.
	
	Glory be to the Father, and \textbf{to} the Son * and to the \textbf{Ho}ly Spirit.
	
	As it was in the beginning is now, and ever \textbf{shall} be * world with\textbf{out} end. Amen.
\end{multicols}

\repant{ant/qui_habitas.gabc}



\veant{3}{i. iii}
\skiplines{1}\gregorioscore{ant/adjutorium_nostrum.gabc}

\greannotation{\veformat{Ps. 123}}
\greannotation{\veformat{i. iii.}}
\skiplines{1}\gabcsnippet{
	(c4) If(f) the(g) (:?) Lord(hr) had() not() been() on() our() side,() now() let() Is()ra(h)<b>el</b>(g) say(h) <sp>*</sp>(:) if(hr) the() Lord() himself() had() not() been() on() our() side,() when() men() rose(h) <b>up</b>(g) a(f)gainst(g gr) us.(ghg)
}

\begin{multicols}{2}
	They had swallowed us \textbf{up} quick * when they were so wrathful\textbf{ly} displeased at us.
	
	Truly, the waters had \textbf{drowned} us * and the stream had gove \textbf{o}ver our soul.
	
	The deep waters of \textbf{the} proud * had gone even \textbf{o}ver our soul.
	
	But praised be \textbf{the} Lord * who hath not given us over for a \textbf{prey} unto their teeth.
	
	Our soul is escaped even as a bird out of the snare of the \textbf{fow}ler * the snare is broken, and we \textbf{are} delivered.
	
	Our help standeth in the Name of \textbf{the} Lord * who hath made \textbf{hea}ven and earth.
	
	Glory be to the Father, and to \textbf{the} Son * and to \textbf{the} Holy Spirit.
	
	As it was in the beginning, is now, and ever \textbf{shall} be * world with\textbf{out} end. Amen.
\end{multicols}

\repant{ant/adjutorium_nostrum.gabc}




\veant{4}{viii. ii}
\skiplines{1}\gregorioscore{ant/benefac_domine.gabc}

\greannotation{\veformat{Ps. 124}}
\greannotation{\veformat{viii. ii.}}
\skiplines{1}\gabcsnippet{
	(c3) They(e) that(f) (:?) put(hr) their() trust() in() the() Lord() shall() be() even() as() the() mount(h) <b>Zi</b>(i)on(hr h) <sp>*</sp>(:) which(hr) may() not() be() removed,() but stand()eth(h) <b>fast</b>(f) for(h)ev(i)er.(hr h)
}

\begin{multicols}{2}
	The hills stand about Je\textbf{ru}salem * even so standeth the Lord round about his people, from this time \textbf{forth} forevermore.
	
	For the rod of the ungodly cometh not into the lot of the \textbf{right}eous * lest the righteous put their hand \textbf{un}to wickedness.
	
	Do well, \textbf{O} Lord * unto those that are \textbf{good} and true of heart.
	
	As for such as turn back unto their own \textbf{wick}edness * the Lord shall lead them forth with the evildoers; but peace shall be \textbf{up}on Israel.
	
	Glory be to the Father, and \textbf{to} the Son * and to the \textbf{Ho}ly Spirit.
	
	As it was in the beginning is now, and ever \textbf{shall} be * world with\textbf{out} end. Amen.
\end{multicols}

\repant{ant/benefac_domine.gabc}



\veant{5}{i. iv}
\skiplines{1}\gregorioscore{ant/facti_sumus.gabc}

\greannotation{\veformat{Ps. 125}}
\greannotation{\veformat{i. iv.}}
\skiplines{1}\gabcsnippet{
	(c4) When(f) the(g) (:?) Lord(hr) turned() again() the() captivity() of(h) <b>Zi</b>(g)on(h) <sp>*</sp>(:) then(hr) were() we() like(h) <b>un</b>(g)to(f) them(g) that(gr) dream.(h)
}

\begin{multicols}{2}
	Then was our mouth filled with \textbf{laugh}ter * \textbf{and} our tongue with joy.
	
	Then said they among the \textbf{hea}then * The Lord hath \textbf{done} great things for them.
	
	Truly, the Lord hath done great things for us al\textbf{rea}dy * \textbf{where}of we rejoice.
	
	Turn our captivity, \textbf{O} Lord * as the \textbf{ri}vers in the south.
	
	They that sow \textbf{in} tears * \textbf{---} shall reap in joy.
	
	He that now goeth on his way weeping, and beareth forth \textbf{good} seed * shall doubtless come again with joy, and bring \textbf{his} sheaves with him.
	
	Glory be to the Father, and to \textbf{the} Son * and to \textbf{the} Holy Spirit.
	
	As it was in the beginning, is now, and ever \textbf{shall} be * world with\textbf{out} end. Amen.
\end{multicols}

\repant{ant/facti_sumus.gabc}
		
		\vechapter{Wednesday at Vespers}
		
			\veant{1}{viii. ii}
\gregorioscore{ant/beatus_vir.gabc}

\greannotation{\veformat{Ps. 126}}
\greannotation{\veformat{viii. ii.}}
\skiplines{1}\gabcsnippet{
	(c3) Ex(e)cept(f) (:?) the(hr) Lord(h) <b>build</b>(i) the(hr) house() <sp>*</sp>(:) their(hr) labor() is() but(h) <b>lost</b>(f) that(h) build(i) it.(hr)
}

\begin{multicols}{2}
	Except the Lord keep the \textbf{ci}ty * the watchman \textbf{wa}keth but in vain.

	It is but lost labor that ye haste to rise up early, and so late take rest, and eat the bread of \textbf{care}fulness * for so he giveth \textbf{his} beloved sleep.
	
	Lo, children and the fruit \textbf{of} the womb * are the heritage and gift that \textbf{com}eth of the Lord.
	
	Like as arrows in the hand of the \textbf{gi}ant * even so are \textbf{the} young children.
	
	Happy is the man that hath his quiver \textbf{full} of them * they shall not be ashamed when they speak with their e\textbf{ne}mies in the gate.
	
	Glory be to the Father, and \textbf{to} the Son * and to the \textbf{Ho}ly Spirit.
	
	As it was in the beginning is now, and ever \textbf{shall} be * world with\textbf{out} end. Amen.
\end{multicols}

\repant{ant/beatus_vir.gabc}

\pagebreak[0]

\veant{2}{viii. ii}
\skiplines{1}\gregorioscore{ant/beati_omnes.gabc}

\greannotation{\veformat{Ps. 127}}
\greannotation{\veformat{viii. ii.}}
\skiplines{0}\gabcsnippet{
	(c3) Bles(e)sed(f) (:?) are(hr) all() they() that(h) <b>fear</b>(i) the(hr) Lord() <sp>*</sp>(:) (hr) <b>and</b>(f) walk(h) in(i) his(hr) ways.()
}

\begin{multicols}{2}
	For thou shalt eat the labors \textbf{of} thine hands * O well is thee, and \textbf{hap}py shalt thou be.
	
	Thy wife shall be as the \textbf{fruit}ful vine * upon \textbf{the} walls of thine house.
	
	Thy children like the olive \textbf{bran}ches * round a\textbf{bout} thy table.
	
	Lo, thus shall the man be \textbf{bles}sed * that \textbf{fear}eth the Lord.
	
	The Lord from out of Zion shall so \textbf{bless} thee * that thou shalt see Jerusalem in prosperity \textbf{all} thy life long.
	
	Yea, that thou shalt see thy children's \textbf{child}ren * and peace \textbf{up}on Israel.
	
	Glory be to the Father, and \textbf{to} the Son * and to the \textbf{Ho}ly Spirit.
	
	As it was in the beginning is now, and ever \textbf{shall} be * world with\textbf{out} end. Amen.
\end{multicols}

\repant{ant/beati_omnes.gabc}

\pagebreak[0]

\veant{3}{viii. ii}
\skiplines{1}\gregorioscore{ant/benediximus_vobis.gabc}

\greannotation{\veformat{Ps. 128}}
\greannotation{\veformat{viii. ii.}}
\skiplines{1}\gabcsnippet{
	(c3) Ma(e)ny(f) (:?) a(hr) time() have() they() fought() against() me() from() my(h) <b>youth</b>(i) up(hr) <sp>*</sp>(:) may(hr) <b>Is</b>(f)ra(h)el(i) now(hr) say.()
}

\begin{multicols}{2}
	Yea, many a time have they vexed me from my \textbf{youth} up * but they have not pre\textbf{vailed} against me.
	
	The plowers plowed up\textbf{on} my back * and \textbf{made} long furrows.
	
	But the \textbf{right}eous Lord * hath hewn the snare of the ungod\textbf{ly} in pieces.
	
	Let them be confounded and turned \textbf{back}ward * as many as have evil \textbf{will} at Zion.
	
	Let them be even as the grass growing upon the \textbf{house}tops * which withereth afore \textbf{it} be plucked up.
	
	Whereof the mower filleth \textbf{not} his hand * neither he that bindeth up the sheaves \textbf{in} his bosom.
	
	So that they who go by say not so much as, The blessing of the Lord be \textbf{with} you * we bless you in \textbf{the} Name of the Lord.
	
	Glory be to the Father, and \textbf{to} the Son * and to the \textbf{Ho}ly Spirit.
	
	As it was in the beginning is now, and ever \textbf{shall} be * world with\textbf{out} end. Amen.
\end{multicols}

\repant{ant/benediximus_vobis.gabc}

\pagebreak[0]


\veant{4}{viii. iii}
\skiplines{1}\gregorioscore{ant/de_profundis.gabc}

\greannotation{\veformat{Ps. 129}}
\greannotation{\veformat{viii. iii.}}
\skiplines{1}\gabcsnippet{
	(c3) Out(e) of(f) (:?) the(hr) deep() have() I() called() un()to(h) <b>thee</b>,(i) O(hr) Lord() <sp>*</sp>(:) (hr) <b>Lord</b>(g) hear(f) my(hr) voice.()
}

\begin{multicols}{2}
	O let thine ears con\textbf{si}der well * the \textbf{voice} of my complaint.
	
	If thou, Lord, wilt be extreme to mark what is \textbf{done} amiss * O Lord, who \textbf{may} abide it?
	
	For there is mercy \textbf{with} thee * and therefore shalt \textbf{thou} be feared.
	
	I look for the Lord; my soul doth \textbf{wait} for him * in his \textbf{word} is my trust.
	
	My soul fleeth un\textbf{to} the Lord * before the morning watch, I say, be\textbf{fore} the morning watch.
	
	O Israel, trust in the Lord, for with hte Lord there is \textbf{mer}cy * and with him is plen\textbf{teous} redemption.
	
	And he shall redeem \textbf{Is}rael * \textbf{from} all his sins.
	
	Glory be to the Father, and \textbf{to} the Son * and to the \textbf{Ho}ly Spirit.
	
	As it was in the beginning is now, and ever \textbf{shall} be * world with\textbf{out} end. Amen.
\end{multicols}

\repant{ant/de_profundis.gabc}

\pagebreak[0]


\veant{5}{iv. vii}
\skiplines{1}\gregorioscore{ant/speret_israel.gabc}

\greannotation{\veformat{Ps. 130}}
\greannotation{\veformat{iv. vii.}}
\skiplines{1}\gabcsnippet{
	(c4) Lord,(hr) I() am(h) <b>not</b>(g) high(h) mind(i)ed(hr) <sp>*</sp>(:) I(hr) have() no(h) <b>proud</b>(g) looks.(h)
}

\begin{multicols}{2}
	I do not exercise myself \textbf{in} great matters * which are too high for me.
	
	But I refrain my soul, and keep it low, like as a child that is weaned \textbf{from} his mother * yea, my soul is even as a weaned child.
	
	O Isra\textbf{el}, trust in the Lord * from this time forth forevermore.
	
	Glory be to the Fa\textbf{ther}, and to the Son * and to the Holy Spirit.
	
	As it was in the beginning, is now, and \textbf{ev}er shall be * world without end. Amen.
\end{multicols}

\skiplines{-1}\repant{ant/speret_israel.gabc}

\instruct{Turn back to the \whiteribbon for the chapter.}
		
		\vechapter{Thursday at Vespers}
		
			\veant{1}{iv. vii}
\gregorioscore{ant/et_omnis.gabc}

\greannotation{\veformat{Ps. 131}}
\greannotation{\veformat{iv. vii.}}
\skiplines{1}\gabcsnippet{
	(c4) Lord,(hr) re(h)<b>mem</b>(g)ber(h) Da(i)vid(hr) <sp>*</sp>(:) and(hr) all() his(h) <b>trou</b>(g)ble.(h)
}

\begin{multicols}{2}
	How he \textbf{sware} unto the Lord * and vowed a vow unto the almighty God of \textbf{Ja}cob.
	
	I will not come within the taber\textbf{na}cle of mine house * nor climb up into \textbf{my} bed.
	
	I will not suffer mine eyes to sleep, nor mine eye\textbf{lids} to slumber * neither the temples of my head to take a\textbf{ny} rest.
	
	Until I find out a place for the \textbf{tem}ple of the Lord * a habitation for the mighty God of \textbf{Ja}cob.
	
	Lo, we heard of the \textbf{same} at Ephrata * and found it in \textbf{the} wood.
	
	We will go into his \textbf{ta}bernacle * and fall low on our knees before his \textbf{foot}stool.
	
	Arise, O Lord, in\textbf{to} thy resting place * thou, and the ark of \textbf{thy} strength.
	
	Let thy priests be \textbf{clothed} with righteousness * and let thy saints sing with joy\textbf{ful}ness.
	
	For thy \textbf{ser}vant David's sake * turn not away the presence of thine A\textbf{noin}ted\footnote{The Latin reads \textit{Christi tui.}}.
	
	The Lord hath made a faithful oath \textbf{un}to David * and he shall not shrink \textbf{from} it.
	
	Of the fruit \textbf{of} thy body * shall I set upon \textbf{thy} seat.
	
	If thy children will keep my covenant, and my testimonies that \textbf{I} shall learn them * their children also shall sit upon thy throne fore\textbf{ver}more.
	
	For the Lord hath chosen Zion to be a habi\textbf{ta}tion for himself * he hath longed \textbf{for} her.
	
	This shall be my \textbf{rest} forever * here will I dwell, for I have a delight \textbf{there}in.
	
	I will bless her victu\textbf{als} with increase * and will satisfy her poor \textbf{with} bread.
	
	I will deck her priests \textbf{with} salvation * and her saints shall rejoice \textbf{and} sing.
	
	There shall I make the horn of Da\textbf{vid} to flourish * I have ordained a lantern for mine A\textbf{noin}ted\footnote{Likewise here \textit{Christo meo.}}.
	
	As for her enemies, I \textbf{shall} clothe them with shame * but upon himself shall his crown \textbf{flour}ish.
	
	Glory be to the Fa\textbf{ther}, and to the Son * and to the Holy \textbf{Spir}it.
	
	As it was in the beginning, is now, and \textbf{ev}er shall be * world without end. \textbf{A}men.
\end{multicols}

\repant{ant/et_omnis.gabc}


\pagebreak[0]


\veant{2}{i. iv}
\skiplines{1}\gregorioscore{ant/ecce_quam_bonum.gabc}

\greannotation{\veformat{Ps. 132}}
\greannotation{\veformat{i. iv.}}
\skiplines{1}\gabcsnippet{
	(c4) Be(f)hold(g) (:?) how(hr) good() and() joyful() a() thing(h) <b>it</b>(g) is(h) <sp>*</sp>(:) for(hr) brethren() to() dwell() to(h)ge(h)<b>ther</b>(g) in(f) uni(gr)ty.(h)
}

\begin{multicols}{2}
	It is like the precious oil upon the head, that ran down unto \textbf{the} beard * even unto Aaron's beard, and went down to the skirts \textbf{of} his clothing.
	
	Like as the dew of \textbf{Her}mon * which fell upon the \textbf{hill} of Zion.
	
	For there the Lord promised his \textbf{bles}sing * and \textbf{life} forevermore.
	
	Glory be to the Father, and to \textbf{the} Son * and to the \textbf{Ho}ly Spirit.
	
	As it was in the beginning, is now, and ever \textbf{shall} be * world with\textbf{out} end. Amen.
\end{multicols}

\skiplines{-1}\repant{ant/ecce_quam_bonum.gabc}


\pagebreak[0]

\veant{3}{iii. vi}
\skiplines{1}\gregorioscore{ant/omnia_quecunque.gabc}

\greannotation{\veformat{Ps. 134}}
\greannotation{\veformat{iii. vi.}}
\skiplines{1}\gabcsnippet{
	(c4) O(g) praise(h) (:?) the(jr) Lord,() laud() ye() the(j) <b>Name</b>(k) of(jr) <b>the</b>(i) Lord(jr) <sp>*</sp>(:) praise(jr) it,() O(j) <b>ye</b>(h) ser(j)vants(i) of(h) the(gr) Lord.()
}

\begin{multicols}{2}
	Ye who stand in the \textbf{house} of \textbf{the} Lord * in the courts \textbf{of} the house of our God.
	
	O praise the Lord, for the \textbf{Lord} is \textbf{gra}cious * O sing praises unto his Name, \textbf{for} it is lovely.
	
	For why? the Lord hath chosen \textbf{Ja}cob un\textbf{to} himself * and Israel for \textbf{his} own possession.
	
	For I \textbf{know} that the \textbf{Lord} is great * and that our \textbf{Lord} is above all gods.
	
	Whatsoever the Lord pleased, that did he in \textbf{hea}ven, and \textbf{in} earth * and in the sea, and \textbf{in} all deep places.
	
	He bringeth forth the clouds from the \textbf{ends} of \textbf{the} world * and sendeth forth lightnings with the rain, bringing the winds \textbf{out} of his treasures.
	
	He smote the \textbf{first}born of \textbf{E}gypt * from man \textbf{ev}en unto the beasts.
	
	He hath sent tokens and wonders into the midst of thee, O thou \textbf{land} of \textbf{E}gypt * upon Pharaoh, \textbf{and} all his servants.
	
	He smote \textbf{di}verse \textbf{na}tions * \textbf{and} slew mighty kings.
	
	Sehon king of the Amorites; and Og the \textbf{king} of \textbf{Ba}san * and all the \textbf{king}doms of Canaan.
	
	And gave their land to \textbf{be} a \textbf{her}itage * even a heritage unto Is\textbf{ra}el his people.
	
	Thy Name, O Lord, en\textbf{dur}eth for \textbf{ev}er * so doth thy memorial, O Lord, from one genera\textbf{tion} to another.
	
	For the Lord will a\textbf{venge} his \textbf{peo}ple * and be gracious \textbf{un}to his servants.
	
	As for the images of the heathen, they are but \textbf{sil}ver \textbf{and} gold * \textbf{the} work of men's hands.
	
	They have \textbf{mouths}, and \textbf{speak} not * eyes have \textbf{they}, but they see not.
	
	They have ears, and \textbf{yet} they \textbf{hear} not * neither is there \textbf{a}ny breath in their mouths.
	
	They who make them are \textbf{like} un\textbf{to} them * and so are all they \textbf{who} put their trust in them.
	
	Praise the Lord, ye \textbf{house} of \textbf{Is}rael * praise the Lord, \textbf{ye} house of Aaron.
	
	Praise the Lord, ye \textbf{house} of \textbf{Le}vi * ye that \textbf{fear} the Lord, praise the Lord.
	
	Praised be the Lord \textbf{out} of \textbf{Zi}on * who dwel\textbf{leth} at Jerusalem.
	
	Glory be to the \textbf{Fa}ther, and \textbf{to} the Son * and to \textbf{the} Holy Spirit.
	
	As it was in the beginning, is now, and \textbf{ev}er \textbf{shall} be * world \textbf{with}out end. Amen.
\end{multicols}

\repant{ant/omnia_quecunque.gabc}

\pagebreak[0]

\veant{4}{iii. vi}
\skiplines{1}\gregorioscore{ant/quoniam_in_eternum.gabc}

\greannotation{\veformat{Ps. 135}}
\greannotation{\veformat{iii. vi.}}
\skiplines{1}\gabcsnippet{
	(c4) O(g) give(h) (:?) thanks(jr) unto() the() Lord,() for(j) <b>he</b>(k) is(jr) <b>gra</b>(i)cious(jr) <sp>*</sp>(:) and(jr) his() mercy() en(j)<b>dur</b>(h)eth(j) for(i)ev(h)er.(gr)
}

\begin{multicols}{2}
	O give thanks unto the \textbf{God} of \textbf{all} gods * for his mercy en\textbf{dur}eth forever.
	
	O thank the \textbf{Lord} of \textbf{all} lords * for his mercy en\textbf{dur}eth forever.
	
	Who alone \textbf{do}eth great \textbf{won}ders * for his mercy en\textbf{dur}eth forever.
	
	Who by his excellent wisdom \textbf{made} the \textbf{hea}vens * for his mercy en\textbf{dur}eth forever.
	
	Who laid out the earth a\textbf{bove} the \textbf{wa}ters * for his mercy en\textbf{dur}eth forever.
	
	Who \textbf{hath} made \textbf{great} lights * for his mercy en\textbf{dur}eth forever.
	
	The \textbf{sun} to \textbf{rule} the day * for his mercy en\textbf{dur}eth forever.
	
	The moon and the stars to \textbf{gov}ern \textbf{the} night * for his mercy en\textbf{dur}eth forever.
	
	Who smote Egypt, \textbf{with} their \textbf{first}born * for his mercy en\textbf{dur}eth forever.
	
	And brought Israel \textbf{from} a\textbf{mong} them * for his mercy en\textbf{dur}eth forever.
	
	With a mighty \textbf{hand} and \textbf{streched} out arm * for his mercy en\textbf{dur}eth forever.
	
	Who divided the \textbf{Red} sea in \textbf{two} parts * for his mercy en\textbf{dur}eth forever.
	
	And made Israel to go \textbf{through} the \textbf{midst} of it * for his mercy en\textbf{dur}eth forever.
	
	But as for Pharaoh and his host, he overthrew them \textbf{in} the \textbf{Red} sea * for his mercy en\textbf{dur}eth forever.
	
	Who lead his people \textbf{through} the \textbf{wild}erness * for his mercy en\textbf{dur}eth forever.
	
	Who \textbf{smote} the \textbf{great} kings * for his mercy en\textbf{dur}eth forever.
	
	Yea, and slew \textbf{migh}ty kings * for his mercy en\textbf{dur}eth forever.
	
	Sehon \textbf{king} of the \textbf{A}morites * for his mercy en\textbf{dur}eth forever.
	
	And Og the \textbf{king} of \textbf{Ba}san * for his mercy en\textbf{dur}eth forever.
	
	And gave away their land \textbf{for} a \textbf{her}itage * for his mercy en\textbf{dur}eth forever.
	
	Even for a heritage unto Isra\textbf{el} his \textbf{ser}vant * for his mercy en\textbf{dur}eth forever.
	
	Who remembered us when we \textbf{were} in \textbf{trou}ble * for his mercy en\textbf{dur}eth forever.
	
	And hath delivered us \textbf{from} our \textbf{en}emies * for his mercy en\textbf{dur}eth forever.
	
	Who giveth \textbf{food} to \textbf{all} flesh * for his mercy en\textbf{dur}eth forever.
	
	O give thanks unto the \textbf{God} of \textbf{hea}ven * for his mercy en\textbf{dur}eth forever.
	
	O give thanks un\textbf{to} the \textbf{Lord} of lords * for his mercy en\textbf{dur}eth forever.
	
	Glory be to the \textbf{Fa}ther, and \textbf{to} the Son * and to \textbf{the} Holy Spirit.
	
	As it was in the beginning, is now, and \textbf{ev}er \textbf{shall} be * world \textbf{with}out end. Amen.
\end{multicols}

\clearpage\skiplines{-1}\repant{ant/quoniam_in_eternum.gabc}

\veant{5}{viii. i}
\skiplines{1}\gregorioscore{ant/hymnum_cantate.gabc}

\greannotation{\veformat{Ps. 136}}
\greannotation{\veformat{viii. i.}}
\skiplines{1}\gabcsnippet{
	(c3) By(e) the(f) (:?) wa(h)ters(hr) of() Babylon() we() sat(h) <b>down</b>(i) and(hr) wept() <sp>*</sp>(:) when(hr) we() remembered(h) <b>thee</b>,(g) O(h) Zi(f)on.(er)
}

\begin{multicols}{2}
	As for our harps, we \textbf{hanged} them up * upon the trees \textbf{that} are therein.
	
	For they who led us away captive required of us then a song and melody in our \textbf{hea}viness * Sing us one of the \textbf{songs} of Zion.
	
	How shall we sing the \textbf{Lord's} song * \textbf{in} a strange land?
	
	If I forget thee, O Je\textbf{ru}salem * let my right hand for\textbf{get} her cunning.
	
	If I do not remember thee, let my tongue cleave to the roof \textbf{of} my mouth * yea, if I prefer not Jerusalem a\textbf{bove} my chief joy.
	
	Remember the children of Edom, O Lord, in the day of Je\textbf{ru}salem * how they said, Down with it, down it, \textbf{ev}en to the ground.
	
	O daughter of Babylon, wasted with \textbf{mis}ery * yea, happy shall he be who rewardeth thee, as \textbf{thou} hast served us.
	
	Blessed shall he be who taketh thy \textbf{child}ren * and throweth \textbf{them} against the stones.
	
	Glory be to the Father, and \textbf{to} the Son * and to the \textbf{Ho}ly Spirit.
	
	As it was in the beginning is now, and ever \textbf{shall} be * world with\textbf{out} end. Amen.
\end{multicols}

\skiplines{-1}\repant{ant/hymnum_cantate.gabc}

\instruct{Turn back to the \whiteribbon for the chapter.}
		
		\vechapter{Friday at Vespers}
		
			\veant{1}{v. i}
\gregorioscore{ant/in_conspectu_angelorum.gabc}

\greannotation{\veformat{Ps. 137}}
\greannotation{\veformat{v. i.}}
\skiplines{1}\gabcsnippet{
	(c4) I(f) will(h) (:?) give(jr) thanks() unto() thee,() O() Lord,() with() my(j) <b>whole</b>(k) heart(jr j) <sp>*</sp>(:) e(jr)ven() before() the() gods() will() I() sing(jr) <b>praise</b>(k) un(ir)<b>to</b>(j) thee.(hr h) 
}

\begin{multicols}{2}
	I will worship toward thy holy temple, and praise thy Name, because of thy loving-kindness \textbf{and} truth * for thou hast magnified thy Name, and thy \textbf{Word} a\textbf{bove} all things.
	
	When I called upon thee, thou \textbf{heard}est me * and enduedst my \textbf{soul} with \textbf{much} strength.
	
	All the kings of the earth shall praise \textbf{thee}, O Lord * for they have heard the \textbf{words} of \textbf{thy} mouth.
	
	Yea, they shall sing of the ways \textbf{of} the Lord * that great is the \textbf{glo}ry \textbf{of} the Lord.
	
	For though the Lord be high, yet hath he respect unto the \textbf{low}ly * as for the proud, he beholdeth \textbf{them} a\textbf{far} off.
	
	Though I walk in the midst of trouble, yet shalt thou re\textbf{fresh} me * thou shalt stretch forth thy hand upon the furiousness of mine enemies, and thy \textbf{right} hand shall \textbf{save} me.
	
	The Lord shall make good his loving-kindness to\textbf{ward} me * yea, thy mercy, O Lord, endureth forever; despise not then the \textbf{works} of thine \textbf{own} hands.
	
	Glory be to the Father, and \textbf{to} the Son * and to the \textbf{Ho}ly \textbf{Spir}it.
	
	As it was in the beginning, is now, and ever \textbf{shall} be * world with\textbf{out} end. \textbf{A}men.
\end{multicols} 

\repant{ant/in_conspectu_angelorum.gabc}

\pagebreak[0]

\veant{2}{iii. vi}
\skiplines{1}\gregorioscore{ant/domine_probasti.gabc}

\greannotation{\veformat{Ps. 138}}
\greannotation{\veformat{iii. vi.}}
\skiplines{1}\gabcsnippet{
	(c4) O(g) Lord,(h) (:?) thou(jr) hast() searched() me(j) <b>out</b>,(k) and(jr) <b>known</b>(i) me(jr j) <sp>*</sp>(:) thou(jr) knewest() my down()-sitting,() and() mine() uprising;() thou un()der()stand(j)<b>est</b>(h) my(j) thoughts(i) long(h) be(gr)fore.(g)
}

\begin{multicols}{2}
	Thou art about \textbf{my} path, and a\textbf{bout} my bed * and \textbf{spi}est out all my ways.
	
	For lo, there is not a \textbf{word} in \textbf{my} tongue * but thou, O Lord, knowest \textbf{it} altogether.
	
	Thou hast fashioned me be\textbf{hind} and \textbf{be}fore * and laid \textbf{thine} hand upon me.
	
	Such knowledge is too wonderful and \textbf{ex}cellent \textbf{for} me * I can\textbf{not} attain unto it.
	
	If I climb up into \textbf{hea}ven, \textbf{thou} art there * if I go down to hell, \textbf{thou} art there also.
	
	If I take the wings \textbf{of} the \textbf{morn}ing * and remain in the ut\textbf{ter}most parts of the sea.
	
	Even there also shall \textbf{thy} hand \textbf{lead} me * and thy \textbf{right} hand shall hold me.
	
	If I say, Now the \textbf{dark}ness shall \textbf{co}ver me * then shall \textbf{my} night be turned to day.
	
	Yea, the darkness is no darkness with thee, but the night is as \textbf{clear} as \textbf{the} day * the darkness and light \textbf{to} thee are both alike.
	
	For my \textbf{reins} are thine * thou hast covered \textbf{me} in my mother's womb.
	
	I will give thanks unto thee, for I am fearfully and \textbf{won}der\textbf{ful}ly made * marvelous are thy works, and that my \textbf{soul} knoweth right well.
	
	My bones \textbf{are} not \textbf{hid} from thee * though I be made secretly, and fa\textbf{shioned} beneath in the earth.
	
	Thine eyes did see my substance, yet \textbf{be}ing im\textbf{per}fect * and in thy book were all \textbf{my} members written.
	
	Which day by \textbf{day} were \textbf{fa}shioned * when as \textbf{yet} there were none of them.
	
	How dear are thy counsels \textbf{un}to \textbf{me}, O God * O how \textbf{great} is the sum of them.
	
	If I tell them, they are more in \textbf{num}ber \textbf{than} the sand * when I wake up, I \textbf{am} present with thee.
	
	Wilt thou not slay the \textbf{wick}ed, \textbf{O} God? * depart from \textbf{me}, ye bloodthirsty men.
	
	For they speak unrighteous\textbf{ly} a\textbf{gainst} thee * and thine ene\textbf{mies} take thy Name in vain.
	
	Do not I hate them, O \textbf{Lord}, who \textbf{hate} thee * and am not I grieved with those who \textbf{rise} up against thee?
	
	Yea, I \textbf{hate} them \textbf{right} sore * even as though \textbf{they} were mine enemies.
	
	Try me, O God, and seek the \textbf{ground} of \textbf{my} heart * prove me, and \textbf{ex}amine my thoughts.
	
	Look well if there be any way of \textbf{wick}edness \textbf{in} me * and lead me in the \textbf{way} everlasting.
	
	Glory be to the \textbf{Fa}ther, and \textbf{to} the Son * and to \textbf{the} Holy Spirit.
	
	As it was in the beginning, is now, and \textbf{ev}er \textbf{shall} be * world \textbf{with}out end. Amen.
\end{multicols}

\repant{ant/domine_probasti.gabc}

\pagebreak[0]


\veant{3}{iv. vi}
\skiplines{0}\gregorioscore{ant/a_viro_iniquo.gabc}

\greannotation{\veformat{Ps. 139}}
\greannotation{\veformat{iv. vi.}}
\skiplines{1}\gabcsnippet{
	(c4) De(h)li(g:?)ver(hr) me,() O() Lord,(h) <b>from</b>(g) the(h) e(i)vil(hr) man(h) <sp>*</sp>(:) and(hr) preserve() me() from() the() wicked(h) <b>man</b>.(g)
}

\begin{multicols}{2}
	Who imagine \textbf{mis}chief in their hearts * and stir up strife all the day \textbf{long}.
	
	They have sharpened their tongues \textbf{like} a serpent * adder's poison is under their \textbf{lips}.
	
	Keep me, O Lord, from the hands of \textbf{the} ungodly * preserve me from the wicked men, who are purposed to overthrow my go\textbf{ings}.
	
	The proud have laid a snare for me, and spread a \textbf{net} abroad with cords * yea, and set traps in my \textbf{way}.
	
	I said unto the \textbf{Lord}, Thou art my God * hear the voice of my prayers, O \textbf{Lord}.
	
	O Lord God, thou strength of \textbf{my} salvation * thou hast covered my head in the day of bat\textbf{tle}.
	
	Let not the ungodly have his de\textbf{si}re, O Lord * let not his mischievous imagination prosper, lest they be too \textbf{proud}.
	
	Let the mischief of their own lips fall up\textbf{on} the head of them * who compass me a\textbf{bout}.
	
	Let hot burning coals \textbf{fall} upon them * let them be cast into the fire, and into the pit, that they never rise up a\textbf{gain}.
	
	A man full of words shall not pros\textbf{per} upon the earth * evil shall hunt the wicked person to overthrow \textbf{him}.
	
	Sure I am that the Lord \textbf{will} avenge the poor * and maintain the cause of the help\textbf{less}.
	
	The righteous also shall give \textbf{thanks} unto thy Name * and the just shall continue in thy \textbf{sight}.
	
	Glory be to the Fa\textbf{ther}, and to the Son * and to the Holy Spi\textbf{rit}.
	
	As it was in the beginning, is now, and \textbf{ev}er shall be * world without end. A\textbf{men}.
\end{multicols}

\repant{ant/a_viro_iniquo.gabc}

\pagebreak[0]


\veant{4}{viii. ii}
\skiplines{1}\gregorioscore{ant/domine_clamavi.gabc}

\greannotation{\veformat{Ps. 140}}
\greannotation{\veformat{viii. ii.}}
\skiplines{1}\gabcsnippet{
	(c3) Lord,(e) I(f) (:?) call(hr) upon() thee,() haste() thee(h) <b>un</b>(i)to(hr) me(h) <sp>*</sp>(:) and(hr) consider() my() voice() when(h) <b>I</b>(f) cry(h) un(i)to(hr) thee.(h)
}

\begin{multicols}{2}
	Let my prayer be set forth in thy sight as the \textbf{in}cense * and let the lifting up of my hands be an \textbf{even}ing sacrifice.
	
	Set a watch, O Lord, be\textbf{fore} my mouth * and keep \textbf{the} door of my lips.
	
	O let not mine heart be inclined to any \textbf{e}vil thing * let me not be occupied in ungodly works with the men that work wickedness, neither let me eat of such \textbf{things} as please them.
	
	Let the righteous rather smite me \textbf{friend}ly * \textbf{and} reprove me.
	
	But let not their precious balms \textbf{break} my head * yea, I will pray yet a\textbf{gainst} their wickedness.
	
	Let their judges be overthrown in stony \textbf{pla}ces * that they may hear my \textbf{words}, for they are sweet.
	
	Our bones lie scattered be\textbf{fore} the pit * like as when one breaketh and heweth \textbf{wood} upon the earth.
	
	But mine eyes look unto thee, \textbf{O} Lord God * in thee is my trust, O \textbf{cast} not out my soul.
	
	Keep me from the snare that they have \textbf{laid} for me * and from the traps of the \textbf{wick}ed doers.
	
	Let the ungodly fall into their own net to\textbf{ge}ther * and let me ev\textbf{er} escape them.
	
	Glory be to the Father, and \textbf{to} the Son * and to the \textbf{Ho}ly Spirit.
	
	As it was in the beginning is now, and ever \textbf{shall} be * world with\textbf{out} end. Amen.
\end{multicols}

\repant{ant/domine_clamavi.gabc}


\pagebreak[0]

\veant{5}{viii. ii}
\skiplines{1}\gregorioscore{ant/portio_mea.gabc}

\greannotation{\veformat{Ps. 141}}
\greannotation{\veformat{viii. ii.}}
\skiplines{1}\gabcsnippet{
	(c3) I(e) cried(f) (:?) un(hr)to() the() Lord(h) <b>with</b>(i) my(hr) voice(h) <sp>*</sp>(:) yea,(hr) even() unto() the() Lord() did() I() make() my(h) <b>supp</b>(f)li(h)ca(i)tion.(hr h)
}

\begin{multicols}{2}
	I poured out my complaints be\textbf{fore} him * and shewed him \textbf{of} my trouble.
	
	When my spirit was in heaviness thou knewest \textbf{my} path * in the way wherein I walked have they privily \textbf{laid} a snare for me.
	
	I looked also upon my \textbf{right} hand * and saw there was no man \textbf{who} would know me.
	
	I had no place to flee \textbf{un}to * and no \textbf{man} cared for my soul.
	
	I cried unto thee, O \textbf{Lord}, and said * Thou art my hope, and my portion in the land \textbf{of} the living.
	
	Consider \textbf{my} complaint * for I \textbf{am} brought very low.
	
	O deliver me from my perse\textbf{cu}tors * for they \textbf{are} too strong for me.
	
	Bring my soul out of prison, that I may give thanks un\textbf{to} thy Name * which thing if thou wilt grant me, then shall the righteous resort un\textbf{to} my company.
	
	Glory be to the Father, and \textbf{to} the Son * and to the \textbf{Ho}ly Spirit.
	
	As it was in the beginning is now, and ever \textbf{shall} be * world with\textbf{out} end. Amen.
\end{multicols}

\repant{ant/portio_mea.gabc}

\instruct{Turn back to the \whiteribbon for the chapter.}
		
		\vechapter{Saturday at Vespers}
		
			\veant{1}{vi}
\gregorioscore{ant/benedictus_dominus.gabc}

\greannotation{\veformat{Ps. 143}}
\greannotation{\veformat{vi. i.}}
\skiplines{1}\gabcsnippet{
	(f3) Bles(hv)ed(iv) (:?) be(jrv) the() Lord(jv) <b>my</b>(i) strength(jv) <sp>*</sp>(:) who(jrv) teacheth() my() hands() to() war,() and() my(jv) <b>fin</b>(h)gers(ij) to(i) (hr) fight.(h)
}

\begin{multicols}{2}
	My hope and my fortress, my castle and deliverer, my defender in whom \textbf{I} trust * who subdueth my people \textbf{that} is under me.
	
	Lord, what is man that thou hast such respect un\textbf{to} him? * or the son of man, that thou \textbf{so} regardest him?
	
	Man is like a thing \textbf{of} nought * his time passeth away \textbf{like} a shadow.
	
	Bow thy heavens, O Lord, and \textbf{come} down * touch the moun\textbf{tains}, and they shall smoke.
	
	Cast forth thy lightning, and \textbf{tear} them * shoot out thine arrows, \textbf{and} consume them.
	
	Send down thine hand from \textbf{a}bove * deliver me, and take me out of the great waters, from the hand \textbf{of} strange children.
	
	Whose mouth talketh of van\textbf{i}ty * and their right hand is a right \textbf{hand} of wickedness.
	
	I will sing a new song unto thee, \textbf{O} God * and sing praises unto thee upon \textbf{a} ten-stringed lute.
	
	Thou hast given victory un\textbf{to} kings * and hast delivered David thy servant from the \textbf{pe}ril of the sword.
	
	Save me, and deliver me from the hand of strange \textbf{child}ren * whose mouth talketh of vanity, and their right hand is a right hand \textbf{of} iniquity.
	
	That our sons may grow up as the \textbf{young} plants * and that our daughters may be as the polished corners \textbf{of} the temple.
	
	That our garners may be full and plenteous with all manner \textbf{of} store * that our sheep may bring forth thousands and ten \textbf{thou}sands in our streets.
	
	That our oxen may be strong to labor, that there be no \textbf{de}cay * no leading into captivity, and no com\textbf{plain}ing in our streets.
	
	Happy are the people that are in such \textbf{a} case * yea, blessed are the people who have \textbf{the} Lord for their God.
	
	Glory be to the Father, and to \textbf{the} Son * and to \textbf{the} Holy Spirit.
	
	As it was in the beginning is now, and ever \textbf{shall} be * world with\textbf{out} end. Amen.
\end{multicols}

\skiplines{1}\repant{ant/benedictus_dominus.gabc}  \clearpage

\veant{2}{viii. iii}
\gregorioscore{ant/in_eternum.gabc}

\greannotation{\veformat{Ps. 144}}
\greannotation{\veformat{viii. iii.}}
\skiplines{1}\gabcsnippet{
	(c4) I(g) will(hv) (:?) mag(jrv)nify() thee,() O(jv) <b>God</b>,(kv) my(jr) King() <sp>*</sp>(:) and(jr) I() will() praise() thy() Name() for(j)ev(j)<b>er</b>(i) and(h) ev(jv)er.(jr)
}

\begin{multicols}{2}
	Every day will I give thanks \textbf{un}to thee * and praise thy Name forev\textbf{er} and ever.
	
	Great is the Lord, and marvelous, worthy \textbf{to} be praised * there is no end \textbf{of} his greatness.
	
	One generation shall praise thy works unto a\textbf{noth}er * and de\textbf{clare} thy power.
	
	As for me, I will be talking of thy \textbf{wor}ship * thy glory, thy \textbf{praise}, and wondrous works.
	
	So that men shall speak of the might of thy marve\textbf{lous} acts * and I will also tell \textbf{of} thy greatness.
	
	The memorial of thine abundant kindness \textbf{shall} be shewed * and men shall sing \textbf{of} thy righteousness.
	
	The Lord is gracious and \textbf{mer}ciful * long-suffering, and \textbf{of} great goodness.
	
	The Lord is loving unto \textbf{ev}ery man * and his mercy is \textbf{ov}er all his works.
	
	All thy works praise \textbf{thee}, O Lord * and thy saints \textbf{give} thanks unto thee.
	
	They shew the glory of thy \textbf{king}dom * and talk \textbf{of} thy power.
	
	That thy power, thy glory, and mightiness of thy \textbf{king}dom * might \textbf{be} known unto men.
	
	Thy kingdom is an everlasting \textbf{king}dom * and thy dominion endureth through\textbf{out} all ages.
	
	The Lord upholdeth all \textbf{such} as fall * and lifteth up all \textbf{those} that are down.
	
	The eyes of all wait upon \textbf{thee}, O Lord * and thou givest them their meat \textbf{in} due season.
	
	Thou openest \textbf{thine} hand * and fillest all things liv\textbf{ing} with plenteousness.
	
	The Lord is righteous in \textbf{all} his ways * and ho\textbf{ly} in all his works.
	
	The Lord is nigh unto all them that call up\textbf{on} him * yea, all such as call up\textbf{on} him faithfully.
	
	He will fulfill the desire of them that \textbf{fear} him * he also will hear their cry, \textbf{and} will help them.
	
	The Lord preserveth all that \textbf{love} him * but scattereth abroad all \textbf{the} ungodly.
	
	My mouth shall speak the praise \textbf{of} the Lord * and let all flesh give thanks unto his holy Name forev\textbf{er} and ever.
	
	Glory be to the Father, and \textbf{to} the Son * and to the \textbf{Ho}ly Spirit.
	
	As it was in the beginning is now, and ever \textbf{shall} be * world with\textbf{out} end. Amen.
\end{multicols}

\skiplines{1}\repant{ant/in_eternum.gabc} \clearpage

\veant{3}{iv. i}
\gregorioscore{ant/laudabo.gabc}

\greannotation{\veformat{Ps. 145}}
\greannotation{\veformat{iv. i.}}
\skiplines{1}\gabcsnippet{
	(c4) Praise(hv) the(g) (:?) Lord,(hrv) O() my() soul;() while() I() live,(hv) <b>will</b>(g) I(hv) praise(iv) the(hr) Lord <sp>*</sp>(:) yea,(hr) as() long() as() I() have() any() being,() I() will() sing() prai(h)<b>ses</b>(g) un(hv)to(ih) (gr) <b>my</b>(gvFE) God.(e)
}

\begin{multicols}{2}
	O put not your trust in princes, nor in \textbf{an}y son of man * for there \textbf{is} no help in them.
	
	For when the breath of man goeth forth he shall turn \textbf{a}gain to his earth * and then \textbf{all} his thoughts perish.
	
	Blessed is he that hath the God of \textbf{Ja}cob for his help * and whose hope \textbf{is} in the Lord his God.
	
	Who made heaven and earth, the sea, and \textbf{all} that therein is * who keepeth his \textbf{pro}mise forever.
	
	Who helpeth them to \textbf{right} that suffer wrong * who \textbf{feed}eth the hungry.
	
	The Lord looseth men \textbf{out} of prison * the Lord \textbf{giv}eth sight to the blind.
	
	The Lord helpeth them \textbf{that} are fallen * the Lord car\textbf{eth} for the righteous.
	
	The Lord careth for the strangers; he defendeth the father\textbf{less} and widow * as for the way of the ungodly, he turn\textbf{eth} it upside down.
	
	The Lord thy God, O Zion, shall be \textbf{King} forevermore * and throughout \textbf{all} generations.
	
	Glory be to the Fath\textbf{er}, and to the Son * and to \textbf{the} Holy \textbf{Spir}it.
	
	As it was in the beginning is now, and \textbf{ev}er shall be * world \textbf{with}out end. \textbf{A}men.
\end{multicols}

\skiplines{0.5}\repant{ant/laudabo.gabc}

\veant{4}{viii. iii}
\skiplines{1}\gregorioscore{ant/deo_nostro.gabc}

\greannotation{\veformat{Ps. 146}}
\greannotation{\veformat{viii. iii.}}
\skiplines{1}\gabcsnippet{
	(c3) O(e) praise(fv) (:?) the(hrv) Lord,() for() it() is() a() good() thing() to() sing() praises() un(hv)<b>to</b>(iv) our(hr) God(h) <sp>*</sp>(:) yea,(hr) a() joyful() and() pleasant() thing() it() is(h) <b>to</b>(g) be(f) thank(hrv)ful.(hv)
}

\begin{multicols}{2}
	The Lord doth build up Je\textbf{ru}salem * and gather together the out\textbf{casts} of Israel.
	
	He healeth those that are broken \textbf{in} heart * and giveth medicine to \textbf{heal} their sickness.
	
	He telleth the number \textbf{of} the stars * and calleth \textbf{them} by all their names.
	
	Great is our Lord, and great is his \textbf{pow}er * yea, and his wis\textbf{dom} is infinite.
	
	The Lord setteth \textbf{up} the meek * and bringeth down to the ground \textbf{the} ungodly.
	
	O sing unto the Lord with thanks\textbf{giv}ing * sing praises upon the harp un\textbf{to} our God.
	
	Who covereth the heavens with clouds, and prepareth rain \textbf{for} the earth * and maketh the grass to grow upon the mountains, and herb \textbf{for} the use of men.
	
	Who giveth fodder unto the \textbf{cat}tle * and feedeth the young ravens that \textbf{call} upon him.
	
	He hath no pleasure in the strength \textbf{of} a horse * neither delighteth he in \textbf{an}y man's legs.
	
	But the Lord's delight is in them that \textbf{fear} him * and put their trust \textbf{in} his mercy.
	
	Glory be to the Father, and \textbf{to} the Son * and to the \textbf{Ho}ly Spirit.
	
	As it was in the beginning is now, and ever \textbf{shall} be * world with\textbf{out} end. Amen.
\end{multicols}

\skiplines{1}\repant{ant/deo_nostro.gabc} \clearpage

\veant{5}{iv. vii}
\gregorioscore{ant/lauda_hierusalem.gabc}

\greannotation{\veformat{Ps. 147}}
\greannotation{\veformat{iv. vii.}}
\skiplines{1}\gabcsnippet{
	(c4) Praise(hv) the(g) (:?) Lord,(hrv) <b>O</b>(g) Je(hv)ru(iv)sa(hr)lem(h) <sp>*</sp>(:) praise(hr) thy() God,() O() <b>Zi</b>(g)on.(hv)
}

\begin{multicols}{2}
	For he hath made fast \textbf{the} bars of thy gates * and hath blest thy children with\textbf{in} thee.
	
	He maketh peace \textbf{in} thy borders * and filleth thee with the flour \textbf{of} wheat.
	
	He sendeth forth his command\textbf{ment} upon earth * and his word runneth very \textbf{swift}ly.
	
	He \textbf{giv}eth snow like wool * and scattereth the hoarfrost like \textbf{ash}es.
	
	He casteth forth his \textbf{ice} like morsels * who is able to abide \textbf{his} frost?
	
	He sendeth out his \textbf{word}, and melteth them * he bloweth his wind, and the wat\textbf{ers} flow.
	
	He sheweth his word \textbf{un}to Jacob * his statutes and ordinances unto Is\textbf{ra}el.
	
	He hath not dealt so with \textbf{an}y nation * neither have the heathen knowledge of \textbf{his} laws.
	
	Glory be to the Fath\textbf{er}, and to the Son * and to the Holy \textbf{Spir}it.
	
	As it was in the beginning, is now and \textbf{ev}er shall be * world without end. \textbf{A}men.
\end{multicols}

\repant{ant/lauda_hierusalem_2.gabc}

\instruct{Turn back to the \whiteribbon.}
		
		\vechapter{Compline}
		
			\greannotation{\veformat{Ant.}}
\greannotation{\veformat{viii. i.}}
\gregorioscore{ant/miserere_michi_domine.gabc}

\greannotation{\veformat{Ps. 4}}
\greannotation{\veformat{viii. i.}}
\skiplines{1}\gabcsnippet{
	(c4) Hear(gv) me(hv) (:?) when(jvr) I() call,() O() God() of() my(jv) <b>right</b>(kv)eous(jr)ness(j) <sp>*</sp>(:) thou(jr) hast() set() me() at() liberty() when() I() was() in() trouble;() have() mercy() upon() me,() and() hear(j)<b>ken</b>(i) un(jv)to(h) my(gr) prayer.(g)
}

\begin{multicols}{2}
	O ye sons of men, how long will ye blaspheme mine \textbf{hon}or? * and have such pleasure in vanity, and seek \textbf{af}ter lying?
	
	Know this also, that the Lord hath chosen to himself the man that is \textbf{god}ly * when I call upon the Lord, \textbf{he} will hear me.
	
	Stand in awe, and \textbf{sin} not * commune with your own heart, and in your \textbf{cham}ber, and be still.
	
	Offer the sacrifice of \textbf{right}eousness * and put your \textbf{trust} in the Lord.
	
	There be many \textbf{that} say * Who will \textbf{shew} us any good?
	
	Lord, \textbf{lift} thou up * the light of thy counte\textbf{nance} upon us.
	
	Thou hast put gladness \textbf{in} my heart * since the time that their corn, and wine, and \textbf{o}il increased.
	
	I will lay me down in peace, and \textbf{take} my rest * for it is thou, Lord, only, who makest me \textbf{dwell} in safety.
	
	Glory be to the Father, and \textbf{to} the Son * and to the \textbf{Ho}ly Spirit.
	
	As it was in the beginning is now, and ever \textbf{shall} be * world with\textbf{out} end. Amen.
\end{multicols}

\pagebreak[1]

\greannotation{\veformat{Ps. 90}}
\greannotation{\veformat{viii. i.}}
\skiplines{1}\gabcsnippet{
	(c4) Who(gv)so(hv) (:?) dwell(jrv)eth() under() the() defense() of() the(jv) <b>most</b>(kv jr) High(j) <sp>*</sp>(:) shall(jr) abide() under() the() shadow() of(j) <b>the</b>(i) Al(jv)migh(h gr)ty.(g)
}

\begin{multicols}{2}
	I will say unto the Lord, Thou art my hope, and my \textbf{strong}hold * my God, \textbf{in} him will I trust.
	
	For he shall deliver thee from the snare of the \textbf{hunt}er * and from the \textbf{noi}some pestilence.
	
	He shall defend thee under his wings, and thou shalt be safe under his \textbf{fea}thers * his faithfulness and truth shall be thy \textbf{shield} and buckler.
	
	Thou shalt not be afraid for any terror \textbf{by} night * nor for the arrow that \textbf{fli}eth by day.
	
	For the pestilence that walketh in \textbf{dark}ness * nor for the sickness that destroyeth \textbf{in} the noonday.
	
	A thousand shall fall beside thee, and ten thousand at thy \textbf{right} hand * but it shall \textbf{not} come nigh thee.
	
	Yea, with thine eyes shalt thou \textbf{be}hold * and see the reward of \textbf{the} ungodly.
	
	For thou, Lord, \textbf{art} my hope * thou hast set thine house of \textbf{de}fense very high.
	
	There shall no evil happen \textbf{un}to thee * neither shall any plague come \textbf{nigh} thy dwelling.
	
	For he shall give his angels charge \textbf{ov}er thee * to keep \textbf{thee} in all thy ways.
	
	They shall bear thee \textbf{in} their hands * that thou hurt not thy \textbf{foot} against a stone.
	
	Thou shalt go upon the lion and \textbf{ad}der * the young lion and the dragon shalt thou tread \textbf{un}der thy feet.
	
	Because he hath set his love upon me, therefore will I de\textbf{liv}er him * I will set him up, because \textbf{he} hath known my Name.
	
	He shall call upon me, and I will \textbf{hear} him * yea, I am with him in trouble; I will deliver him, and bring \textbf{him} to honor.
	
	With long life will I satis\textbf{fy} him * and shew him \textbf{my} salvation.
	
	Glory be to the Father, and \textbf{to} the Son * and to the \textbf{Ho}ly Spirit.
	
	As it was in the beginning is now, and ever \textbf{shall} be * world with\textbf{out} end. Amen.
\end{multicols}

\pagebreak[1]

\greannotation{\veformat{Ps. 133}}
\greannotation{\veformat{viii. i.}}
\skiplines{1}\gabcsnippet{
	(c4) Be(gv)hold(hv) (:?) now,(jvr) <b>praise</b>(kv) the(jr) Lord(j) <sp>*</sp>(:) all(jr) ye(j) <b>ser</b>(i)vants(jv) of(h) the(gr) Lord.(g)
}

\begin{multicols}{2}
	Ye that by night stand in the house \textbf{of} the Lord * even in the courts of the \textbf{house} of our God.
	
	Lift up your hands in the sanctu\textbf{ar}y * \textbf{and} praise the Lord.
	
	The Lord who made heaven \textbf{and} earth * give thee blessing \textbf{out} of Zion.
	
	Glory be to the Father, and \textbf{to} the Son * and to the \textbf{Ho}ly Spirit.
	
	As it was in the beginning is now, and ever \textbf{shall} be * world with\textbf{out} end. Amen.
\end{multicols}

\repant{ant/miserere_michi_domine.gabc}

\instruct{Turn back to the \whiteribbon for the chapter.}
	
	\vepart{Common Tones}
		
		\vechapter{Vespers on Sunday}
	
	\skiplines{-1}%	Whose wisdom joined in meet array\\
%	The morn and eve, and named them day:\\
%	Night comes with all its darkling fears,\\
%	Regard thy people's prayers and tears.
%	
%	Lest, sunk in sin, and whelmed with strife\\
%	They lose the gift of endless life;\\
%	While, thinking but the thoughts of time,\\
%	They wave new chains of woe and crime.
%	
%	But grant them grace that they may strain\\
%	The heav'nly gate and prize to gain:\\
%	Each harmful lure aside to cast,\\
%	And purge away each error past.
%	
%	O Father, that we ask be done,\\
%	Through Jesus Christ, thine only Son:\\
%	Who, with the Holy Ghost and thee,\\
%	Doth live and reign eternally.

\gresetinitiallines{0}

\greannotation{\veformat{Hymn}}
\greannotation{\veformat{viii.}}

\vecomment{Lucis Creator optime}

\gresetlastline{justified}

\gregorioscore{hymn/lucis_creator_optime/1.gabc}
\skiplines{\vehymn}\begin{nstabbing}
	\>2. Whose \>wis-\>dom \>joined \>in \>meet \>ar-\>ray\\
	\>3. Lest, \>sunk \>in \>sin, \>and \>whelmed \>with \>strife\\
	\>4. But \>grant \>them \>grace \>they \>they \>may \>strain\\
	\>5. O \>Fa-\>ther \>that \>we \>ask \>be \>done,
\end{nstabbing}

\gresetinitiallines{0}
\skiplines{1}\gregorioscore{hymn/lucis_creator_optime/2.gabc}
\skiplines{\vehymn}\begin{nstabbing}
	\>2. The \>morn \>and \>eve, \>and \>named \>them \>day:\\
	\>3. They \>lose \>the \>gift \>of \>end-\>less \>life;\\
	\>4. The \>heav'n-\>ly \>gate \>and \>prize \>to \>gain:\\
	\>5. Through \>Je-\>sus \>Christ, \>thine \>on-\>ly \>Son;
\end{nstabbing}

\skiplines{1}\gregorioscore{hymn/lucis_creator_optime/3.gabc}
\skiplines{\vehymn}\begin{nstabbing}
	\>2. Night \>comes \>with \>all \>its \>dark-\>ling \>fears,\\
	\>3. While, \>think-\>ing \>but \>the \>thoughts \>of \>time,\\
	\>4. Each \>harm-\>ful \>lure \>a-\>side \>to \>cast,\\
	\>5. Who, \>with \>the \>Ho-\>ly \>Ghost \>and \>thee,
\end{nstabbing}

\clearpage\gregorioscore{hymn/lucis_creator_optime/4.gabc}
\skiplines{\vehymn}\begin{nstabbing}
	\>2. Re-\>gard \>thy \>peo-\>ple's \>prayers \>and \>tears.\\
	\>3. They \>weave \>new \>chains \>of \>woe \>and \>crime.\\
	\>4. And \>purge \>a-\>way \>each \>er-\>ror \>past.\\
	\>5. Doth \>live \>and \>reign \>e-\>ter-\>na-\>ly. \>A-\>men.
\end{nstabbing}

\gresetlastline{ragged}
\gresetinitiallines{1}
	
	\gresetinitiallines{0}
	\skiplines{0}\gabcsnippet{
		(c3) <sp>v</sp> Let(h) my(h) prayer(h) be(h) set(h) forth,(h) O(f) Lord.(g)
		<sp>r</sp>(::) In(h) thy(h) sight(h) as(h) the(h) in(h)cense.(f)
	}
	\gresetinitiallines{1}
	
	\instruct{Turn to the \redribbon and sing the Magnificat as there appointed.}
	
	\greannotation{\veformat{Kyrie}}
	\greannotation{\veformat{XI}}
	\skiplines{1}\gregorioscore{kyrie/xi.gabc}

\vechapter{Vespers on Monday}

	\skiplines{-1}%	The floods above thou didst ordain;\\
%	The floods below thou didst restrain:\\
%	That moisture might attemper heat,\\
%	Lest the parched earth should ruin meet.
%	
%	Upon our souls, good Lord, bestow\\
%	The gift of grace in endless flow:\\
%	Lest some renewed deceit or wile\\
%	Of former sin should us beguile.
%	
%	Let faith discover heav'nly light;\\
%	So shall its rays direct us right:\\
%	And let this faith each error chase,\\
%	And never give to falsehood place.
%	
%	O Father, that we ask be done,\\
%	Through Jesus Christ, thine only Son;\\
%	Who, with the Holy Ghost and thee,\\
%	Doth live and reign eternally.
%	
%	\skiplines{1}\gresetinitiallines{0}\gabcsnippet{(f3) A(fg)men.(fef)}\gresetinitiallines{1}

\gresetinitiallines{0}

\greannotation{\veformat{Hymn}}
\greannotation{\veformat{ii.}}

\vecomment{Immense caeli Conditor}

\gresetlastline{justified}

\gregorioscore{hymn/immense_celi_conditor/1.gabc}%
\skiplines{\vehymn}\begin{nstabbing}
	\>2. The \>floods \>a-\>bove \>thou \>didst \>or-\>dain,\\
	\>3. U-\>pon \>our \>souls, \>good \>Lord, \>bes-\>tow\\
	\>4. Let \>faith \>dis-\>co-\>ver \>heav'n-\>ly \>light;\\
	\>5. O \>Fa-\>ther \>that \>we \>ask \>be \>done,
\end{nstabbing}

\gresetinitiallines{0}
\skiplines{1}\gregorioscore{hymn/immense_celi_conditor/2.gabc}%
\skiplines{\vehymn}\begin{nstabbing}
	\>2. The \>floods \>be-\>low \>thou \>didst \>res-\>train:\\
	\>3. The \>gift \>of \>grace \>in \>end-\>less \>flow:\\
	\>4. So \>shall \>its \>rays \>di-\>rect \>us \>right:\\
	\>5. Through \>Je-\>sus \>Christ, \>thine \>on-\>ly \>Son;
\end{nstabbing}

\skiplines{1}\gregorioscore{hymn/immense_celi_conditor/3.gabc}
\skiplines{\vehymn}\begin{nstabbing}
	\>2. That \>mois-\>ture \>might \>at-\>tem-\>per \>heat,\\
	\>3. Lest \>some \>re-\>newed \>de-\>ceit \>or \>wile\\
	\>4. And \>let \>this \>faith \>each \>er-\>ror \>chase,\\
	\>5. Who, \>with \>the \>Ho-\>ly \>Ghost \>and \>thee,
\end{nstabbing}

\skiplines{0}\gregorioscore{hymn/immense_celi_conditor/4.gabc}
\skiplines{\vehymn}\begin{nstabbing}
	\>2. Lest \>the \>parched \>earth \>should \>ru-\>in \>meet.\\
	\>3. Of \>for-\>mer \>sin \>should \>us \>be-\>guile.\\
	\>4. And \>ne-\>ver \>give \>to \>false-\>hood \>place.\\
	\>5. Doth \>live \>and \>reign \>e-\>ter-\>na-\>ly. \>A-\>men.
\end{nstabbing}

\gresetlastline{ragged}
\gresetinitiallines{1}
	
	\gresetinitiallines{0}
	\skiplines{0}\gabcsnippet{
		(c3) <sp>v</sp> Let(h) my(h) prayer(h) be(h) set(h) forth,(h) O(f) Lord.(g)
		<sp>r</sp>(::) In(h) thy(h) sight(h) as(h) the(h) in(h)cense.(f)
	}
	\gresetinitiallines{1}
	
	\veant{M}{iv}
	\skiplines{1}\gregorioscore{ant/magnificat_te_semper.gabc}
	
	\greannotation{\veformat{Cant.}}
	\greannotation{\veformat{iv.}}
	\skiplines{1}\gabcsnippet{
		(c4) My(hg) soul(gh :? hr) <b>doth</b>(hg) mag(gi)ni(ir)<b>fy</b>(ih) the(hr) Lord(h) <sp>*</sp>(:) and(hr) my() <b>spir</b>(hi)it(hr) hath() rejoiced() <b>in</b>(g) God(h) my(ih) Sav(gvFE)ior.(e) 
	}
	
	\begin{multicols}{2}
	For \textbf{he} hath re\textbf{gard}ed * the \textbf{low}liness \textbf{of} his handmaiden.
	
	For \textbf{be}hold, from \textbf{hence}forth * all ge\textbf{ner}ations \textbf{shall} call me blessed.
	
	For he that is mighty \textbf{hath} magni\textbf{fied} me * --- and \textbf{ho}ly is his Name.
	
	And his mercy is \textbf{on} them that \textbf{fear} him * --- throughout \textbf{all} generations.
	
	He hath \textbf{shewed} strength with \textbf{his} arm * he hath \textbf{scat}tered the proud in the imagi\textbf{na}tion of their hearts.
	
	He hath put down \textbf{the} mighty from \textbf{their} seat * and hath ex\textbf{al}ted \textbf{the} humble and meek.
	
	He hath filled \textbf{the} hungry with \textbf{good} things * and the \textbf{rich} he hath \textbf{sent} empty away.
	
	He, remembering his mercy, hath holpen \textbf{his} servant Is\textbf{ra}el * as he \textbf{prom}ised to our forefathers, to Abraham and \textbf{his} seed forever.
	
	Glory be to \textbf{the} Father, and \textbf{to} the Son * --- and to \textbf{the} Holy Spirit.
	
	As it was in the beginning, is now, \textbf{and} ever \textbf{shall} be * --- world \textbf{with}out end. Amen.
\end{multicols}
	
	\skiplines{-1}\repant{ant/magnificat_te_semper.gabc}
	
	\instruct{Turn to the \whiteribbon at the conclusion rite, pp. \pageref{vespers:conclusion}.}
	
	\greannotation{\veformat{Kyrie}}
	\greannotation{\veformat{XVI}}
	\skiplines{1}\gregorioscore{kyrie/xvi.gabc}

\vechapter{Vespers on Tuesday}

	\skiplines{-1}\gresetinitiallines{0}

\greannotation{\veformat{Hymn}}
\greannotation{\veformat{ii.}}

\vecomment{Telluris ingens Conditor}

\gresetlastline{justified}

\gregorioscore{hymn/telluris_ingens_conditor/1.gabc}%
\begin{nstabbing}
	\>That \>so, \>with \>flowers \>of \>gold-\>en \>hue,\\
	\>Our \>spir-\>it's \>rank-\>ling \>wounds \>ef-\>face\\
	\>Let \>ev-\>ery \>soul \>thy \>law \>o-\>bey,\\
	\>O \>Fa-\>ther \>that \>we \>ask \>be \>done,
\end{nstabbing}

\gresetinitiallines{0}
\skiplines{1}\gregorioscore{hymn/telluris_ingens_conditor/2.gabc}%
\begin{nstabbing}
	\>The \>seeds \>of \>each \>it \>might \>re-\>new;\\
	\>With \>dew-\>y \>fresh-\>ness \>of \>thy \>grace:\\
	\>And \>keep \>from \>ev-\>ery \>e-\>vil \>way;\\
	\>Through \>Je-\>sus \>Christ, \>thine \>on-\>ly \>Son;
\end{nstabbing}

\skiplines{1}\gregorioscore{hymn/telluris_ingens_conditor/3.gabc}
\begin{nstabbing}
	\>And \>fruit-\>trees \>bear-\>ing \>fruit \>might \>yield,\\
	\>That \>grief \>may \>cleanse \>each \>deed \>of \>ill,\\
	\>Re-\>joice \>each \>prom-\>ised \>good \>to \>win,\\
	\>Who, \>with \>the \>Ho-\>ly \>Ghost \>and \>thee,
\end{nstabbing}

\skiplines{1}\gregorioscore{hymn/telluris_ingens_conditor/4.gabc}
\begin{nstabbing}
	\>And \>plea-\>sant \>pas-\>ture \>of \>the \>field.\\
	\>And \>o'er \>each \>lust \>may \>tri-\>umph \>still.\\
	\>And \>flee \>from \>ev-\>ery \>mor-\>tal \>sin.\\
	\>Doth \>live \>and \>reign \>e-\>ter-\>na-\>ly. \>A-\>men.
\end{nstabbing}

\gresetlastline{ragged}
\gresetinitiallines{1}

%	That so, with flowers of golden hue,\\
%	The seeds of each it might renew;\\
%	And fruit-trees bearing fruit might yield,\\
%	And pleasant pasture of the field.
%	
%	Our spirit's rankling wounds efface\\
%	With dewy freshness of thy grace:\\
%	That grief may cleanse each deed of ill,\\
%	And o'er each lust may triumph still.
%	
%	Let every soul thy law obey,\\
%	And keep from every evil way;\\
%	Rejoice each promised good to win,\\
%	And flee from every mortal sin.
%	
%	O Father, that we ask be done,\\
%	Through Jesus Christ, thine only Son;\\
%	Who, with the Holy Ghost and thee,\\
%	Doth live and reign eternally.
%	
%	\skiplines{1}\gresetinitiallines{0}\gabcsnippet{(f3) A(fg)men.(fef)}\gresetinitiallines{1}
	
	\gresetinitiallines{0}
	\skiplines{0}\gabcsnippet{
		(c3) <sp>v</sp> Let(h) my(h) prayer(h) be(h) set(h) forth,(h) O(f) Lord.(g)
		<sp>r</sp>(::) In(h) thy(h) sight(h) as(h) the(h) in(h)cense.(f)
	}
	\gresetinitiallines{1}
	
	\veant{M}{iv}
	\skiplines{1}\gregorioscore{ant/exultavit_spiritus_meus.gabc}
	
	\greannotation{\veformat{Cant.}}
	\greannotation{\veformat{iv.}}
	\skiplines{0}\gabcsnippet{
		(c4) My(hg) soul(gh :? hr) <b>doth</b>(hg) mag(gi)ni(ir)<b>fy</b>(ih) the(hr) Lord(h) <sp>*</sp>(:) and(hr) my() <b>spir</b>(hi)it(hr) hath() rejoiced() <b>in</b>(g) God(h) my(ih) Sav(gvFE)ior.(e) 
	}
	
	\begin{multicols}{2}
	For \textbf{he} hath re\textbf{gard}ed * the \textbf{low}liness \textbf{of} his handmaiden.
	
	For \textbf{be}hold, from \textbf{hence}forth * all ge\textbf{ner}ations \textbf{shall} call me blessed.
	
	For he that is mighty \textbf{hath} magni\textbf{fied} me * --- and \textbf{ho}ly is his Name.
	
	And his mercy is \textbf{on} them that \textbf{fear} him * --- throughout \textbf{all} generations.
	
	He hath \textbf{shewed} strength with \textbf{his} arm * he hath \textbf{scat}tered the proud in the imagi\textbf{na}tion of their hearts.
	
	He hath put down \textbf{the} mighty from \textbf{their} seat * and hath ex\textbf{al}ted \textbf{the} humble and meek.
	
	He hath filled \textbf{the} hungry with \textbf{good} things * and the \textbf{rich} he hath \textbf{sent} empty away.
	
	He, remembering his mercy, hath holpen \textbf{his} servant Is\textbf{ra}el * as he \textbf{prom}ised to our forefathers, to Abraham and \textbf{his} seed forever.
	
	Glory be to \textbf{the} Father, and \textbf{to} the Son * --- and to \textbf{the} Holy Spirit.
	
	As it was in the beginning, is now, \textbf{and} ever \textbf{shall} be * --- world \textbf{with}out end. Amen.
\end{multicols}
	
	\repant{ant/exultavit_spiritus_meus.gabc}
	
	\instruct{Turn to the \whiteribbon at the conclusion rite, pp. \pageref{vespers:conclusion}; cantors, sing this Kyrie.}
	
	\greannotation{\veformat{Kyrie}}
	\greannotation{\veformat{XVI}}
	\skiplines{1}\gregorioscore{kyrie/xvi.gabc}

\vechapter{Vespers on Wednesday}
	
	\skiplines{-1}\gresetinitiallines{0}

\greannotation{\veformat{Hymn}}
\greannotation{\veformat{ii.}}

\vecomment{Caeli Deus sanctissime}

\gresetlastline{justified}

\gregorioscore{hymn/celi_deus_sanctissime/1.gabc}%
\begin{nstabbing}
	\>Thou, \>when \>the \>fourth \>day \>was \>be-\>gun,\\
	\>To \>night \>and \>day, \>by \>cer-\>tain \>line,\\
	\>En-\>light-\>en \>thou \>the \>hearts \>of \>men;\\
	\>O \>Fa-\>ther \>that \>we \>ask \>be \>done,
\end{nstabbing}

\gresetinitiallines{0}
\skiplines{1}\gregorioscore{hymn/celi_deus_sanctissime/2.gabc}%
\begin{nstabbing}
	\>Didst \>frame \>the \>cir-\>cle \>of \>the \>sun,\\
	\>Their \>var'-\>ing \>bounds \>thou \>didst \>as-\>sign;\\
	\>Pol-\>lu-\>ted \>souls \>make \>pure \>a-\>gain;\\
	\>Through \>Je-\>sus \>Christ, \>thine \>on-\>ly \>Son;
\end{nstabbing}

\skiplines{1}\gregorioscore{hymn/celi_deus_sanctissime/3.gabc}
\begin{nstabbing}
	\>And \>set \>the \>moon \>for \>or-\>dered \>change,\\
	\>And \>gav'st \>a \>sig-\>nal, \>known \>and \>meet,\\
	\>Un-\>loose \>the \>bands \>of \>guilt \>with-\>in;\\
	\>Who, \>with \>the \>Ho-\>ly \>Ghost \>and \>thee,
\end{nstabbing}

\skiplines{1}\gregorioscore{hymn/celi_deus_sanctissime/4.gabc}
\begin{nstabbing}
	\>And \>pla-\>nets \>for \>their \>wi-\>der \>range.\\
	\>For \>months \>be-\>gun \>and \>months \>com-\>plete.\\
	\>Re-\>move \>the \>bur-\>den \>of \>our \>sin.\\
	\>Doth \>live \>and \>reign \>e-\>ter-\>na-\>ly. \>A-\>men.
\end{nstabbing}

\gresetlastline{ragged}
\gresetinitiallines{1}

%	Thou, when the fourth day was begun,\\
%	Didst frame the circle of the sun,\\
%	And set the moon for ordered change,\\
%	And planets for their wider range.
%	
%	To night and day, by certain line,\\
%	Their var'ing  bounds thou didst assign;\\
%	And gav'st a signal, known and meet,\\
%	For months begun and months complete.
%	
%	Enlighten thou the hearts of men;\\
%	Polluted souls make pure again;\\
%	Unloose the bands of guilt within;\\
%	Remove the burden of our sin.
%	
%	O Father, that we ask be done,\\
%	Through Jesus Christ, thine only Son;\\
%	Who, with the Holy Ghost and thee,\\
%	Doth live and reign eternally.
%	
%	\skiplines{1}\gresetinitiallines{0}\gabcsnippet{(f3) A(fg)men.(fef)}\gresetinitiallines{1}

	
	\gresetinitiallines{0}
	\skiplines{0}\gabcsnippet{
		(c3) <sp>v</sp> Let(h) my(h) prayer(h) be(h) set(h) forth,(h) O(f) Lord.(g)
		<sp>r</sp>(::) In(h) thy(h) sight(h) as(h) the(h) in(h)cense.(f)
	}
	\gresetinitiallines{1}
	
	\veant{M}{v}
	\skiplines{1}\gregorioscore{ant/respexisti.gabc}
	
	\greannotation{\veformat{Cant.}}
	\greannotation{\veformat{v.}}
	\skiplines{1}\gabcsnippet{
		(c3) My(d) soul(f) (:?) doth(hr) magnify(h) <b>the</b>(i) Lord(hr) <sp>*</sp>(:) and(hr) my() spirit() hath() rejoiced() in(h) <b>God</b>(iv) my(gr) <b>Sav</b>(h)ior.(fr f)
	}
	
	\begin{multicols}{2}
	For he hath re\textbf{gard}ed * the lowliness of \textbf{his} hand\textbf{mai}den.
	
	For behold, from \textbf{hence}forth * all generations shall \textbf{call} me \textbf{bles}sed.
	
	For he that is mighty hath magni\textbf{fied} me * and \textbf{ho}ly is \textbf{his} Name.
	
	And his mercy is on them that \textbf{fear} him * throughout all \textbf{ge}ne\textbf{ra}tions.
	
	He hath shewed strength with \textbf{his} arm * he hath scattered the proud in the imagi\textbf{na}tion of \textbf{their} hearts.
	
	He hath put down the mighty from \textbf{their} seat * and hath ex\textbf{al}ted the hum\textbf{ble} and meek.
	
	He hath filled the hungry with \textbf{good} things * and the rich he hath sent \textbf{emp}ty \textbf{a}way.
	
	He, remembering his mercy, hath holpen his servant \textbf{Is}rael * as he promised to our forefathers, to Abraham and his \textbf{seed} for\textbf{ev}er.
	
	Glory be to the Father, and \textbf{to} the Son * and to the \textbf{Ho}ly \textbf{Spir}it.
	
	As it was in the beginning, is now, and ever \textbf{shall} be * world with\textbf{out} end. \textbf{A}men.
\end{multicols}
	
	\repant{ant/respexisti.gabc}
	
	\instruct{Turn to the \whiteribbon at the conclusion rite, pp. \pageref{vespers:conclusion}; cantors, sing this Kyrie.}
	
	\greannotation{\veformat{Kyrie}}
	\greannotation{\veformat{XVI}}
	\skiplines{1}\gregorioscore{kyrie/xvi.gabc}

\vechapter{Vespers on Thursday}

	\skiplines{-1}\gresetinitiallines{0}
\gresetlastline{justified}

\greannotation{\veformat{Hymn.}}
\greannotation{\veformat{ii.}}

\vecomment{Magne Deus potentie}

\gregorioscore{hymn/magne_deus_potentie/1.gabc}
\skiplines{\vehymn}\begin{nstabbing}
	\>2. Ap-\>point-\>ing \>fish-\>es \>in \>the \>sea,\\
	\>3. Grant \>that \>thy \>ser-\>vants, \>by \>the \>tide\\
	\>4. Let \>none \>des-\>pair \>through \>sin's \>dis-\>tress,\\
	\>5. O \>Fa-\>ther \>that \>we \>ask \>be \>done,
\end{nstabbing}

\skiplines{1}\gregorioscore{hymn/magne_deus_potentie/2.gabc}
\skiplines{\vehymn}\begin{nstabbing}
	\>2. And \>fowls \>in \>o-\>pen \>air \>to \>be;\\
	\>3. Of \>Blood \>and \>Wa-\>ter \>pu-\>ri\>fied,\\
	\>4. Be \>none \>puffed \>up \>with \>boast-\>ful-\>ness:\\
	\>5. Through \>Je-\>sus \>Christ, \>thine \>on-\>ly \>Son;
\end{nstabbing}

\skiplines{1}\gregorioscore{hymn/magne_deus_potentie/3.gabc}
\skiplines{\vehymn}\begin{nstabbing}
	\>2. That \>each, \>by \>or-\>i-\>gin \>the \>same,\\
	\>3. No \>guil-\>ty \>fall \>from \>thee \>may \>know,\\
	\>4. That \>con-\>trite \>hearts \>be \>not \>dis-\>mayed,\\
	\>5. Who, \>with \>the \>Ho-\>ly \>Ghost \>and \>thee,
\end{nstabbing}

\skiplines{0}\gregorioscore{hymn/magne_deus_potentie/4.gabc}
\skiplines{\vehymn}\begin{nstabbing}
	\>2. Its \>sep'-\>rate \>dwel-\>ling \>place \>might \>claim.\\
	\>3. Nor \>e-\>ter-\>nal \>death \>un-\>der-\>go.\\
	\>4. Not \>haugh-\>ty \>souls \>in \>ru-\>in \>laid.\\
	\>5. Doth \>live \>and \>reign \>e-\>ter-\>na-\>ly. \>A-\>men.
\end{nstabbing}


\gresetinitiallines{1}
\gresetlastline{ragged}

%	Appointing fishes in the sea,\\
%	And fowls in open air to be;\\
%	That each, by origin the same,\\
%	Its sep'rate dwelling place might claim.
%	
%	Grant that thy servants, by the tide\\
%	Of Blood and Water purified,\\
%	No guilty fall from thee may know,\\
%	Nor death eternal undergo.
%	
%	Let none despair through sin's distress,\\
%	Be none puffed up with boastfulness:\\
%	That contrite hearts be not dismayed,\\
%	Nor haughty souls in ruin laid.
%	
%	O Father, that we ask be done,\\
%	Through Jesus Christ, thine only Son;\\
%	Who, with the Holy Ghost and thee,\\
%	Doth live and reign eternally.
	
	\gresetinitiallines{0}
	\skiplines{0}\gabcsnippet{
		(c3) <sp>v</sp> Let(h) my(h) prayer(h) be(h) set(h) forth,(h) O(f) Lord.(g)
		<sp>r</sp>(::) In(h) thy(h) sight(h) as(h) the(h) in(h)cense.(f)
	}
	\gresetinitiallines{1}
	
	\veant{M}{v}
	\skiplines{1}\gregorioscore{ant/deposuit_potentes.gabc}
	
	\greannotation{\veformat{Cant.}}
	\greannotation{\veformat{v.}}
	\skiplines{1}\gabcsnippet{
		(c4) My(f) soul(gh :? hr) <b>doth</b>(hg) mag(gi)ni(hr)fy(h) <b>the</b>(hg) Lord(gh) <sp>*</sp>(:) and(hr) my() <b>spir</b>(hi)it(hr) hath() rejoiced() in(h) <b>God</b>(g) my(f) Sav(gh)ior.(g)
	}
	
	\begin{multicols}{2}
	For he hath re\textbf{gard}ed * the lowliness of \textbf{his} handmaiden.
	
	For behold, from \textbf{hence}forth * all  generations shall \textbf{call} me blessed.
	
	For he that is mighty hath magni\textbf{fied} me * and \textbf{ho}ly is his Name.
	
	And his mercy is on them who \textbf{fear} him * throughout all \textbf{gen}erations.
	
	He hath shewed strength with \textbf{his} arm * he hath scattered the proud in the imagi\textbf{na}tion of their hearts.
	
	He hath put down the mighty from \textbf{their} seat * and hath exalted the \textbf{hum}ble and meek.
	
	He hath filled the hungry with \textbf{good} things * and the rich he hath sent \textbf{emp}ty away.
	
	He, remembering his mercy, hath holpen his servant Is\textbf{ra}el * as he promised to our forefathers, to Abraham and his \textbf{seed} forever.
	
	Glory be to the Father, and to \textbf{the} Son * and to the \textbf{Ho}ly Spirit.
	
	As it was in the beginning, is now, and ever \textbf{shall} be * world with\textbf{out} end. Amen.
\end{multicols}
	
	\repant{ant/deposuit_potentes.gabc}
	
	\instruct{Turn to the \whiteribbon at the conclusion rite, pp. \pageref{vespers:conclusion}; cantors, sing this Kyrie.}
	
	\greannotation{\veformat{Kyrie}}
	\greannotation{\veformat{XVI}}
	\skiplines{1}\gregorioscore{kyrie/xvi.gabc}

\vechapter{Vespers on Friday}

	\skiplines{-1}\gresetinitiallines{0}
\gresetlastline{justified}

\greannotation{\veformat{Hymn.}}
\greannotation{\veformat{ii.}}

\vecomment{Plasmator hominis}

\gregorioscore{hymn/plasmator_hominis/1.gabc}
\skiplines{\vehymn}\begin{nstabbing}
	\>2. The \>migh-\>ty \>forms \>that \>fill \>the \>land,\\
	\>3. From \>all \>thy \>ser-\>vants \>chase \>a-\>way\\
	\>4. In \>heav'n \>thine \>end-\>less \>joys \>bes-\>tow,\\
	\>5. O \>Fa-\>ther \>that \>we \>ask \>be \>done,
\end{nstabbing}

\skiplines{1}\gregorioscore{hymn/plasmator_hominis/2.gabc}
\skiplines{\vehymn}\begin{nstabbing}
	\>2. Ins-\>tinct \>with \>life \>at \>thy \>com-\>mand,\\
	\>3. What-\>e'er \>of \>though \>im-\>pure \>to-\>day\\
	\>4. But \>grant \>thy \>gifts \>of \>grace \>be-\>low;\\
	\>5. Through \>Je-\>sus \>Christ, \>thine \>on-\>ly \>Son;
\end{nstabbing}

\skiplines{1}\gregorioscore{hymn/plasmator_hominis/3.gabc}
\skiplines{\vehymn}\begin{nstabbing}
	\>2. Thou \>gav'st \>sub-\>dued \>to \>hu-\>man-\>kind\\
	\>3. Hath \>ming-\>led \>with \>the \>heart's \>in-\>tent,\\
	\>4. From \>chains \>of \>strife \>our \>souls \>re-\>lease,\\
	\>5. Who, \>with \>the \>Ho-\>ly \>Ghost \>and \>thee,
\end{nstabbing}

\skiplines{1}\gregorioscore{hymn/plasmator_hominis/4.gabc}
\skiplines{\vehymn}\begin{nstabbing}
	\>2. For \>ser-\>vice \>in \>their \>rank \>as-\>signed.\\
	\>3. Or \>with \>the \>ac-\>tions \>hath \>been \>blent.\\
	\>4. Bind \>fast \>the \>gen-\>tle \>bands \>of \>peace.\\
	\>5. Doth \>live \>and \>reign \>e-\>ter-\>na-\>ly. \>A-\>men.
\end{nstabbing}


\gresetinitiallines{1}
\gresetlastline{ragged}

%	The mighty forms that fill the land,\\
%	Instinct with life at thy command,\\
%	Thou gav'st subdued to humankind\\
%	For service in their rank assigned.
%	
%	From all thy servants chase away\\
%	Whate'er of thought impure today\\
%	Hath mingled with the heart's intent,\\
%	Or with the actions hath been blent.
%	
%	In heav'n thine endless joys bestow,\\
%	But grant thy gifts of grace below;\\
%	From chains of strife our souls release,\\
%	Bind fast the gentle bands of peace.
%	
%	O Father, that we ask be done,\\
%	Through Jesus Christ, thine only Son;\\
%	Who, with the Holy Ghost and thee,\\
%	Doth live and reign eternally.
%	
%	\skiplines{1}\gresetinitiallines{0}\gabcsnippet{(f3) A(fg)men.(fef)}\gresetinitiallines{1}
	
	\gresetinitiallines{0}
	\skiplines{0}\gabcsnippet{
		(c3) <sp>v</sp> Let(h) my(h) prayer(h) be(h) set(h) forth,(h) O(f) Lord.(g)
		<sp>r</sp>(::) In(h) thy(h) sight(h) as(h) the(h) in(h)cense.(f)
	}
	\gresetinitiallines{1}
	
	\veant{M}{vii}
	\skiplines{1}\gregorioscore{ant/suscepit_deus.gabc}
	
	\greannotation{\veformat{Cant.}}
	\greannotation{\veformat{vii.}}
	\skiplines{1}\gabcsnippet{
		(c3) My(hg) soul(hi) (:?) doth(ir) <b>mag</b>(ik)ni(jr)fy(j) <b>the</b>(i) Lord(h) <sp>*</sp>(:) and(ir) my() spirit() hath() rejoiced() in(i) <b>God</b>(j) my(i) Sav(h)ior.(gf)
	}
	
	\begin{multicols}{2}
	For he \textbf{hath} re\textbf{gard}ed * the lowliness of \textbf{his} handmaiden.
	
	For \textbf{be}hold, from \textbf{hence}forth * all generations shall \textbf{call} me blessed.
	
	For he that is mighty hath \textbf{mag}ni\textbf{fied} me * and ho\textbf{ly} is his Name.
	
	And his mercy is on \textbf{them} that \textbf{fear} him * throughout all \textbf{gen}erations.
	
	He hath shewed \textbf{strength} with \textbf{his} arm * he hath scattered the proud in the imagina\textbf{tion} of their hearts.
	
	He hath put down the \textbf{migh}ty from \textbf{their} seat * and hath exalted the \textbf{hum}ble and meek.
	
	He hath filled the \textbf{hun}gry with \textbf{good} things * and the rich he hath sent \textbf{em}pty away.
	
	He, remembering his mercy, hath holpen his \textbf{ser}vant Is\textbf{ra}el * as he promised to our forefathers, to Abraham and his \textbf{seed} forever.
	
	Glory be to the \textbf{Fa}ther, and to \textbf{the} Son * and to the \textbf{Ho}ly Spirit.
	
	As it was in the beginning, is now, \textbf{and} ever \textbf{shall} be * world with\textbf{out} end. Amen.
\end{multicols}
	
	\repant{ant/suscepit_deus.gabc}

	\instruct{Turn to the \whiteribbon at the conclusion rite, pp. \pageref{vespers:conclusion}; cantors, sing this Kyrie.}
	
	\greannotation{\veformat{Kyrie}}
	\greannotation{\veformat{XVI}}
	\skiplines{1}\gregorioscore{kyrie/xvi.gabc}

\vechapter{Vespers on Saturday}

	\skiplines{-1}\gresetinitiallines{0}
\gresetlastline{justified}

\greannotation{\veformat{Hymn.}}
\greannotation{\veformat{viii.}}

\vecomment{O Lux beata Trinitas}

\gregorioscore{hymn/o_lux_beata_trinitas/1.gabc}
\begin{nstabbing}
	\>To \>thee \>our \>morn-\>ing \>song \>of \>praise,\\
	\>All \>laud \>to \>God \>the \>Fa-\>ther \>be;
\end{nstabbing}

\pagebreak[1]

\skiplines{1}\gregorioscore{hymn/o_lux_beata_trinitas/2.gabc}
\begin{nstabbing}
	\>To \>thee \>our \>even-\>ing \>prayer \>we \>raise;\\
	\>All \>praise, \>e-\>ter-\>nal \>Son, \>to \>thee;
\end{nstabbing}

\pagebreak[1]

\skiplines{1}\gregorioscore{hymn/o_lux_beata_trinitas/3.gabc}
\begin{nstabbing}
	\>Thy \>glo-\>ry \>suppl-\>'ant \>we \>a-\>dore\\
	\>All \>glo-\>ry, \>as \>is \>ev-\>er \>meet,
\end{nstabbing}

\pagebreak[1]

\clearpage\gregorioscore{hymn/o_lux_beata_trinitas/4.gabc}
\begin{nstabbing}
	\>For-\>ev-\>er \>and \>for-\>ev-\>er-\>more.\\
	\>To \>God \>the \>ho-\>ly \>Pa-\>ra-\>clete. \>A-\>men.
\end{nstabbing}

\pagebreak[1]


\gresetinitiallines{1}
\gresetlastline{ragged}

%	To thee our morning song of praise,\\
%	To thee our evening prayer we raise;\\
%	Thy glory suppl'ant we adore\\
%	Forever and forevermore.
%	
%	All laud to God the Father be;\\
%	All praise, eternal Son, to thee;\\
%	All glory, as is ever meet,\\
%	To God the holy Paraclete.
%	
%	\skiplines{1}\gresetinitiallines{0}\gabcsnippet{(c3) A(ghg)men.(g)}\gresetinitiallines{1}
	
	\gresetinitiallines{0}
	\skiplines{1}\gabcsnippet{
		(c3) <sp>v</sp> Let(h) our(h) even(h)ing(h) prayer(h) come(h) up(h) be(h)fore(h) thee,(h) O(f) Lord.(g)
		<sp>r</sp>(::) And(h) let(h) thy(h) mer(h)cy(h) de(h)scend(h) u(h)pon(f) us.(f)
	}
	\gresetinitiallines{1}
	
	\instruct{Turn to the \redribbon and sing the Magnificat as there appointed. Cantors, sing this kyrie at the conclusion rite.}
	
	\greannotation{\veformat{Kyrie}}
	\greannotation{\veformat{XI}}
	\skiplines{1}\gregorioscore{kyrie/xi.gabc}
			
		\vechapter{Compline}
		
			\resp{In manus tuas}{vi}
\gregorioscore{resp/in_manus_tuas.gabc}

\gresetinitiallines{1}

\greannotation{\veformat{Hymn}}
\greannotation{\veformat{viii.}}

\vecomment{Te lucis ante terminum}

\gresetlastline{justified}

\gregorioscore{hymn/te_lucis_ante_terminum/1.gabc}%
\begin{nstabbing}
	\>From \>all \>ill \>dreams \>de-\>fend \>our \>eyes,\\
	\>O \>Fa-\>ther \>that \>we \>ask \>be \>done,
\end{nstabbing}

\gresetinitiallines{0}
\skiplines{1}\gregorioscore{hymn/te_lucis_ante_terminum/2.gabc}%
\begin{nstabbing}
	\>From \>night-\>ly \>fears \>and \>fant-\>a-\>sies:\\
	\>Through \>Je-\>sus \>Christ, \>thine \>on-\>ly \>Son;
\end{nstabbing}

\skiplines{1}\gregorioscore{hymn/te_lucis_ante_terminum/3.gabc}
\begin{nstabbing}
	\>Tread \>un-\>der \>foot \>our \>ghost-\>ly \>foe,\\
	\>Who, \>with \>the \>Ho-\>ly \>Ghost \>and \>thee,
\end{nstabbing}

\skiplines{1}\gregorioscore{hymn/te_lucis_ante_terminum/4.gabc}
\begin{nstabbing}
	\>That \>no \>pol-\>lu-\>tion \>we \>may \>know.\\
	\>Doth \>live \>and \>reign \>e-\>ter-\>na-\>ly. \>A-\>men.
\end{nstabbing}

\gresetlastline{ragged}
\gresetinitiallines{1}

\gresetinitiallines{0}
\skiplines{0}\gabcsnippet{
	(c3) <sp>v</sp> Keep(h) us,(h) O(h) Lord,(h) as(h) the(h) ap(h)ple(h) of(h) thine(f) eye.(g)
	<sp>r</sp>(::) Hide(h) us(h) un(h)der(h) the(h) sha(h)dow(h) of(h) thy(h) wings.(f)
}\gresetinitiallines{1}


\greannotation{\veformat{Cant.}}
\greannotation{\veformat{iii. iv.}}
\skiplines{1}\gregorioscore{ant/salva_nos.gabc}

\vecomment{Canticle of Simeon}
\vecomment{Luke ii, 29-32}

\greannotation{\veformat{Cant.}}
\greannotation{\veformat{iii. iv.}}
\gabcsnippet{
	(c3) Lord,(e) now(f) (:?) let(hr)test(h) thou() thy(h) <b>ser</b>(i)vant(hr) de(h)<b>part</b>(g) in(hr) peace(h) <sp>*</sp>(:) a(hr)cord(h)<b>ing</b>(f) to(hr) thy(h) word.(g) 
}

\begin{multicols}{2}
	For mine \textbf{eyes} have seen * thy \textbf{sal}vation.
	
	Which thou \textbf{hast} prepared * before the face of \textbf{all} people.
	
	A light to the revelation \textbf{of} the \textbf{Gen}tiles * and the glory of thy peo\textbf{ple} Israel.
	
	Glory be to the \textbf{Fa}ther, and \textbf{to} the Son * and to the Ho\textbf{ly} Spirit.
	
	As it was in the beginning, is now, and \textbf{ev}er \textbf{shall} be * world without \textbf{end}. Amen.
\end{multicols}

\repant{ant/salva_nos.gabc}

\instruct{Turn to the \whiteribbon at the conclusion rite, pp. \pageref{compline:conclusion}.}

\greannotation{\veformat{Anth.}}
\greannotation{\veformat{v.}}
\skiplines{1}\gregorioscore{anthem/salve_regina.gabc}

\pagebreak[1]

\gresetinitiallines{0}
\skiplines{1}\gabcsnippet{
	(c3) <sp>v</sp> Pray(h) for(h) us,(h) O(h) ho(h)ly(h) Mo(h)ther(h) of(f) God.(g)
	<sp>r</sp>(::) That(h) we(h) may(h) be(h) made(h) wor(h)thy(h) of(h) the(h) prom(h)is(h)es(h) of(h) Christ.(f)
}\gresetinitiallines{1}

\instruct{Turn to the \whiteribbon and make quietly the concluding prayers.}
	
	\vepart{Temporal}
	
		\vechapter{Second Sunday after Epiphany}
		
			\skiplines{-1}\vesectionp{At First Vespers}
			
				% All Sundays after Epiphany at First Vespers use these propers.
				
				\greannotation{\veformat{Ant.}}
				\greannotation{\veformat{i. i.}}
				\gregorioscore{ant/peccata_mea_domine.gabc}
				
				\vecomment{Canticle of Mary \enquote{Magnificat}}
				
				\greannotation{\veformat{Cant.}}
				\greannotation{\veformat{i. i.}}
				\gabcsnippet{
					(f3) My(h) soul(i:?) doth(jr) magni--()fy(j) <b>the</b>(i) Lord(j) <sp>*</sp>(:) and(jr) my() spirit() hath() rejoiced() in(j) <b>God</b>(i) my(h) Sav(ij)ior.(ir)
				}
			
				\begin{multicols}{2}
	For he hath re\textbf{gard}ed * the lowliness of \textbf{his} handmaiden.
	
	For behold, from \textbf{hence}forth * all  generations shall \textbf{call} me blessed.
	
	For he that is mighty hath magni\textbf{fied} me * and \textbf{ho}ly is his Name.
	
	And his mercy is on them who \textbf{fear} him * throughout all \textbf{gen}erations.
	
	He hath shewed strength with \textbf{his} arm * he hath scattered the proud in the imagi\textbf{na}tion of their hearts.
	
	He hath put down the mighty from \textbf{their} seat * and hath exalted the \textbf{hum}ble and meek.
	
	He hath filled the hungry with \textbf{good} things * and the rich he hath sent \textbf{emp}ty away.
	
	He, remembering his mercy, hath holpen his servant Is\textbf{ra}el * as he promised to our forefathers, to Abraham and his \textbf{seed} forever.
	
	Glory be to the Father, and to \textbf{the} Son * and to the \textbf{Ho}ly Spirit.
	
	As it was in the beginning, is now, and ever \textbf{shall} be * world with\textbf{out} end. Amen.
\end{multicols}
				
				\repant{ant/peccata_mea_domine.gabc}
				
				\instruct{Turn to the \whiteribbon at the conclusion rite, pp. \pageref{vespers:conclusion}. Cantors, turn to the \blueribbon for the Kyrie.}
			
			\cleardoublepage\vesection{At Second Vespers}
		
		\vechapter{Third Sunday after Epiphany}
		
		\vechapter{Fourth Sunday after Epiphany}
		
		\vechapter{Fifth Sunday after Epiphany}
		
		\vechapter{Sixth Sunday after Epiphany}
		
		\vechapter{Third Week before Lent}
		
			\skiplines{-1}\vesectionp{Sunday at First Vespers}
			
			\cleardoublepage\vesection{Sunday at Second Vespers}
			
			\cleardoublepage\vesection{Tuesday at Vespers}
			
			\cleardoublepage\vesection{Wednesday at Vespers}
			
			\cleardoublepage\vesection{Thursday at Vespers}
			
			\cleardoublepage\vesection{Friday at Vespers}
			
		\vechapter{Second Week before Lent}
		
			\skiplines{-1}\vesectionp{Sunday at First Vespers}
		
			\cleardoublepage\vesection{Sunday at Second Vespers}
			
			\cleardoublepage\vesection{Tuesday at Vespers}
			
			\cleardoublepage\vesection{Wednesday at Vespers}
			
			\cleardoublepage\vesection{Thursday at Vespers}
			
			\cleardoublepage\vesection{Friday at Vespers}
			
		\vechapter{First Week before Lent}
		
			\skiplines{-1}\vesectionp{Sunday at First Vespers}
			
			\cleardoublepage\vesection{Sunday at Second Vespers}
			
			\cleardoublepage\vesection{Tuesday at Vespers}
			
			\cleardoublepage\vesection{Ash Wednesday at Vespers}
			
			\cleardoublepage\vesection{Thursday after Ashes at Vespers}
			
			\cleardoublepage\vesection{Friday after Ashes at Vespers}
	
	\vepart{Sanctoral}
	
		\vechapter{Saint Agnes, Virgin-Martyr}
		
		\vechapter{Conversion of Saint Paul}
		
		\vechapter{Purification of the\\Blessed Virgin Mary}
		
		\vechapter{Saint Agatha, Virgin-Martyr}
		
		\vechapter{Saint Valentine,\\Priest and Martyr}
		
		\vechapter{Saints Perpetua and Felicity, Martyrs}
	
\end{document}