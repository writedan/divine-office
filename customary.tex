\documentclass[14pt,twoside]{extarticle}
\usepackage{moresize}
\usepackage{fontspec}
\usepackage{aeguill}
\usepackage{titlesec}
\usepackage[maxlevel=3]{csquotes}
\usepackage{lettrine}
\usepackage{xifthen}
\usepackage{yfonts}
\usepackage{parskip}
\usepackage[dvipsnames]{xcolor}
\usepackage{fancyhdr}
\usepackage{hanging}
\usepackage[autocompile,allowdeprecated=false]{gregoriotex}
\usepackage[twoside]{geometry}
\usepackage{multicol}
\usepackage{paracol}
\usepackage[latin,english]{babel}
\usepackage{enumitem}
\usepackage{tabto}
\usepackage{titlesec}
\usepackage[savepos]{zref}
\usepackage{setspace}
\usepackage{xstring}
\def\ReplaceStr#1{%
	\IfSubStr{#1}{\newline}{%
		\StrSubstitute{#1}{\newline}{ }}{#1}}
%\usepackage{libertine}

\setlength{\columnseprule}{0.025pt}
\setlength{\columnsep}{1.5em}
\setlength{\topskip}{7pt}

\def\vehymn{-0.50}

\makeatletter
\patchcmd{\pcol@buildcolseprule}%
{\hrule\@height\@tempdima\@width\columnseprule}%
{\hrule\@height\dimexpr\@tempdima-0.5ex\@width\columnseprule\@depth-1pt}{}{}
\makeatother

\gredefsizedsymbol{GreCross}{greextra}{Cross}
\gredefsizedsymbol{GreDagger}{greextra}{Dagger}
\gredefsizedsymbol{GreCrossAlt}{greextra}{Cross.alt}
\gredefsizedsymbol{GreStarHeight}{greextra}{StarHeight}
\gredefsizedsymbol{greABar}{greextra}{ABar}
\gredefsizedsymbol{greVBar}{greextra}{VBar}
\gredefsizedsymbol{greRBar}{greextra}{RBar}

\gresetspecial{r}{\textcolor{red}{\Rbar{}.}}
\gresetspecial{v}{\textcolor{red}{\Vbar{}.}}
\gresetspecial{a}{\textcolor{red}{\Abar{}.}}
\gresetspecial{+}{\textcolor{red}{\GreCross{17}}}
\gresetspecial{1}{\textcolor{red}{\GreDagger{17}}}
\gresetspecial{2}{\textcolor{red}{\GreCrossAlt{17}}}
\gresetspecial{*}{\textcolor{red}{\GreStarHeight{17}}}
%%% for spacing in chant %%%
\gresetspecial{ }{\textcolor{white}{x}}
\gresetspecial{  }{\textcolor{white}{xx}}
\gresetspecial{   }{\textcolor{white}{xxx}}
\gresetspecial{    }{\textcolor{white}{xxxx}}

\grechangestaffsize{30}
%\grechangedim{abovelinestextheight}{0.25cm}{fixed}
%\grechangedim{spacebeneathtext}{0.5cm}{fixed}
%\grechangedim{annotationseparation}{0.25cm}{fixed}


\newcommand{\skiplines}[1]{\pagebreak[1]\vspace*{#1\baselineskip}}

\makeatletter
\newcommand{\veformat}[1]{\textcolor{red}{\textbf{#1}}}
\newcommand{\veant}[2]{\greannotation{#1. \veformat{Ant.}}\greannotation{\veformat{#2.}}}
\newcommand{\instruct}[1]{\textcolor{red}{\textit{#1}}}

\newcommand{\whiteribbon}[0]{\textcolor{black}{\textbf{white ribbon}} }%
\newcommand{\greenribbon}[0]{\textcolor{ForestGreen}{\textbf{green ribbon}} }%
\newcommand{\redribbon}[0]{\textcolor{red}{\textbf{red ribbon}} }%
\newcommand{\blueribbon}[0]{\textcolor{blue}{\textbf{blue ribbon}} }%
\makeatother

%%% HEADERS %%%
\makeatletter
\newcommand{\vepart}[1]{%
	\cleardoublepage%
	\vspace*{\fill}%
	\begin{center}%
		\setstretch{2}
		{\Huge \selectfont \uppercase{ \textbf{#1}}}\par%
	\end{center}%
	\vspace*{\fill}%
	\fancyhead{}
}%
\makeatother

\newcommand{\vechapter}[1]{%
	\cleardoublepage%
	\fancyhead[RO,RE]{text}
	\StrSubstitute{#1}{\\}{\space}[\vetempmacrolol]%
	\fancyhead[LO,LE]{\vetempmacrolol}%
	\begin{center}
		\setstretch{1.5}
		{\LARGE \selectfont \color{red} \uppercase{ \textbf{#1}}}\par%
	\end{center}
	\skiplines{1}
}%
\makeatother

\newcommand{\vesectionp}[1]{%
	\skiplines{1}
	\fancyhead[RO,RE]{#1}%
	\begin{center}
		\setstretch{1.5}
		{\Large \selectfont \uppercase{ \textbf{#1}}}\par%
	\end{center}
	\skiplines{1}
}%
\makeatother

\newcommand{\vesection}[1]{%
	\clearpage\skiplines{-2}\vesectionp{#1}%
}%
\makeatother

\makeatletter
\newcommand{\veheading}[1]{%
	\skiplines{1}%
	\begin{center}
		\setstretch{1.5}
		{\selectfont \textit{#1}}\par%
	\end{center}
	\skiplines{1}
}%
\makeatother

\makeatletter
\newcommand{\vecomment}[1]{%
	\hfill{\small\selectfont \textit{#1}.}%
}%
\makeatother

%%% OTHER FORMATS %%%

\makeatletter
\newcommand{\red}[1]{%
	{\color{red} #1}%
}%
\makeatother

\makeatletter
\newcommand{\chapter}[1]{%
	\gresetinitiallines{1}%
	\greannotation{\veformat{Chap.}}%
	\greannotation{\veformat{Dir.}}%
	\gabcsnippet{#1}%
}%
\makeatother

\makeatletter
\newcommand{\resp}[2]{%
	\vecomment{#1}
	
	\gresetinitiallines{1}%
	\greannotation{\veformat{Resp.}}%
	\greannotation{\veformat{#2.}}%
}%
\makeatother

\makeatletter
\newcommand{\tone}[2]{%
	\greannotation{\veformat{Ps.}}
	\greannotation{\veformat{#1.}}
	
	\gabcsnippet{#2}
}%
\makeatother

\makeatletter
\newcommand{\repant}[1]{%
	\gresetspecial{*}{}
	\nopagebreak[4]\skiplines{1}\nopagebreak[4]\gresetinitiallines{0}\gregorioscore{#1}\gresetinitiallines{1}
	\gresetspecial{*}{\textcolor{red}{\GreStarHeight{17}}}
}%
\makeatother



%%% FANCY %%%
\makeatletter  
\newcounter{score}
\newcounter{tabstop}[score]
\newcommand{\grealign}{%
	\@bsphack%
	\ifgre@boxing\else%
	\kern\gre@dimen@begindifference%
	\stepcounter{tabstop}%
	\expandafter\zsavepos{stop-\thescore-\thetabstop}%
	\kern-\gre@dimen@begindifference%
	\fi%
	\@esphack%
}

\newcommand{\setstops}{%
	\gdef\nstabbing@stops{%
		\hspace*{-\oddsidemargin}\hspace{-1in}%
		\hspace*{\zposx{stop-\thescore-1} sp}\=%
	}%
	\count@=\@ne
	\loop\ifnum\count@<\value{tabstop}%
	\begingroup\edef\x{\endgroup
		\noexpand\g@addto@macro\noexpand\nstabbing@stops{%
			\noexpand\hspace{-\noexpand\zposx{stop-\thescore-\the\count@} sp}%
			\noexpand\hspace{\noexpand\zposx{stop-\thescore-\the\numexpr\count@+1} sp}\noexpand\=%
		}%
	}\x
	\advance\count@\@ne
	\repeat
	\nstabbing@stops\kill
}
\makeatother

\newenvironment{nstabbing}
{\setlength{\topsep}{0pt}%
	\setlength{\partopsep}{0pt}%
	\tabbing%
	\setstops}
{\endtabbing\stepcounter{score}}
 \usepackage{lscape}
 \usepackage{longtable}
 \usepackage{arydshln}
 

\begin{document}
	\newgeometry{margin=0pt}
	\begin{titlepage}
		\vspace*{\fill}
		
		\begin{center}
			
			\textbf{\Huge\color{black} JUSTIFICATIONS}
			
			\textbf{\Huge\color{black} \&}
			
			\textbf{\Huge\color{black} RUBRICS}
			
		\end{center}
		
		\vspace*{\fill}
	\end{titlepage}
	
	\newgeometry{top=0.75in,bottom=0.75in,inner=0.75in,outer=0.75in,includefoot}
	
	\pagenumbering{arabic}
	\gresetinitiallines{0}
	
	\raggedbottom
	
	\newgeometry{top=0.75in,bottom=0.75in,inner=0.75in,outer=0.75in,includefoot}
	
	\pagenumbering{arabic}
	\gresetinitiallines{0}
	
	\raggedbottom
	
	\vechapter{Introduction}
	
		Often have I been reproached that we have been provided the post-concillar reformed hours, or worse yet that we are under no obligation to pray the hours. Both are true, yet they desert virtue for value and righteousness for license: the canonists of old long maintained that no novelty ought to be introduced into the worship of the Church, but everything continued as received from the very apostolic authority. I shall therefore argue that the worship of God in psalm, being the hours, is more incumbent upon us than for our predecessors, that the apostolic rite of the Holy Roman Church is the authentic form of the hours we ought to celebrate, and that the post-concillar hours are whole-cloth novelty and unsuitable for the corporate prayer of the church.
		
		First, we note oftentimes Scripture makes reference to allotting specific times of the day to prayer, especially prayer in common. The psalms say \enquote{Seven times a day do I praise thee because of thy righteous judgments}\footnote{Ps. 118, 164-165.} and in the same place \enquote{At midnight will I rise to give thanks unto thee because of thy righteous judgments}\footnote{Ps. 118, 62.}. In other place the psalmist tells us \enquote{Evening and morning and at noon I will speak and declare}\footnote{Ps. 54, 16.} with whom the prophet Daniel concurs: \enquote{He kneeled upon upon his knees three times a day, and gave thanks before God}\footnote{Daniel 6, 10.}. The Gospels are plenteous with examples of Christ praying: early in the morning\footnote{Mark 1,35.}, in the night watches\footnote{Luke 6, 12.}, and often retreating\footnote{Mark 1, 35; 6, 46; Luke 5, 16; Matthew 4, 1; 14, 23.}. More than merely praying, Christ commanded us \enquote{that men ought always to pray}\footnote{Luke 18, 1.} which Saint Paul the Apostle echoes: \enquote{Rejoice always, pray without ceasing}\footnote{1 Thessalonians 5, 16.}.
		
		But \enquote{we know not how to pray as we ought}\footnote{Romans 8, 26}; thus Christ has sent the Holy Spirit into the Church at Pentecost that the divine mission of salvation not cease upon the day of his Ascension and return unto his Father. Nay, rather \enquote{our old self is crucified with him, that the body of sin might be destroyed}\footnote{Romans 6, 6.}. Baptism is the adoption by regeneration as children of God and thus incorporation into the mystical Body of Christ, which we call the Church. The liturgy of the Church is thus nothing more than the continued prayer of Christ in the world unto our almighty Father.
		
		This is the point which distinguishes the liturgical from the devotional or supplicatory or whatever other forms of prayer imaginable: that, while all prayer originates not from the corrupted human nature but of the godly regenerative power of the Holy Spirit, liturgical prayer is not a creation of the human mind but a gift returned unto God \enquote{for all his benefits towards me}\footnote{Ps. 116, 12.}. It is the prayer of the Church together, not because we say the same things at the same times, but because it is really the divine Head of the Church praying, who is to say Christ, and this is the wellspring of worship from whence comes our unity in prayer.
		
		This the Apostles understood upon the Pentecost and set about the Work of God\footnote{\textit{Opus Dei}, which is what Saint Benedict referred to the Office as in his Rule.} immediately. When the Church as a community is first mentioned in Scripture, they are seen together at prayer: \enquote{with several women, including Mary the Mother of Jesus, and with his brothers}\footnote{Acts 1, 14.}. From the outset, then, the Church has never ceased day or night to offer praises unto God.
		
		It came about over the course of time that certain times, or \textit{hours}, of the day were set aside for this communal prayer. From the very outset, the psalms occupied the life of prayer, for they are prayers directly inspired by God for us to praise him, to offer him thanksgiving, to ask of his mercy, to even ask for material good, to supplicate the ceasing of our sin, to send forth the second Advent of the Lord. The whole of human life is contained within the psalter. To the psalms were added the reading of the other parts of Scripture, and even this is the primary end of the Liturgy of the Word: not to instruct us the listeners, but to offer back to God his first revelation, precisely in the same way we offer back to him his final revelation of Christ in the Eucharist. The liturgical hours, therefore, complete the worship in sacrifice of the Mass with the worship of praise\footnote{C.f. \textit{Liturgy of Saint John Chrysostom}, anaphora rite.}.
		
		These times of day, as demonstrated, were informed by Scripture and apostolic practice. The Latin Church came to recognize eight hours: Vigils\footnote{Vigils' more common nomenclature is \textit{Matins}, from the Latin word for the morning. It is intended as a midnight hour of prayer, a tradition nearly only maintained now by the Carthusians, but was delayed increasingly until it combined with Lauds (from the Latin for \textit{praises}), forming a \enquote{morning praises} office.}, Lauds, Prime, Terce, Sext, None, Vespers, Compline. Each has a symbolic meaning derived both from the time of day as well as from the content of its psalmody. We shall investigate each in turn.
		
		\clearpage\begin{longtable}[c]{llllllll}
			&
			Sun. &
			Mon. &
			Tue. &
			Wed. &
			Thu. &
			Fri. &
			Sat. \\
			\hline
			\endhead
			%
			Invitatory &
			94 &
			--- &
			--- &
			--- &
			--- &
			--- &
			--- \\
			\hdashline
			Vigils &
			\begin{tabular}[c]{@{}l@{}}1  \\ 2  \\ 3  \\ 6  \\ 7  \\ 8  \\ 9  \\ 10  \\ 11\\ 12\\ 13\\ 14\\ 15\\ 16\\ 17\\ 18\\ 19\\ 20\end{tabular} &
			\begin{tabular}[c]{@{}l@{}}26\\ 27\\ 28\\ 29\\ 30\\ 31\\ 32\\ 33\\ 34\\ 35\\ 36\\ 37\end{tabular} &
			\begin{tabular}[c]{@{}l@{}}38\\ 39\\ 40\\ 41\\ 43\\ 44\\ 45\\ 46\\ 47\\ 48\\ 49\\ 51\end{tabular} &
			\begin{tabular}[c]{@{}l@{}}52\\ 54\\ 55\\ 56\\ 57\\ 58\\ 59\\ 60\\ 61\\ 63\\ 65\\ 67\end{tabular} &
			\begin{tabular}[c]{@{}l@{}}68\\ 69\\ 70\\ 71\\ 72\\ 73\\ 74\\ 75\\ 76\\ 77\\ 78\\ 79\end{tabular} &
			\begin{tabular}[c]{@{}l@{}}80\\ 81\\ 82\\ 83\\ 84\\ 85\\ 86\\ 87\\ 88\\ 93\\ 95\\ 96\end{tabular} &
			\begin{tabular}[c]{@{}l@{}}97\\ 98\\ 99\\ 100\\ 101\\ 102\\ 103\\ 104\\ 105\\ 107\\ 107\\ 108\end{tabular} \\
			\hline
			Lauds &
			\begin{tabular}[c]{@{}l@{}}92\\ 99\\ 62\\ 66\\ Dan. 3\\ 148\\ 149\\ 150\end{tabular} &
			\begin{tabular}[c]{@{}l@{}}50\\ 5\\ 62\\ 66\\ Is. 12\\ 148\\ 149\\ 150\end{tabular} &
			\begin{tabular}[c]{@{}l@{}}50\\ 42\\ 62\\ 66\\ Is. 38\\ 148\\ 149\\ 150\end{tabular} &
			\begin{tabular}[c]{@{}l@{}}50\\ 64\\ 62\\ 66\\ 1 Sam. 12\\ 148\\ 149\\ 150\end{tabular} &
			\begin{tabular}[c]{@{}l@{}}50\\ 89\\ 62\\ 66\\ Ex. 15\\ 148\\ 149\\ 150\end{tabular} &
			\begin{tabular}[c]{@{}l@{}}50\\ 142\\ 62\\ 66\\ Hab. 3\\ 148\\ 149\\ 150\end{tabular} &
			\begin{tabular}[c]{@{}l@{}}50\\ 91\\ 62\\ 66\\ Deut. 32\\ 148\\ 149\\ 150\end{tabular} \\
			\hline
			Prime &
			\begin{tabular}[c]{@{}l@{}}21\\ 22\\ 23\\ 24\\ 25\\ 53\\ 117\\ 118A\\ 118B\end{tabular} &
			\begin{tabular}[c]{@{}l@{}}53\\ 118A\\ 118B\end{tabular} &
			--- &
			--- &
			--- &
			--- &
			--- \\
			\hline
			Terce &
			\begin{tabular}[c]{@{}l@{}}118C\\ 118D\\ 118E\end{tabular} &
			--- &
			--- &
			--- &
			--- &
			--- &
			--- \\
			\hline
			Sext &
			\begin{tabular}[c]{@{}l@{}}118F\\ 118G\\ 118H\end{tabular} &
			--- &
			--- &
			--- &
			--- &
			--- &
			--- \\
			\hline
			None &
			\begin{tabular}[c]{@{}l@{}}118I\\ 118J\\ 118K\end{tabular} &
			--- &
			--- &
			--- &
			--- &
			--- &
			--- \\
			\hline
			Vespers &
			\begin{tabular}[c]{@{}l@{}}109\\ 110\\ 111\\ 112\\ 113\end{tabular} &
			\begin{tabular}[c]{@{}l@{}}114\\ 115\\ 116\\ 119\\ 120\end{tabular} &
			\begin{tabular}[c]{@{}l@{}}121\\ 122\\ 123\\ 124\\ 125\end{tabular} &
			\begin{tabular}[c]{@{}l@{}}126\\ 127\\ 128\\ 129\\ 130\end{tabular} &
			\begin{tabular}[c]{@{}l@{}}131\\ 132\\ 134\\ 135\\ 136\end{tabular} &
			\begin{tabular}[c]{@{}l@{}}137\\ 138\\ 139\\ 140\\ 141\end{tabular} &
			\begin{tabular}[c]{@{}l@{}}143\\ 144\\ 145\\ 146\\ 147\end{tabular} \\
			\hline
			Compline &
			\begin{tabular}[c]{@{}l@{}}4\\ 90\\ 133\end{tabular} &
			--- &
			--- &
			--- &
			--- &
			--- &
			---\\
			\hline
		\end{longtable}
	
	\clearpage
	
	Most liturgical days begin at Vigils, which was said at midnight\footnote{As we explain the meaning of each hour, it is important to recall that the hours developed in the first few centuries of Christianity when there were no clocks. References to time therefore does not have any relationship to the times of the clock, but to the times of the sunrise and sunset. Midnight here means the equidistant point from sunset to the next sunrise.}. Vigils begins with the inviatatory, which is always Ps. 94, wherein the ministers process in while chanting its verses, to which the congregation responds with a set antiphon. Vigils has either one or three \textit{nocturns}, or watches. A watch always consists of psalms, a versicle, a silent \textit{Paternoster}, and three lessons, each followed by a solemn, elaborate responsorial chant. Sunday and festal Vigils have three nocturns; ferial Vigils have one. The ferial nocturn is always twelve psalms long, because this number was revealed to the Desert Fathers by an angel as the ideal number of psalms to pray in one sitting, as Saint John Cassian reports\footnote{\textit{Institutes} ii, 4.}. Vigils of three watches typically has three psalms appointed at each, but dominical\footnote{Sunday.} Vigils has a twelve-psalm first nocturn as feria.
	
	At the Vigil hour, the Church keeps watch for the resurrection of Christ, who lies in the grave after Compline, and we comfort our divine Head who at this time began his sorrowful Passion in the garden, as we read: \enquote{\enquote{I am deeply grieved, even to death; remain here, and keep awake.} And going a little farther, he threw himself on the ground and prayed}\footnote{Mark 14, 34-35.}. On some days there are three nocturns, because Saints Mark and Matthew record thrice Christ prayed \enquote{Let this cup pass from me}\footnote{Mark 14, 36; Matthew 29, 39.} while Saint Luke records it only once\footnote{Luke 22, 42.}. Thus we pray with him this hour \enquote{Thy will be done} some days once, others thrice.
	
	Vigils is the only hour where extended lessons from Scripture are read because the Lord has said \enquote{Behold! I am coming like a thief. Blessed is the one who stays awake, keeping his garments on, that he may not go about naked and be seen exposed!}\footnote{Revelation 16, 15}. These garments are those received at baptism which we are instructed then to return unstained to our Creator. Saint Paul explains succinctly: \enquote{For whatsoever things were written aforetime were written for our learning, that we through patience and comfort of the scriptures might have hope}\footnote{Romans 15, 4.}. And hope for what? For the resurrection: \enquote{receive ye one another as Christ also received us, to the glory of God}\footnote{Romans 15, 7.}, for Christ has received us as his brothers, \enquote{co-heirs to eternal life}\footnote{\textit{Eucharistic Prayer II}.}.
	
	The psalmody of Vigils does not belong to it by theme, as the reformers have attempted to establish a basis for and thus re-organize the psalter. Nay rather, it is continuous reading of the psalms in their appointed order by Scripture in like manner as Saint Benedict tells us of the Desert Fathers, who would pray the psalms in order\footnote{\enquote{We read our holy fathers strenuously accomplished in one day this which that we tepid ones may fulfill in a whole week.} \textit{Rule of Saint Benedict} 18.}. Indeed instead, the other hours appropriated their own psalmody according to their own character, aside from Vespers, and the remnants were taken up by the Church in Vigils for the consolation of her Savior.
	
	Lauds follows soon after Vigils at the break of day, and for this reason includes the psalm \enquote{to thee do I watch at break of day}\footnote{Psalm 62, 1.}. Lauds, from the Latin \textit{laudes} meaning praises, celebrates the resurrection of the Son, as we read in the holy Gospel: \enquote{Now after the Sabbath, toward the dawn of the first day of the week, Mary Magdalene and the other Mary went to see the tomb}\footnote{Matthew 28, 1.}. The long watches of the night are over: let the Church receive the resurrected Savior. Thus also do we look towards the King \enquote{who will enter this world with mighty hosts of angel}\footnote{\textit{Liturgy of Saint John Chrysostom}, cherubikon.} on the last dawn.
	
	The psalms of Lauds are well-attuned to their time. There are five components of its psalmody: psalm 92 and 50, the variable psalm, psalms 62 and 66, the canticle, and the \textit{laudes} themselves of psalms 148, 149, and 150.
\end{document}