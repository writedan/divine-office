\documentclass[14pt,twoside]{extarticle}
\usepackage{moresize}
\usepackage{fontspec}
\usepackage{aeguill}
\usepackage{titlesec}
\usepackage[maxlevel=3]{csquotes}
\usepackage{lettrine}
\usepackage{xifthen}
\usepackage{yfonts}
\usepackage{parskip}
\usepackage[dvipsnames]{xcolor}
\usepackage{fancyhdr}
\usepackage{hanging}
\usepackage[autocompile,allowdeprecated=false]{gregoriotex}
\usepackage[twoside]{geometry}
\usepackage{multicol}
\usepackage{paracol}
\usepackage[latin,english]{babel}
\usepackage{enumitem}
\usepackage{tabto}
\usepackage{titlesec}
\usepackage[savepos]{zref}
%\usepackage{libertine}

\setlength{\columnseprule}{0.025pt}
\setlength{\columnsep}{1.5em}
\setlength{\topskip}{7pt}

\def\vehymn{-0.50}

\makeatletter
\patchcmd{\pcol@buildcolseprule}%
{\hrule\@height\@tempdima\@width\columnseprule}%
{\hrule\@height\dimexpr\@tempdima-0.5ex\@width\columnseprule\@depth-1pt}{}{}
\makeatother

\gredefsizedsymbol{GreCross}{greextra}{Cross}
\gredefsizedsymbol{GreDagger}{greextra}{Dagger}
\gredefsizedsymbol{GreCrossAlt}{greextra}{Cross.alt}
\gredefsizedsymbol{GreStarHeight}{greextra}{StarHeight}
\gredefsizedsymbol{greABar}{greextra}{ABar}
\gredefsizedsymbol{greVBar}{greextra}{VBar}
\gredefsizedsymbol{greRBar}{greextra}{RBar}

\gresetspecial{r}{\textcolor{red}{\Rbar{}.}}
\gresetspecial{v}{\textcolor{red}{\Vbar{}.}}
\gresetspecial{a}{\textcolor{red}{\Abar{}.}}
\gresetspecial{+}{\textcolor{red}{\GreCross{17}}}
\gresetspecial{1}{\textcolor{red}{\GreDagger{17}}}
\gresetspecial{2}{\textcolor{red}{\GreCrossAlt{17}}}
\gresetspecial{*}{\textcolor{red}{\GreStarHeight{17}}}
%%% for spacing in chant %%%
\gresetspecial{ }{\textcolor{white}{a}}
\gresetspecial{  }{\textcolor{white}{be}}
\gresetspecial{   }{\textcolor{white}{God}}
\gresetspecial{    }{\textcolor{white}{text}}

\grechangestaffsize{30}
%\grechangedim{abovelinestextheight}{0.25cm}{fixed}
%\grechangedim{spacebeneathtext}{0.5cm}{fixed}
\grechangedim{annotationseparation}{0.25cm}{fixed}

\newcommand{\skiplines}[1]{\vspace*{#1\baselineskip}}

\newcommand{\veformat}[1]{\textcolor{red}{\textbf{#1}}}
\newcommand{\veant}[2]{\greannotation{#1. \veformat{Ant.}}\greannotation{\veformat{#2.}}}
\newcommand{\instruct}[1]{\textcolor{red}{\textit{#1}}}

\makeatletter
\newcommand{\whiteribbon}[0]{\textcolor{black}{\textbf{white ribbon}} }%
\newcommand{\greenribbon}[0]{\textcolor{ForestGreen}{\textbf{green ribbon}} }%
\newcommand{\redribbon}[0]{\textcolor{red}{\textbf{red ribbon}} }%
\newcommand{\blueribbon}[0]{\textcolor{blue}{\textbf{blue ribbon}} }%
\makeatother

%%% HEADERS %%%
\newcommand{\vepart}[1]{%
	\cleardoublepage%
	\vspace*{\fill}%
	\begin{center}%
		{\Huge \selectfont \uppercase{ \textbf{#1}}}%
	\end{center}%
	\vspace*{\fill}%
}%
\makeatother

\newcommand{\vechapter}[1]{%
	\cleardoublepage%
	\begin{center}
		{\LARGE \selectfont \color{red} \uppercase{ \textbf{#1}}}\par%
	\end{center}
	\skiplines{1}
}%
\makeatother

\newcommand{\vesectionp}[1]{%
	\skiplines{1}
	\begin{center}
		{\Large \selectfont \uppercase{ \textbf{#1}}}%
	\end{center}
	\skiplines{1}
}%
\makeatother

\newcommand{\vesection}[1]{%
	\clearpage\skiplines{-2}\vesectionp{#1}%
}%
\makeatother

\makeatletter
\newcommand{\veheading}[1]{%
	\skiplines{0.5}%
	\begin{center}
		{\selectfont \textit{#1}}%
	\end{center}
	\skiplines{0.5}
}%
\makeatother

\makeatletter
\newcommand{\vecomment}[1]{%
	\hfill{\small\selectfont \textit{#1}.}%
}%
\makeatother

%%% OTHER FORMATS %%%

\makeatletter
\newcommand{\red}[1]{%
	{\color{red} #1}%
}%
\makeatother

\makeatletter
\newcommand{\chapter}[1]{%
	\gresetinitiallines{1}%
	\greannotation{\veformat{Chap.}}%
	\greannotation{\veformat{Dir.}}%
	\gabcsnippet{#1}%
}%
\makeatother

\makeatletter
\newcommand{\resp}[2]{%
	\vecomment{#1}
	
	\gresetinitiallines{1}%
	\greannotation{\veformat{Resp.}}%
	\greannotation{\veformat{#2.}}%
}%
\makeatother

\makeatletter
\newcommand{\tone}[2]{%
	\greannotation{\veformat{Ps.}}
	\greannotation{\veformat{#1.}}
	
	\gabcsnippet{#2}
}%
\makeatother

\makeatletter
\newcommand{\repant}[1]{%
	\gresetspecial{*}{}
	\nopagebreak[4]\skiplines{1}\nopagebreak[4]\gresetinitiallines{0}\gregorioscore{#1}\gresetinitiallines{1}
	\gresetspecial{*}{\textcolor{red}{\GreStarHeight{17}}}
}%
\makeatother



%%% FANCY %%%
\makeatletter  
\newcounter{score}
\newcounter{tabstop}[score]
\newcommand{\grealign}{%
	\@bsphack%
	\ifgre@boxing\else%
	\kern\gre@dimen@begindifference%
	\stepcounter{tabstop}%
	\expandafter\zsavepos{stop-\thescore-\thetabstop}%
	\kern-\gre@dimen@begindifference%
	\fi%
	\@esphack%
}

\newcommand{\setstops}{%
	\gdef\nstabbing@stops{%
		\hspace*{-\oddsidemargin}\hspace{-1in}%
		\hspace*{\zposx{stop-\thescore-1} sp}\=%
	}%
	\count@=\@ne
	\loop\ifnum\count@<\value{tabstop}%
	\begingroup\edef\x{\endgroup
		\noexpand\g@addto@macro\noexpand\nstabbing@stops{%
			\noexpand\hspace{-\noexpand\zposx{stop-\thescore-\the\count@} sp}%
			\noexpand\hspace{\noexpand\zposx{stop-\thescore-\the\numexpr\count@+1} sp}\noexpand\=%
		}%
	}\x
	\advance\count@\@ne
	\repeat
	\nstabbing@stops\kill
}
\makeatother

\newenvironment{nstabbing}
{\setlength{\topsep}{0pt}%
	\setlength{\partopsep}{0pt}%
	\tabbing%
	\setstops}
{\endtabbing\stepcounter{score}}

\begin{document}
	\newgeometry{margin=0pt}
	\begin{titlepage}
		\vspace*{\fill}
		
		\begin{center}
			
			\textbf{\Huge\color{red} CUSTOMARY}
			
			ACCORDING TO THE RITE OF THE
			
			\textbf{\LARGE HOLY ROMAN CHURCH}
			
			\vspace*{\fill}
			
			Pertaining to the celebration of the\\Liturgy of the Hours
			
		\end{center}
		
		\vspace*{\fill}
	\end{titlepage}
	
	\newgeometry{top=0.75in,bottom=0.75in,inner=0.75in,outer=0.75in,includefoot}
	
	\pagenumbering{arabic}
	\gresetinitiallines{0}
	
	\raggedbottom
	
	%\vechapter{Liturgical Calendar}
	\skiplines{-1}\vesectionp{January}
	\begin{enumerate}[noitemsep,nolistsep]
		\item \red{Circumcision of the Lord.} Major feast.
		\item Octave of Saint Stephen. \red{9 lect. triplex.}
		\item Octave of Saint John. \red{9 lect. triplex.}
		\item Octave of the Innocents. \red{9 lect. triplex.}
		\item \textcolor{purple}{Eve of the Epiphany.} \red{9 lect. duplex}
		\item \red{\textbf{Epiphany of the Lord.}} Solemnity. \textit{With octave.}
		\item Of the Octave. \red{9 lect. duplex.}
		\item Of the Octave. \red{9 lect. duplex.}
		\item Of the Octave. \red{9 lect. duplex.}
		\item Of the Octave. \red{9 lect. duplex.}
		\item Of the Octave. \red{9 lect. duplex.}
		\item Of the Octave. \red{9 lect. duplex.}
		\item Octave of the Epiphany. \red{9 lect. triplex.}
		\item Felix, Priest and Martyr. \red{3 lect. simplex.}
		\item Paul, Hermit. \red{3 lect. simplex.}
		\item Marcellus, Pope and Martyr. \red{3 lect. simplex.}
		\item Anthony of Egypt,Abbot. \red{3 lect. simplex.}
		\item Prisca, Virgin and Martyr. \red{3 lect. simplex.}
		\item Marius, Martha, Audifax, and Abachum, Martyrs. \red{3 lect. simplex.}
		\item Fabian and Sebastian, Martyrs. \red{9 lect. duplex.}
		\item Agnes, Virgin and Martyr. \red{9 lect. duplex.}
		\item Vincent and Anastasius, Martyrs. \red{9 lect. duplex.}
		\item Emerentiana, Virgin and Martyr. \red{3 lect. simplex.}
		\item Timothy, Bishop and Martyr. \red{3 lect. simplex.}
		\item Conversion of Saint Paul, Apostle. \red{9 lect. triplex.}
		\item Polycarp, Bishop and Martyr. \red{3 lect. simplex.}
		\item John Chrysostom, Bishop and Doctor. \red{9 lect. duplex.}
		\item Thomas Aquinas, Priest and Doctor. \red{9 lect. duplex.}
		\item % 29
		\item Gregory Nazianzus,  Bishop and Doctor. \red{9 lect. duplex.}
		\item % 31
	\end{enumerate}

\vesectionp{February}	
	
	\begin{enumerate}[nolistsep,noitemsep]
		
		\item Ignatius, Bishop and Martyr. \red{3 lect. duplex.}
		\item \red{Purification of the Blessed Virgin Mary.} Principle feast.
		\item Blase, Bishop and Martyr. \red{3 lect. simplex.}
		\item % 4
		\item Agatha, Virgin and Martyr. \red{9 lect. duplex.}
		\item Dorothy, Virgin and Martyr.\red{3 lect. simplex.}
		\item % 7
		\item % 8
		\item Apollonia, Virgin and Martyr.\red{3 lect. simplex.}
		\item Scholastica, Virgin. \red{3 lect. simplex.}
		\item % 11
		\item % 12
		\item % 13
		\item Valentine, Priest and Martyr. \red{3 lect. simplex.}
		\item Faustinus and Jovita, Martyrs.
		\item % 16
		\item % 17
		\item Simeon, Bishop and Martyr. \red{3 lect. simplex.}
		\item % 19
		\item % 20
		\item % 21
		\item Cathedra of Saint Peter, Apostle. \red{9 lect. triplex.}
		\item % 23
		\item % 24
		\item % 25
		\item % 26
		\item % 27
		\item % 28
	\end{enumerate}
	
	\vesectionp{March}
	
	\begin{enumerate}[nolistsep,noitemsep]
		\item % 1
		\item % 2
		\item % 3
		\item % 4
		\item % 5
		\item % 6
		\item Perpetua and Felicity, Martyrs. \red{3 lect. simplex.}
		\item % 8
		\item Forty Saints of Sebaste, Martyrs. \red{3 lect. simplex.}
		\item % 10
		\item % 11
		\item \red{Gregory, Pope and Doctor.} Minor feast.
		\item % 13
		\item % 14
		\item % 15
		\item % 16
		\item Patrick, Bishop and Confessor. \red{9 lect. duplex.}
		\item % 18
		\item \red{Joseph, Spouse of the Blessed Virgin Mary. } Minor feast.
		\item % 20
		\item Benedict of Nursia, Abbot. \red{9 lect. duplex.}
		\item % 22
		\item % 23
		\item % 24
		\item \red{Annunciation of the Blessed Virgin Mary.} Major feast.
		\item % 26
		\item % 27
		\item % 28
		\item % 29
		\item % 30
		\item % 31
		
	\end{enumerate}
	
	\vechapter{Order of the Choir}
	
		\skiplines{-1}\vesectionp{Ministers of the Liturgy}
	
		\begin{enumerate}
			\item The four minor orders are, in order, the porter, cantor, lector and acolyte; the three major orders are the subdeacon, deacon, and priest. These major orders may only be undertaken by those ordained in ministry; the minor orders may be undertaken by anyone instituted as such, or deputized by need.
		
			\item The full liturgical celebration requires the presence of one person in each major order. However, besides the Mass and the sacraments, all liturgical celebrations can be reduced to a reader's service presided by those instituted or deputized into the minor orders.
			
			\item Among the minor orders, when either need arises or prudence demands it, women may be instituted as porters and cantors, and in extreme need as lectors and acolytes. But wherever this can be avoided, it is to be avoided.
			
			\item The bishop alone may institute men (or women) to liturgical ministry; otherwise deputation by necessity requires sufficient training. Let no person unfit for the solemnity of the liturgy by his deeds or character serve the liturgy.
			
			
		
		\vesectionp{College of the Chapel}
		
			\item Those instituted to ministry or deputized for ministry in the long term to the service of the chapel constitute the chapel college; each member being a canon. The campus minister appoints one canon to be the dean of the college, or the college elects a dean if they reject this appointment. The dean appoints two other officers: the precantor and the treasurer.
			
			\item If there is a pastor well-regarded by the college, he is to become the dean by right of his ordination. However, the college exists for the celebration of the liturgy; if he is thus a hindrance, he need not be involved by right.
			
			\item The dean is responsible for the regulation of the liturgy, the recruitment and training of canons, and the catechesis of the faithful pertaining to the liturgy. The precantor is responsible for the training of canons and generally the whole faithful in the art of Gregorian chant. The treasurer is responsible for the maintenance of the college's vessels, vestments, implements, and other such liturgical instruments. The canons as a college are responsible for the celebration of the liturgy befitting its nature on behalf of and with the faithful.
			
			\item No person may enter the college without the consent of the dean after consultation with the whole college. As a corollary, the dean retains the right to expel any canon for serious dereliction of morals or duties. This regulation of the college belongs to the dean.
			
			\item On each Sunday, for the week following the next Sunday, the dean must arrange for the liturgical ministers of each service celebrated by the college in the chapel for that week. Mass requires a priest, deacon, and subdeacon; at least two cantors, two acolytes, and seven porters. The hours require, for a full celebration, a priest, deacon, and subdeacon, two acolytes, two cantors, one lector, and six porters. Certain liturgical celebrations require a great number as prescribed. 
			
			\item However, for reader's services, Mass can never be celebrated, but the hours require only an officiant, a lector, and two cantors in the usual case.
				
		\vesectionp{Vestment of Ministers}
			
			\item At Mass, all ministers wear the cassock, surplice, and fascia or cincture. The major orders, however, omit the surplice and instead additionally wear an amice and alb. The subdeacon wears a tunicle and maniple; the deacon wears a stole and dalmatic; and the priest wears the stole, maniple, and chasuble. However, on penitential days, the deacon and subdeacon may not wear dalmatic and tunicle but only folded chasubles.
			
			\item At the hours, all ministers likewise wear cassock, surplice, and fascia or cincture. The priest does not wear a chasuble but a cope. The tunicle and dalmatic are still worn, except on penitential days as at Mass. No person wears a maniple nor amice.
			
			\item During reader's services, all ministers are to be dressed as above, but the officiant is to wear a mantle in the liturgical color.
			
			\item The cassock is always black; the surplice, amice, alb, and cincture always white; the tunicle, dalmatic, chasuble, stole, and maniple are always in the liturgical color, as is the mantle when worn.
			
			\item Women, when serving wear a black tunic, a full white scapular, a white cincture, a white coif, and a black veil. These correspond to the cassock, surplice, fascia, amice, and cope respectively. Let this also be observed at Mass.
			
			\item When rulers of the choir are required by the liturgical solemnity, they are to be vested as with other cantors, but they under the surplice and fascia is worn a red scapular, which is considered like to a mantle; their fascia is also red. On penitential days of any sort, these reds are commuted to violets.
			
			\item This is the rite of vesting to be observed:
			
				\begin{enumerate}
					\item First, each must wash his hands while saying: \begin{quote}
						Give virtue to my hands, O Lord, that being cleansed from all stain of sin I may serve thee with purity of mind and of body.
					\end{quote}
				
					\item Taking each vestment in turn, first kissing it, then vesting with, first the cassock: \begin{quote}
						O Lord, the portion of mine inheritance and my chalice, thou art he who will restore mine inheritance.
					\end{quote}
				
					\item The fascia: \begin{quote}
						Gird me, O Lord, with the cincture of purity, and quench in my heart the first of concupiscence, that the virtues of continence and chastity may abide in me.
					\end{quote}
				
					\item The amice, when worn: \begin{quote}
						Place, O Lord, the helmet of salvation upon my head to repeal the assualts of the devil.
					\end{quote}
				
					\item The surplice or alb: \begin{quote}
						Invest me, O Lord, as a new man, who was created by God in justice and the holiness of truth.
					\end{quote}
				
					\item The mantle: \begin{quote}
						Pour thy grace upon me, O Lord, as the oil upon the beard of Aaron.
					\end{quote}
				
					\item The stole: \begin{quote}
						Lord, restore unto me the stole of immortality, which I lost through the collusion of our first parents, and, unworthy as I am to approach thy sacred mysteries, may I yet gain eternal joy.
					\end{quote}
				
					\item The maniple: \begin{quote}
						May I deserve, O Lord, to bear the maniple of weeping and sorrow in order that I may joyfully reap the reward of my labors.
					\end{quote}
				
					\item The tunicle: \begin{quote}
						May the Lord clothe me with the tunicle of delight and the garment of rejoicing.
					\end{quote}
				
					\item The dalmatic: \begin{quote}
						Lord, clothe me with the garment of salvation and the vestment of gladness, and encompass me always with the dalmatic of justice.
					\end{quote}
				
					\item The chasuble: \begin{quote}
						O Lord, who said \enquote{My yoke is easy and my burden light}, grant that I may so bear it as to attain thy grace.
					\end{quote}
				
					\item The cope: \begin{quote}
						Adorn me, O Lord, with a crown as a bride adorns herself with jewels, and cover me with the robe of salvation.
					\end{quote}
				\end{enumerate}
			
			\item Women say the same prayers as proper to their vestments as explained above.
			
			\item This is the course of liturgical colors:
			
				\begin{enumerate}
					\item Red: \begin{enumerate}
						\item Martyrs.
						\item Apostles.
						\item Beheading of Saint John Baptist.
						\item Pentecost.
					\end{enumerate}
				
					\item Gold or saffron: \begin{enumerate}
						\item Bishops.
						\item Abbots.
						\item Doctors.
						\item Confessors.
					\end{enumerate}
				
					\item Rose: \begin{enumerate}
						\item Holy Innocents.
						\item Virgin-martyrs.
						\item Third Sunday of Advent.
						\item Fourth Sunday in Lent.
					\end{enumerate}
				
					\item Violet: \begin{enumerate}
						\item Sundays and feria of Advent.
						\item Sundays and feria before Lent.
						\item Sundays and feria of Lent.
						\item Sundays and feria of Passiontide.
						\item Embertides.
						\item Eves of feasts.
					\end{enumerate}
				
					\item White: \begin{enumerate}
						\item Feasts of the Lord.
						\item Virgins.
						\item Saint Michael Archangel.
						\item Nativity of Saint John Baptist.
					\end{enumerate}
				
					\item Green: \begin{enumerate}
						\item Sundays and feria after Epiphany.
						\item Sundays and feria after Trinity.
					\end{enumerate}
				
					\item Blue: \begin{enumerate}
						\item Feasts of the Blessed Virgin Mary.
						\item Office of the Blessed Virgin Mary.
					\end{enumerate}
				
					\item Black: \begin{enumerate}
						\item Friday on the Day of Preparation.
						\item Office of the Dead.
					\end{enumerate}
				\end{enumerate}
			
			\item It is the duty of the treasurer that each canon has proper vestments for any celebration of the year.
			
		\vesectionp{Placement in the Quire}
		
			\item The quire is the part of the chapel where the choir is to take their place, at both sides of the altar. Those canons alone who have been assigned on the roster to a particular service are to stand immediately before and facing the altar; other canons present who wish to participate are to stand behind them.
			
			\item The canons alone may stand in the quire; any canon who is not vested may not take his place in the quire, but must be among the congregation.
			
			\item The chapel is orientated towards the east: the north side of the quire is thus the junior side, and the south side the senior side. The major orders or officiant are to be placed on the senior side. The lector is placed on the junior side. Those closest to the west are more senior than those seated to the east.
			
			\item Acolytes, whose number may not exceed two, are placed beyond the pews on the west side, where they stand to the left of the priest and subdeacon (or officiant and lector in their places). Porters, when in use, stand behind the acolytes.
			
			\item When rulers of the choir are called for, they are to stand between the nave and the quire facing each other, either one on each side or two on the senior and one on the junior depending on their number. 
			
			\item When the lector or subdeacon is to read, he is to stand directly before the altar and from there proclaim the reading. Likewise the priest or officiant is to stand in the same place when praying on behalf of the congregation as directed.
			
			\item When the rulers are to sing, there are to proceed to the midst of the quire, standing between the two sides and before the altar.
			
		\vesectionp{To and From the Quire}
		\end{enumerate}
\end{document}