\documentclass[14pt,twoside]{extarticle}
\usepackage{moresize}
\usepackage{fontspec}
\usepackage{aeguill}
\usepackage{titlesec}
\usepackage[maxlevel=3]{csquotes}
\usepackage{lettrine}
\usepackage{xifthen}
\usepackage{yfonts}
\usepackage{parskip}
\usepackage[dvipsnames]{xcolor}
\usepackage{fancyhdr}
\usepackage{hanging}
\usepackage[autocompile,allowdeprecated=false]{gregoriotex}
\usepackage[twoside]{geometry}
\usepackage{multicol}
\usepackage{paracol}
\usepackage[latin,english]{babel}
\usepackage{enumitem}
\usepackage{tabto}
\usepackage{titlesec}
\usepackage[savepos]{zref}
\usepackage{setspace}
\usepackage{xstring}
\def\ReplaceStr#1{%
	\IfSubStr{#1}{\newline}{%
		\StrSubstitute{#1}{\newline}{ }}{#1}}
%\usepackage{libertine}

\setlength{\columnseprule}{0.025pt}
\setlength{\columnsep}{1.5em}
\setlength{\topskip}{7pt}

\def\vehymn{-0.50}

\makeatletter
\patchcmd{\pcol@buildcolseprule}%
{\hrule\@height\@tempdima\@width\columnseprule}%
{\hrule\@height\dimexpr\@tempdima-0.5ex\@width\columnseprule\@depth-1pt}{}{}
\makeatother

\gredefsizedsymbol{GreCross}{greextra}{Cross}
\gredefsizedsymbol{GreDagger}{greextra}{Dagger}
\gredefsizedsymbol{GreCrossAlt}{greextra}{Cross.alt}
\gredefsizedsymbol{GreStarHeight}{greextra}{StarHeight}
\gredefsizedsymbol{greABar}{greextra}{ABar}
\gredefsizedsymbol{greVBar}{greextra}{VBar}
\gredefsizedsymbol{greRBar}{greextra}{RBar}

\gresetspecial{r}{\textcolor{red}{\Rbar{}.}}
\gresetspecial{v}{\textcolor{red}{\Vbar{}.}}
\gresetspecial{a}{\textcolor{red}{\Abar{}.}}
\gresetspecial{+}{\textcolor{red}{\GreCross{17}}}
\gresetspecial{1}{\textcolor{red}{\GreDagger{17}}}
\gresetspecial{2}{\textcolor{red}{\GreCrossAlt{17}}}
\gresetspecial{*}{\textcolor{red}{\GreStarHeight{17}}}
%%% for spacing in chant %%%
\gresetspecial{ }{\textcolor{white}{x}}
\gresetspecial{  }{\textcolor{white}{xx}}
\gresetspecial{   }{\textcolor{white}{xxx}}
\gresetspecial{    }{\textcolor{white}{xxxx}}

\grechangestaffsize{30}
%\grechangedim{abovelinestextheight}{0.25cm}{fixed}
%\grechangedim{spacebeneathtext}{0.5cm}{fixed}
%\grechangedim{annotationseparation}{0.25cm}{fixed}


\newcommand{\skiplines}[1]{\pagebreak[1]\vspace*{#1\baselineskip}}

\makeatletter
\newcommand{\veformat}[1]{\textcolor{red}{\textbf{#1}}}
\newcommand{\veant}[2]{\greannotation{#1. \veformat{Ant.}}\greannotation{\veformat{#2.}}}
\newcommand{\instruct}[1]{\textcolor{red}{\textit{#1}}}

\newcommand{\whiteribbon}[0]{\textcolor{black}{\textbf{white ribbon}} }%
\newcommand{\greenribbon}[0]{\textcolor{ForestGreen}{\textbf{green ribbon}} }%
\newcommand{\redribbon}[0]{\textcolor{red}{\textbf{red ribbon}} }%
\newcommand{\blueribbon}[0]{\textcolor{blue}{\textbf{blue ribbon}} }%
\makeatother

%%% HEADERS %%%
\makeatletter
\newcommand{\vepart}[1]{%
	\cleardoublepage%
	\vspace*{\fill}%
	\begin{center}%
		\setstretch{2}
		{\Huge \selectfont \uppercase{ \textbf{#1}}}\par%
	\end{center}%
	\vspace*{\fill}%
	\fancyhead{}
}%
\makeatother

\newcommand{\vechapter}[1]{%
	\cleardoublepage%
	\fancyhead[RO,RE]{text}
	\StrSubstitute{#1}{\\}{\space}[\vetempmacrolol]%
	\fancyhead[LO,LE]{\vetempmacrolol}%
	\begin{center}
		\setstretch{1.5}
		{\LARGE \selectfont \color{red} \uppercase{ \textbf{#1}}}\par%
	\end{center}
	\skiplines{1}
}%
\makeatother

\newcommand{\vesectionp}[1]{%
	\skiplines{1}
	\fancyhead[RO,RE]{#1}%
	\begin{center}
		\setstretch{1.5}
		{\Large \selectfont \uppercase{ \textbf{#1}}}\par%
	\end{center}
	\skiplines{1}
}%
\makeatother

\newcommand{\vesection}[1]{%
	\clearpage\skiplines{-2}\vesectionp{#1}%
}%
\makeatother

\makeatletter
\newcommand{\veheading}[1]{%
	\skiplines{1}%
	\begin{center}
		\setstretch{1.5}
		{\selectfont \textit{#1}}\par%
	\end{center}
	\skiplines{1}
}%
\makeatother

\makeatletter
\newcommand{\vecomment}[1]{%
	\hfill{\small\selectfont \textit{#1}.}%
}%
\makeatother

%%% OTHER FORMATS %%%

\makeatletter
\newcommand{\red}[1]{%
	{\color{red} #1}%
}%
\makeatother

\makeatletter
\newcommand{\chapter}[1]{%
	\gresetinitiallines{1}%
	\greannotation{\veformat{Chap.}}%
	\greannotation{\veformat{Dir.}}%
	\gabcsnippet{#1}%
}%
\makeatother

\makeatletter
\newcommand{\resp}[2]{%
	\vecomment{#1}
	
	\gresetinitiallines{1}%
	\greannotation{\veformat{Resp.}}%
	\greannotation{\veformat{#2.}}%
}%
\makeatother

\makeatletter
\newcommand{\tone}[2]{%
	\greannotation{\veformat{Ps.}}
	\greannotation{\veformat{#1.}}
	
	\gabcsnippet{#2}
}%
\makeatother

\makeatletter
\newcommand{\repant}[1]{%
	\gresetspecial{*}{}
	\nopagebreak[4]\skiplines{1}\nopagebreak[4]\gresetinitiallines{0}\gregorioscore{#1}\gresetinitiallines{1}
	\gresetspecial{*}{\textcolor{red}{\GreStarHeight{17}}}
}%
\makeatother



%%% FANCY %%%
\makeatletter  
\newcounter{score}
\newcounter{tabstop}[score]
\newcommand{\grealign}{%
	\@bsphack%
	\ifgre@boxing\else%
	\kern\gre@dimen@begindifference%
	\stepcounter{tabstop}%
	\expandafter\zsavepos{stop-\thescore-\thetabstop}%
	\kern-\gre@dimen@begindifference%
	\fi%
	\@esphack%
}

\newcommand{\setstops}{%
	\gdef\nstabbing@stops{%
		\hspace*{-\oddsidemargin}\hspace{-1in}%
		\hspace*{\zposx{stop-\thescore-1} sp}\=%
	}%
	\count@=\@ne
	\loop\ifnum\count@<\value{tabstop}%
	\begingroup\edef\x{\endgroup
		\noexpand\g@addto@macro\noexpand\nstabbing@stops{%
			\noexpand\hspace{-\noexpand\zposx{stop-\thescore-\the\count@} sp}%
			\noexpand\hspace{\noexpand\zposx{stop-\thescore-\the\numexpr\count@+1} sp}\noexpand\=%
		}%
	}\x
	\advance\count@\@ne
	\repeat
	\nstabbing@stops\kill
}
\makeatother

\newenvironment{nstabbing}
{\setlength{\topsep}{0pt}%
	\setlength{\partopsep}{0pt}%
	\tabbing%
	\setstops}
{\endtabbing\stepcounter{score}}

\begin{document}
	\newgeometry{margin=0pt}
	\begin{titlepage}
		\vspace*{\fill}
		
		\begin{center}
			
			\textbf{\Huge\color{red} CUSTOMARY}
			
			ACCORDING TO THE RITE OF THE
			
			\textbf{\LARGE HOLY ROMAN CHURCH}
			
			\vspace*{\fill}
			
			Pertaining to the celebration of the\\Liturgy of the Hours
			
		\end{center}
		
		\vspace*{\fill}
	\end{titlepage}
	
	\newgeometry{top=0.75in,bottom=0.75in,inner=0.75in,outer=0.75in,includefoot}
	
	\pagenumbering{arabic}
	\gresetinitiallines{0}
	
	\raggedbottom
	
	\vechapter{Liturgical Calendar}
	\skiplines{-1}\vesectionp{January}
	\begin{enumerate}[noitemsep,nolistsep]
		\item \red{Circumcision of the Lord.} Major feast.
		\item Octave of Saint Stephen. \red{9 lect. triplex.}
		\item Octave of Saint John. \red{9 lect. triplex.}
		\item Octave of the Innocents. \red{9 lect. triplex.}
		\item \textcolor{purple}{Eve of the Epiphany.} \red{9 lect. duplex}
		\item \red{\textbf{Epiphany of the Lord.}} Solemnity. \textit{With octave.}
		\item Of the Octave. \red{9 lect. duplex.}
		\item Of the Octave. \red{9 lect. duplex.}
		\item Of the Octave. \red{9 lect. duplex.}
		\item Of the Octave. \red{9 lect. duplex.}
		\item Of the Octave. \red{9 lect. duplex.}
		\item Of the Octave. \red{9 lect. duplex.}
		\item Octave of the Epiphany. \red{9 lect. triplex.}
		\item Felix, Priest and Martyr. \red{3 lect. simplex.}
		\item Paul, Hermit. \red{3 lect. simplex.}
		\item Marcellus, Pope and Martyr. \red{3 lect. simplex.}
		\item Anthony of Egypt,Abbot. \red{3 lect. simplex.}
		\item Prisca, Virgin and Martyr. \red{3 lect. simplex.}
		\item Marius, Martha, Audifax, and Abachum, Martyrs. \red{3 lect. simplex.}
		\item Fabian and Sebastian, Martyrs. \red{9 lect. duplex.}
		\item Agnes, Virgin and Martyr. \red{9 lect. duplex.}
		\item Vincent and Anastasius, Martyrs. \red{9 lect. duplex.}
		\item Emerentiana, Virgin and Martyr. \red{3 lect. simplex.}
		\item Timothy, Bishop and Martyr. \red{3 lect. simplex.}
		\item Conversion of Saint Paul, Apostle. \red{9 lect. triplex.}
		\item Polycarp, Bishop and Martyr. \red{3 lect. simplex.}
		\item John Chrysostom, Bishop and Doctor. \red{9 lect. duplex.}
		\item Thomas Aquinas, Priest and Doctor. \red{9 lect. duplex.}
		\item % 29
		\item Gregory Nazianzus,  Bishop and Doctor. \red{9 lect. duplex.}
		\item % 31
	\end{enumerate}

\vesectionp{February}	
	
	\begin{enumerate}[nolistsep,noitemsep]
		
		\item Ignatius, Bishop and Martyr. \red{3 lect. duplex.}
		\item \red{Purification of the Blessed Virgin Mary.} Principle feast.
		\item Blase, Bishop and Martyr. \red{3 lect. simplex.}
		\item % 4
		\item Agatha, Virgin and Martyr. \red{9 lect. duplex.}
		\item Dorothy, Virgin and Martyr.\red{3 lect. simplex.}
		\item % 7
		\item % 8
		\item Apollonia, Virgin and Martyr.\red{3 lect. simplex.}
		\item Scholastica, Virgin. \red{3 lect. simplex.}
		\item % 11
		\item % 12
		\item % 13
		\item Valentine, Priest and Martyr. \red{3 lect. simplex.}
		\item Faustinus and Jovita, Martyrs.
		\item % 16
		\item % 17
		\item Simeon, Bishop and Martyr. \red{3 lect. simplex.}
		\item % 19
		\item % 20
		\item % 21
		\item Cathedra of Saint Peter, Apostle. \red{9 lect. triplex.}
		\item % 23
		\item % 24
		\item % 25
		\item % 26
		\item % 27
		\item % 28
	\end{enumerate}
	
	\vesectionp{March}
	
	\begin{enumerate}[nolistsep,noitemsep]
		\item % 1
		\item % 2
		\item % 3
		\item % 4
		\item % 5
		\item % 6
		\item Perpetua and Felicity, Martyrs. \red{3 lect. simplex.}
		\item % 8
		\item Forty Saints of Sebaste, Martyrs. \red{3 lect. simplex.}
		\item % 10
		\item % 11
		\item \red{Gregory, Pope and Doctor.} Minor feast.
		\item % 13
		\item % 14
		\item % 15
		\item % 16
		\item Patrick, Bishop and Confessor. \red{9 lect. duplex.}
		\item % 18
		\item \red{Joseph, Spouse of the Blessed Virgin Mary. } Minor feast.
		\item % 20
		\item Benedict of Nursia, Abbot. \red{9 lect. duplex.}
		\item % 22
		\item % 23
		\item % 24
		\item \red{Annunciation of the Blessed Virgin Mary.} Major feast.
		\item % 26
		\item % 27
		\item % 28
		\item % 29
		\item % 30
		\item % 31
		
	\end{enumerate}
	
	\vechapter{Order of the Chapel}
	
	\skiplines{-1}\vesectionp{Ministers of the Liturgy}
	
		\begin{enumerate}
		
			\item There are seven orders of liturgical ministers: porters, cantors, lectors, acolytes, subdeacons, deacons, and presbyters. The first four are minor orders and only require deputation, even by necessity; the last three are the major orders which require either institution, in the case of the subdiaconate\footnote{Training as an acolyte in the Mass is sufficient.}, or ordination, as the case of deacons and presbyters.
			
			\item The celebration of the Mass must in all cases by served by a priest, assisted by deacon and subdeacon; the celebration of the hours is best overseen by a priest but may be licitly celebrated without any oversight by only one subdeacon, one lector, and at least two cantors.
			
			\item However, where extreme case arises where there is a lack of suitable ministers but the faithful still demand a liturgical celebration, the priest may licitly assume the roles of deacon and subdeacon; in the hours, the subdeacon and lector may assume additionally the roles of cantors. This should never be tolerated as a regular situation.
			
			\item Liturgical ministry is best reserved for men, but where this is inexpedient or imprudent, woman may licitly perform the ministries of porter and cantor. In extreme cases, even the ministry of lector may be performed by women, but that of acolytes, subdeacons, deacons, and presbyters must be reserved to men alone, by the divine law.
			
			\item Among those found willing and suited for liturgical ministry, the campus minister or his deputy chooses for the year the dean who is responsible for the maintenance of the liturgy according to the books and the organization of its ministers.
			
			\item Each week, the dean names a hebdomadary who assumes the ministry of subdeacon in the overseeing of the hours in absence of a priest.  The dean himself performs the ministry of subdeacon in overseeing the hours on all days ranked minor feast or higher; the first Sunday of Advent, Ash Wednesday, Palm Sunday, the Triduum, and the eve of Pentecost.
			
			\item The hebdomadary for each day of the week must ensure that there is at least two cantors and one lector for all the hours to be celebrated in the chapel. The hebdomadary himself performs the ministry of subdeacon in overseeing the liturgy in the usual case, but may name a replacement if he is unable.
			
		\vesectionp{Services in the Chapel}
				
		\item The chapel should celebrate in choir daily Lauds, Vespers, and Compline. All the more on Saturdays, after Compline, the Vigils of Sunday should also be sung, as well as on the major feast days.
		
		\item Insofar as is possible and expedient, all services are to be offered at their solar times. Lauds thus at sunrise, Prime two hours after sunrise, Terce three hours, Sext six hours, and None nine hours; Vespers at sunset, Compline two hours after sunset, and Vigils six hours after sunset.
		
		\item An hour is not a period of sixty minutes, but hte daytime hours are the time between sunrise and sunset split into twelve equal periods. Liewise, the night-time hours are the time between sunset and the next sunrise split into twelve equal parts.
		
		\item However, Vigils may take the place of Compline or be prepended to Lauds, which may be transferred at most until the time appointed for Prime. Vespers may be slightly anticipated, but should begin or end at or after sunset.
		
		\item When Mass interferes with the appointed time for Vespers, it should be said immediately after Mass and Compline should be omitted. But on other days, the hour of Prime, Terce, Sext, or None should be said before Mass, or if constraints do not permit it, after Mass according to the time of day.
		\end{enumerate}
\end{document}