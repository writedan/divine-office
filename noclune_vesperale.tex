\documentclass[14pt,twoside]{extarticle}
\usepackage{moresize}
\usepackage{fontspec}
\usepackage{aeguill}
\usepackage{titlesec}
\usepackage[maxlevel=3]{csquotes}
\usepackage{lettrine}
\usepackage{xifthen}
\usepackage{yfonts}
\usepackage{parskip}
\usepackage[dvipsnames]{xcolor}
\usepackage{fancyhdr}
\usepackage{hanging}
\usepackage[autocompile,allowdeprecated=false]{gregoriotex}
\usepackage[twoside]{geometry}
\usepackage{multicol}
\usepackage{paracol}
\usepackage[latin,english]{babel}
\usepackage{enumitem}
\usepackage{tabto}
\usepackage{titlesec}
\usepackage[savepos]{zref}
%\usepackage{libertine}

\setlength{\columnseprule}{0.025pt}
\setlength{\columnsep}{1.5em}
\setlength{\topskip}{7pt}

\def\vehymn{-0.50}

\makeatletter
\patchcmd{\pcol@buildcolseprule}%
{\hrule\@height\@tempdima\@width\columnseprule}%
{\hrule\@height\dimexpr\@tempdima-0.5ex\@width\columnseprule\@depth-1pt}{}{}
\makeatother

\gredefsizedsymbol{GreCross}{greextra}{Cross}
\gredefsizedsymbol{GreDagger}{greextra}{Dagger}
\gredefsizedsymbol{GreCrossAlt}{greextra}{Cross.alt}
\gredefsizedsymbol{GreStarHeight}{greextra}{StarHeight}
\gredefsizedsymbol{greABar}{greextra}{ABar}
\gredefsizedsymbol{greVBar}{greextra}{VBar}
\gredefsizedsymbol{greRBar}{greextra}{RBar}

\gresetspecial{r}{\textcolor{red}{\Rbar{}.}}
\gresetspecial{v}{\textcolor{red}{\Vbar{}.}}
\gresetspecial{a}{\textcolor{red}{\Abar{}.}}
\gresetspecial{+}{\textcolor{red}{\GreCross{17}}}
\gresetspecial{1}{\textcolor{red}{\GreDagger{17}}}
\gresetspecial{2}{\textcolor{red}{\GreCrossAlt{17}}}
\gresetspecial{*}{\textcolor{red}{\GreStarHeight{17}}}
%%% for spacing in chant %%%
\gresetspecial{ }{\textcolor{white}{a}}
\gresetspecial{  }{\textcolor{white}{be}}
\gresetspecial{   }{\textcolor{white}{God}}
\gresetspecial{    }{\textcolor{white}{text}}

\grechangestaffsize{30}
%\grechangedim{abovelinestextheight}{0.25cm}{fixed}
%\grechangedim{spacebeneathtext}{0.5cm}{fixed}
\grechangedim{annotationseparation}{0.25cm}{fixed}

\newcommand{\skiplines}[1]{\vspace*{#1\baselineskip}}

\newcommand{\veformat}[1]{\textcolor{red}{\textbf{#1}}}
\newcommand{\veant}[2]{\greannotation{#1. \veformat{Ant.}}\greannotation{\veformat{#2.}}}
\newcommand{\instruct}[1]{\textcolor{red}{\textit{#1}}}

\makeatletter
\newcommand{\whiteribbon}[0]{\textcolor{black}{\textbf{white ribbon}} }%
\newcommand{\greenribbon}[0]{\textcolor{ForestGreen}{\textbf{green ribbon}} }%
\newcommand{\redribbon}[0]{\textcolor{red}{\textbf{red ribbon}} }%
\newcommand{\blueribbon}[0]{\textcolor{blue}{\textbf{blue ribbon}} }%
\makeatother

%%% HEADERS %%%
\newcommand{\vepart}[1]{%
	\cleardoublepage%
	\vspace*{\fill}%
	\begin{center}%
		{\Huge \selectfont \uppercase{ \textbf{#1}}}%
	\end{center}%
	\vspace*{\fill}%
}%
\makeatother

\newcommand{\vechapter}[1]{%
	\cleardoublepage%
	\begin{center}
		{\LARGE \selectfont \color{red} \uppercase{ \textbf{#1}}}\par%
	\end{center}
	\skiplines{1}
}%
\makeatother

\newcommand{\vesectionp}[1]{%
	\skiplines{1}
	\begin{center}
		{\Large \selectfont \uppercase{ \textbf{#1}}}%
	\end{center}
	\skiplines{1}
}%
\makeatother

\newcommand{\vesection}[1]{%
	\clearpage\skiplines{-2}\vesectionp{#1}%
}%
\makeatother

\makeatletter
\newcommand{\veheading}[1]{%
	\skiplines{0.5}%
	\begin{center}
		{\selectfont \textit{#1}}%
	\end{center}
	\skiplines{0.5}
}%
\makeatother

\makeatletter
\newcommand{\vecomment}[1]{%
	\hfill{\small\selectfont \textit{#1}.}%
}%
\makeatother

%%% OTHER FORMATS %%%

\makeatletter
\newcommand{\red}[1]{%
	{\color{red} #1}%
}%
\makeatother

\makeatletter
\newcommand{\chapter}[1]{%
	\gresetinitiallines{1}%
	\greannotation{\veformat{Chap.}}%
	\greannotation{\veformat{Dir.}}%
	\gabcsnippet{#1}%
}%
\makeatother

\makeatletter
\newcommand{\resp}[2]{%
	\vecomment{#1}
	
	\gresetinitiallines{1}%
	\greannotation{\veformat{Resp.}}%
	\greannotation{\veformat{#2.}}%
}%
\makeatother

\makeatletter
\newcommand{\tone}[2]{%
	\greannotation{\veformat{Ps.}}
	\greannotation{\veformat{#1.}}
	
	\gabcsnippet{#2}
}%
\makeatother

\makeatletter
\newcommand{\repant}[1]{%
	\gresetspecial{*}{}
	\nopagebreak[4]\skiplines{1}\nopagebreak[4]\gresetinitiallines{0}\gregorioscore{#1}\gresetinitiallines{1}
	\gresetspecial{*}{\textcolor{red}{\GreStarHeight{17}}}
}%
\makeatother



%%% FANCY %%%
\makeatletter  
\newcounter{score}
\newcounter{tabstop}[score]
\newcommand{\grealign}{%
	\@bsphack%
	\ifgre@boxing\else%
	\kern\gre@dimen@begindifference%
	\stepcounter{tabstop}%
	\expandafter\zsavepos{stop-\thescore-\thetabstop}%
	\kern-\gre@dimen@begindifference%
	\fi%
	\@esphack%
}

\newcommand{\setstops}{%
	\gdef\nstabbing@stops{%
		\hspace*{-\oddsidemargin}\hspace{-1in}%
		\hspace*{\zposx{stop-\thescore-1} sp}\=%
	}%
	\count@=\@ne
	\loop\ifnum\count@<\value{tabstop}%
	\begingroup\edef\x{\endgroup
		\noexpand\g@addto@macro\noexpand\nstabbing@stops{%
			\noexpand\hspace{-\noexpand\zposx{stop-\thescore-\the\count@} sp}%
			\noexpand\hspace{\noexpand\zposx{stop-\thescore-\the\numexpr\count@+1} sp}\noexpand\=%
		}%
	}\x
	\advance\count@\@ne
	\repeat
	\nstabbing@stops\kill
}
\makeatother

\newenvironment{nstabbing}
{\setlength{\topsep}{0pt}%
	\setlength{\partopsep}{0pt}%
	\tabbing%
	\setstops}
{\endtabbing\stepcounter{score}}

\begin{document}
	\newgeometry{margin=0pt}
	\begin{titlepage}
		\vspace*{\fill}
		
		\begin{center}
			
			\textbf{\Huge\color{red} VESPERALE}
			
			ACCORDING TO THE RITE OF THE
			
			\textbf{\LARGE HOLY ROMAN CHURCH}
			
			\vspace*{\fill}
			
			\textbf{\Large\color{red} VOLUME I}
			
			Seasons of Advent and Christmas\\Until the Epiphany of the Lord
			
		\end{center}
		
		\vspace*{\fill}
	\end{titlepage}
	
	\newgeometry{top=0.75in,bottom=0.75in,inner=0.75in,outer=0.75in,includefoot}
	
	\pagenumbering{arabic}
	\gresetinitiallines{0}
	
	\raggedbottom
	
	%\tableofcontents
	
	
	%%% COPY ABOVE %%%
	
	\vechapter{Liturgical Calendar}
	\skiplines{-1}\vesectionp{November}
		\begin{enumerate}[noitemsep, nolistsep]
			\setcounter{enumi}{26}
			\item {\textit{\small Earliest day for Advent.}}
			\item % 28
			\item \footnote{Outside of Advent, here is celebrated the Eve of Saint Andrew.}
			\item \red{Saint Andrew, Apostle.} Minor feast.
		\end{enumerate}
	\vesectionp{December}
		\begin{enumerate}[noitemsep, nolistsep]
			\item %\textit{\small Saint Nahum, Prophet.} % 1
			\item %\textit{\small Saint Habakkuk, Prophet.}  % 2
			\item \textit{\small Latest day for Advent.} %\textit{\small Saint Zephaniah, Prophet.}  % 3
			\item  % 4
			\item % 5
			\item Saint Nicholas, Bishop. \red{9 lect. duplex.}
			\item Saint Ambrose, Bishop and Doctor. \red{9 lect. triplex.}
			\item \red{Conception of the Blessed Virgin Mary.} Major feast.
			\item % 9
			\item % 10
			\item % 11
			\item  % 12
			\item Saint Lucy, Virgin and Martyr. \red{9 lect. duplex.}
			\item  % 14
			\item % 15
			\item %\textit{\small Saint Haggai, Prophet.}  % 16
			\item {\textit{\small O Wisdom.}} % 17
			\item  % 18
			\item  % 19
			\item  % 20
			\item %\textit{\small Saint Micah, Prophet.}  % 21
			\item  % 22
			\item  % 23
			\item \textcolor{purple}{\textbf{Eve of the Nativity.}}
			\item \textbf{\red{Nativity of the Lord.}} Solemnity. \textit{With octave.}
			\item \red{Saint Stephen, Protomartyr.} Major feast.
			\item \red{Saint John, Apostle and Evangelist.} Major feast.
			\item \red{Holy Innocents, Martyrs.} Major feast.
			\item \red{Saint Thomas Becket, Martyr.} Major feast.
			\item Of the Octave of the Nativity. % 30
			\item Of the Octave of the Nativity.
		\end{enumerate}
	\vesectionp{January}
		\begin{enumerate}[noitemsep, nolistsep]
			\item \red{Circumcision of the Lord.} Principle feast.
			\item Octave of Saint Stephen. \red{9 lect. triplex.}
			\item Octave of Saint John. \red{9 lect. triplex.}
			\item Octave of the Innocents. \red{9 lect. triplex.}
			\item \textcolor{purple}{\textbf{Eve of the Epiphany.}}
			\item \red{\textbf{Epiphany of the Lord.}} Solemnity.\footnote{On the following days, until the Sunday after Epiphany, its office is continued.}
		\end{enumerate}
	
	
	\gresetinitiallines{1}
	
	\vepart{Ordinary}
	
		\vechapter{The Order of Vespers}
			\skiplines{-1}\vesectionp{Introduction}
			
				\instruct{Before the service begins, kneel and make these prayers quietly.}
				
				Our Father, who art in heaven, hallowed be thy name. Thy kingdom come. Thy will be done on earth as it is in heaven. Give us this day our daily bread and forgive us our trespasses as we forgive those who trespass against us. And lead us not into temptation but deliver us from evil.
					
				Hail Mary, full of grace, the Lord is with thee. Blessed art thou among women and blessed is the fruit of thy womb, Jesus. Holy Mary, Mother of God, pray for us sinners now and at the hour of our death. Amen.
				
				\instruct{Stand as the minister begins from his place:}
				
				\gresetinitiallines{0}
				\skiplines{0.5}\gabcsnippet{(c3) <sp>v</sp> O(fv) God,(hv) make(hv) speed(iv) to(h) save(g) me.(hv) <sp>r</sp>(::) O(hv) Lord,(hv) make(hv) haste(hv) to(hv) help(g) me.(hv)}
				
				\instruct{The congregation continues together, turning toward the altar, making a profound bow, and crossing themselves at the invocation of the Holy Trinity here and whenever else:}
				
				\skiplines{0.5}\gabcsnippet{(c3) Glor(h)y(h) be(h) to(h) the(h) Fa(h)ther,(h) and(h) to(h) the(h) Son,(h) (,) and(h) to(h) the(h) Ho(h)ly(h) Spir(g)it.(hv) (:) As(h) it(h) was(h) in(h) the(h) be(h)gin(h)ning,(h) is(h) now(h) and(h) ev(h)er(h) shall(g) be:(hv) (,) world(hv) with(hv)out(hv) end.(hv) A(g)men.(hv) (::) A(hv)le(hi)lu(h)ia.(hg)}
				
			\vesectionp{Psalmody}
			
				\instruct{Turn to the \greenribbon and sing the psalms with their antiphons prescribed there in this manner.}
				
				\instruct{The minister intones all antiphons until the asterisk, when he is then joined by the whole choir. Once the antiphon has been intoned, all psalms are sung to its tone until another antiphon is began.}
				
				\instruct{Psalms themselves are sung in this manner: the precentor or succentor sings the verse until the asterisk, when he is then joined by that entire side of the congregation. The other then begins the next verse, and that side of the congregation joins after the asterisk.}
				
				\instruct{Once a psalm is complete, if another antiphon is to be sung, first the previous antiphon is repeated. This is usually after every psalm. that is, each antiphon has one psalm under it, but on occasions multiple psalms are sung to the same antiphon. In this case, the antiphon is only repeated once all its psalms have been sung.}
				
			\vesection{Chapter}
			
				\instruct{Once the psalmody is finished, the lector will read the chapter from his place. This is a short reading of Scripture, no more than three verses in length. The congregation responds:}
				
				\skiplines{0.5}\gabcsnippet{(c3) Thanks(h) be(h) to(h) God.(h)}
				
				\instruct{There is usually a responsory to accompany the chapter, which is sung in this manner.}
				
				\instruct{The lector will sing the responsory until the asterisk, when he is joined by the whole choir. Finishing this, the lector will sing any verses to which the whole congregation sings the prescribed response.}
				
				\instruct{Even when the \enquote{Glory be} is sung, the lector sings it alone, but all the congregation turns toward the altar, makes a profound bow, and crosses themselves as it is sung.}
				
				\instruct{After the responsory is the hymn. One of the cantors will intone the first line of the hymn when he is joined by the whole congregation.}
				
				\instruct{After the hymn is a versicle which the minister proclaims the verse and the whole congregation gives the response.}
				
				\instruct{Turn to the \redribbon and sing the responsory, if any, hymn, and versicle.}
				
			\vesection{Magnificat}
			
				\instruct{After the versicle has been sung, the minister proceeds to stand before the altar and intones the antiphon to the Canticle of Mary until the asterisk, when he is joined by the whole choir. The precentor, succentor, and congregation sings the verses of the canticle in like manner as the psalms.}
				
				\instruct{While the canticle is sung, where possible, the minister incenses the altar.}
				
			\vesectionp{Preces}
			
				\instruct{The preces are said on feria, that is, days besides Sunday when no feast day occurs. But they are not said from December 17 until after the Epiphany. The whole congregation kneels.}
				
				\instruct{The precentor and succentor alternate singing \enquote{Kyrie, eleison} to this melody. Each invocation is sung thrice.}
				
				\skiplines{0.5}\gregorioscore{common/preces/kyrie.gabc}
				
				\instruct{The first invocation and the last are sung by the precentor.}
				
				\instruct{Make this prayer quietly:}
				
				Our Father, who art in heaven, hallowed be thy name. Thy kingdom come. Thy will be done on earth as it is in heaven. Give us this day our daily bread and forgive us our trespasses as we forgive those who trespass against us. And lead us not into temptation but deliver us from evil.
				
				\instruct{Then the petitions are said:}
				
				\skiplines{0.5}\gregorioscore{common/preces/petitions.gabc}
				
				\instruct{This psalm is then recited by the congregation in a low voice:}
				
				\veheading{Psalm l: Miserere mei Deus}
				The King shall rejoice in thy strength, O Lord * exceedingly glad shall he be of thy salvation.

Thou hast given him his heart's desire * and hast not denied him the request of his lips.

For thou shall prevent him with the blessings of goodness * and shall set a crown of pure gold upon his head.

He asked life of thee, and thou gavest him a long life * even forever and ever.

His honor is great in thy salvation * glory and great worship shall thou lay upon him.

For thou shalt given him everlasting felicity * and make him glad with the joy of thy countenance.

And why? because the King putteth his trust in the Lord * and in the mercy of the most High he shall not miscarry.

All thine enemies shall feel thy hand * thy right and shall find out them who hate thee.

Thou shall make them like a fiery oven in the time of thy wrath * the Lord shall destroy them in his displeasure, and the fire shall consume them.

Their fruit shall thou root out of the earth * and their seed from among the children of men.

For they intended mischief against thee * and imagined such a device as they are not able to perform.

Therefore shall thou put them to flight * and the strings of thy bow shall thou make ready against the face of them.

Be thou exalted, O Lord, in thine own strength * so will we sing, and praise thy power.
				
				\instruct{The priest alone stands and proceeds to stand before the altar, where he proclaims this peition:}
				
				\skiplines{0.5}\gregorioscore{common/preces/conclusion.gabc}
				
			\vesection{Conclusion}
			
				\instruct{The minister proceeds again to stand before the altar and says this versicle:}
				
				\skiplines{0.5}\gabcsnippet{(c3) <sp>v</sp> O(h) Lord,(h) hear(h) my(f) prayer.(g) <sp>r</sp>(::) And(h) let(h) my(h) cry(h) come(h) un(h)to(f) thee.(f)}
				
				\instruct{He will then proclaim the invocation \enquote{Let us pray}, after which he will genuflect. If the preces have been said, the congregation kneels during the collect; otherwise they remain standing.}
				
				\instruct{The minister will then pray the collect of the day. The congregation responds:}
				
				\skiplines{0.5}\gabcsnippet{(c3) A(gh)men.(h)}
				
				\instruct{This petition concludes the office:}
				
				\skiplines{0.5}\gregorioscore{common/conclusion.gabc}
				
				\instruct{Then kneel and make this prayer quietly:}
				
				Our Father, who art in heaven, hallowed be thy name. Thy kingdom come. Thy will be done on earth as it is in heaven. Give us this day our daily bread and forgive us our trespasses as we forgive those who trespass against us. And lead us not into temptation but deliver us from evil.
			
		\vechapter{The Order of Compline}
	
	
	\gresetinitiallines{1}
	
	\vepart{Psalter}
		\vechapter{Sunday at Second Vespers}
	\veant{1}{I. v}
	\gregorioscore{ant/sede_a_dextris.gabc}
	
	\skiplines{-1}\veheading{Psalm cix: Dixit Dominus}
	
	\tone{I. v}{
		(f3) The(h) Lord(iv) (:?) said(jrv) un()to(jv) <b>my</b>(i) Lord:(jv) <sp>*</sp>(:) Sit(jrv) thou() on() my() right() hand,() until() I() make() thine() en()e(jv)<b>mies</b>(i) thy(h) foot(iv irv)stool.(ivHGF)
	}

	\skiplines{0.5}\input{psalter/cix/1-v}
	
	
	
	
	\veant{2}{Iv. vi}
	
	\skiplines{1}\gregorioscore{ant/fidelia_omnia.gabc}
	
	\skiplines{-1}\veheading{Psalm xc: Confitebor tibi}
	
	\tone{IV. vi}{
		(f3) I(jv) will(i) (:?) give(jvr) thanks() unto() the() Lord(jv) <b>with</b>(i) my(jv) whole(kv) heart:(jr j) <sp>*</sp>(:) se(jrv)cretly() among() the() faithful,() and() in() the() con()gre()ga(jv)tion.(i)
	}

	\begin{multicols}{2}
	His seed shall be migh\textbf{ty} upon earth * the generation of the faithful shall be blessed.
	
	Riches and plenteousness shall \textbf{be} in his house * and his righteousness endureth forever.
	
	Unto the godly there ariseth up light \textbf{in} the darkness * he is merciful, loving, and righteous.
	
	A good man is merci\textbf{ful}, and lendeth * and will guide his words with discretion.
	
	For he shall ne\textbf{ver} be moved * and the righteous shall be had in everlasting remembrance.
\end{multicols}
	
	
	
	
	\veant{3}{Iv. vi}
	
	\skiplines{1}\gregorioscore{ant/in_mandatis_ejus.gabc}
	
	\veheading{Psalm cxi: Beatus vir}
	
	\tone{IV. vi}{
		(f3) Bles(jv)sed(i) (:?) is(jrv) the() man() that(jv) fear(i)eth(jv) the(kv) Lord(jr j) <sp>*</sp>(:) he(jrv) hath() great() delight() in() his() command(jv)ments.(i)
	}

	\begin{multicols}{2}
	His seed shall be migh\textbf{ty} upon earth * the generation of the faithful shall be blessed.
	
	Riches and plenteousness shall \textbf{be} in his house * and his righteousness endureth forever.
	
	Unto the godly there ariseth up light \textbf{in} the darkness * he is merciful, loving, and righteous.
	
	A good man is merci\textbf{ful}, and lendeth * and will guide his words with discretion.
	
	For he shall ne\textbf{ver} be moved * and the righteous shall be had in everlasting remembrance.
\end{multicols}
	
	
	
	\veant{4}{VII. ii}
	
	\skiplines{1}\gregorioscore{ant/sit_nomen_domini.gabc}
	\clearpage\veheading{Psalm cxii: Laudate pueri}
	{\Large fix this}
	
	
	
	
	
	\veant{5}{T. Per}
	
	\skiplines{1}\gregorioscore{ant/nos_qui_vivimus.gabc}
	
	\veheading{Psalm cxiii: In exitu Israel}

	\gregorioscore{psalter/cxiii.gabc}
	
	\vepart{Temporale}
	
	\vepart{Sanctorale}
	
	\vepart{Commons}
\end{document}