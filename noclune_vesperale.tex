\documentclass[14pt,twoside]{extarticle}
\usepackage{moresize}
\usepackage{fontspec}
\usepackage{aeguill}
\usepackage{titlesec}
\usepackage[maxlevel=3]{csquotes}
\usepackage{lettrine}
\usepackage{xifthen}
\usepackage{yfonts}
\usepackage{parskip}
\usepackage[dvipsnames]{xcolor}
\usepackage{fancyhdr}
\usepackage{hanging}
\usepackage[autocompile,allowdeprecated=false]{gregoriotex}
\usepackage[twoside]{geometry}
\usepackage{multicol}
\usepackage{paracol}
\usepackage[latin,english]{babel}
\usepackage{enumitem}
\usepackage{tabto}
\usepackage{titlesec}
\usepackage[savepos]{zref}
\usepackage{setspace}
\usepackage{xstring}
\def\ReplaceStr#1{%
	\IfSubStr{#1}{\newline}{%
		\StrSubstitute{#1}{\newline}{ }}{#1}}
%\usepackage{libertine}

\setlength{\columnseprule}{0.025pt}
\setlength{\columnsep}{1.5em}
\setlength{\topskip}{7pt}

\def\vehymn{-0.50}

\makeatletter
\patchcmd{\pcol@buildcolseprule}%
{\hrule\@height\@tempdima\@width\columnseprule}%
{\hrule\@height\dimexpr\@tempdima-0.5ex\@width\columnseprule\@depth-1pt}{}{}
\makeatother

\gredefsizedsymbol{GreCross}{greextra}{Cross}
\gredefsizedsymbol{GreDagger}{greextra}{Dagger}
\gredefsizedsymbol{GreCrossAlt}{greextra}{Cross.alt}
\gredefsizedsymbol{GreStarHeight}{greextra}{StarHeight}
\gredefsizedsymbol{greABar}{greextra}{ABar}
\gredefsizedsymbol{greVBar}{greextra}{VBar}
\gredefsizedsymbol{greRBar}{greextra}{RBar}

\gresetspecial{r}{\textcolor{red}{\Rbar{}.}}
\gresetspecial{v}{\textcolor{red}{\Vbar{}.}}
\gresetspecial{a}{\textcolor{red}{\Abar{}.}}
\gresetspecial{+}{\textcolor{red}{\GreCross{17}}}
\gresetspecial{1}{\textcolor{red}{\GreDagger{17}}}
\gresetspecial{2}{\textcolor{red}{\GreCrossAlt{17}}}
\gresetspecial{*}{\textcolor{red}{\GreStarHeight{17}}}
%%% for spacing in chant %%%
\gresetspecial{ }{\textcolor{white}{x}}
\gresetspecial{  }{\textcolor{white}{xx}}
\gresetspecial{   }{\textcolor{white}{xxx}}
\gresetspecial{    }{\textcolor{white}{xxxx}}

\grechangestaffsize{30}
%\grechangedim{abovelinestextheight}{0.25cm}{fixed}
%\grechangedim{spacebeneathtext}{0.5cm}{fixed}
%\grechangedim{annotationseparation}{0.25cm}{fixed}


\newcommand{\skiplines}[1]{\pagebreak[1]\vspace*{#1\baselineskip}}

\makeatletter
\newcommand{\veformat}[1]{\textcolor{red}{\textbf{#1}}}
\newcommand{\veant}[2]{\greannotation{#1. \veformat{Ant.}}\greannotation{\veformat{#2.}}}
\newcommand{\instruct}[1]{\textcolor{red}{\textit{#1}}}

\newcommand{\whiteribbon}[0]{\textcolor{black}{\textbf{white ribbon}} }%
\newcommand{\greenribbon}[0]{\textcolor{ForestGreen}{\textbf{green ribbon}} }%
\newcommand{\redribbon}[0]{\textcolor{red}{\textbf{red ribbon}} }%
\newcommand{\blueribbon}[0]{\textcolor{blue}{\textbf{blue ribbon}} }%
\makeatother

%%% HEADERS %%%
\makeatletter
\newcommand{\vepart}[1]{%
	\cleardoublepage%
	\vspace*{\fill}%
	\begin{center}%
		\setstretch{2}
		{\Huge \selectfont \uppercase{ \textbf{#1}}}\par%
	\end{center}%
	\vspace*{\fill}%
	\fancyhead{}
}%
\makeatother

\newcommand{\vechapter}[1]{%
	\cleardoublepage%
	\fancyhead[RO,RE]{text}
	\StrSubstitute{#1}{\\}{\space}[\vetempmacrolol]%
	\fancyhead[LO,LE]{\vetempmacrolol}%
	\begin{center}
		\setstretch{1.5}
		{\LARGE \selectfont \color{red} \uppercase{ \textbf{#1}}}\par%
	\end{center}
	\skiplines{1}
}%
\makeatother

\newcommand{\vesectionp}[1]{%
	\skiplines{1}
	\fancyhead[RO,RE]{#1}%
	\begin{center}
		\setstretch{1.5}
		{\Large \selectfont \uppercase{ \textbf{#1}}}\par%
	\end{center}
	\skiplines{1}
}%
\makeatother

\newcommand{\vesection}[1]{%
	\clearpage\skiplines{-2}\vesectionp{#1}%
}%
\makeatother

\makeatletter
\newcommand{\veheading}[1]{%
	\skiplines{1}%
	\begin{center}
		\setstretch{1.5}
		{\selectfont \textit{#1}}\par%
	\end{center}
	\skiplines{1}
}%
\makeatother

\makeatletter
\newcommand{\vecomment}[1]{%
	\hfill{\small\selectfont \textit{#1}.}%
}%
\makeatother

%%% OTHER FORMATS %%%

\makeatletter
\newcommand{\red}[1]{%
	{\color{red} #1}%
}%
\makeatother

\makeatletter
\newcommand{\chapter}[1]{%
	\gresetinitiallines{1}%
	\greannotation{\veformat{Chap.}}%
	\greannotation{\veformat{Dir.}}%
	\gabcsnippet{#1}%
}%
\makeatother

\makeatletter
\newcommand{\resp}[2]{%
	\vecomment{#1}
	
	\gresetinitiallines{1}%
	\greannotation{\veformat{Resp.}}%
	\greannotation{\veformat{#2.}}%
}%
\makeatother

\makeatletter
\newcommand{\tone}[2]{%
	\greannotation{\veformat{Ps.}}
	\greannotation{\veformat{#1.}}
	
	\gabcsnippet{#2}
}%
\makeatother

\makeatletter
\newcommand{\repant}[1]{%
	\gresetspecial{*}{}
	\nopagebreak[4]\skiplines{1}\nopagebreak[4]\gresetinitiallines{0}\gregorioscore{#1}\gresetinitiallines{1}
	\gresetspecial{*}{\textcolor{red}{\GreStarHeight{17}}}
}%
\makeatother



%%% FANCY %%%
\makeatletter  
\newcounter{score}
\newcounter{tabstop}[score]
\newcommand{\grealign}{%
	\@bsphack%
	\ifgre@boxing\else%
	\kern\gre@dimen@begindifference%
	\stepcounter{tabstop}%
	\expandafter\zsavepos{stop-\thescore-\thetabstop}%
	\kern-\gre@dimen@begindifference%
	\fi%
	\@esphack%
}

\newcommand{\setstops}{%
	\gdef\nstabbing@stops{%
		\hspace*{-\oddsidemargin}\hspace{-1in}%
		\hspace*{\zposx{stop-\thescore-1} sp}\=%
	}%
	\count@=\@ne
	\loop\ifnum\count@<\value{tabstop}%
	\begingroup\edef\x{\endgroup
		\noexpand\g@addto@macro\noexpand\nstabbing@stops{%
			\noexpand\hspace{-\noexpand\zposx{stop-\thescore-\the\count@} sp}%
			\noexpand\hspace{\noexpand\zposx{stop-\thescore-\the\numexpr\count@+1} sp}\noexpand\=%
		}%
	}\x
	\advance\count@\@ne
	\repeat
	\nstabbing@stops\kill
}
\makeatother

\newenvironment{nstabbing}
{\setlength{\topsep}{0pt}%
	\setlength{\partopsep}{0pt}%
	\tabbing%
	\setstops}
{\endtabbing\stepcounter{score}}

\begin{document}
	\newgeometry{margin=0pt}
	\begin{titlepage}
		\vspace*{\fill}
		
		\begin{center}
			
			\textbf{\Huge\color{red} VESPERAL}
			
			ACCORDING TO THE RITE OF THE
			
			\textbf{\LARGE HOLY ROMAN CHURCH}
			
			\vspace*{\fill}
			
			\textbf{\Large\color{red} VOLUME I}
			
			Seasons of Advent and Christmas\\Through the Epiphany of the Lord
			
		\end{center}
		
		\vspace*{\fill}
	\end{titlepage}
	
	\newgeometry{top=0.75in,bottom=0.75in,inner=0.75in,outer=0.75in,includefoot}
	
	\pagenumbering{arabic}
	\gresetinitiallines{0}
	
	\raggedbottom
	
	%\tableofcontents
	
	
	%%% COPY ABOVE %%%
	
	
	\vechapter{Perpetual Calendar}
	\skiplines{-1}\vesectionp{November}
		\begin{enumerate}[noitemsep, nolistsep]
			\setcounter{enumi}{26}
			\item {\textit{\small Earliest day for Advent.}}
			\item % 28
			\item Saints Saturnius and Sisinnius, Martyrs.\footnote{Outside of Advent, here is celebrated the Eve of Saint Andrew.} \red{3 lect. simplex.}
			\item \red{Saint Andrew, Apostle.} Minor feast.
		\end{enumerate}
	\vesectionp{December}
		\begin{enumerate}[noitemsep, nolistsep]
			\item \textit{\small Saint Nahum, Prophet.} % 1
			\item \textit{\small Saint Habakkuk, Prophet.}  % 2
			\item \textit{\small Saint Zephaniah, Prophet.}  % 3
			\item  % 4
			\item % 5
			\item Saint Nicholas, Bishop and Confessor. \red{9 lect. duplex.}
			\item Saint Ambrose, Bishop and Doctor. \red{9 lect. triplex.}
			\item \red{Conception of the Blessed Virgin Mary.} Major feast.
			\item % 9
			\item % 10
			\item Saint Damasus, Pope and Confessor. \red{3 lect. duplex.} % 11
			\item  % 12
			\item Saint Lucy, Virgin and Martyr. \red{9 lect. duplex.}
			\item  % 14
			\item % 15
			\item \textit{\small Saint Haggai, Prophet.}  % 16
			\item {\textit{\small O Wisdom.}} % 17
			\item  % 18
			\item  % 19
			\item  % 20
			\item \textit{\small Saint Micah, Prophet.}  % 21
			\item  % 22
			\item  % 23
			\item \textcolor{purple}{Eve of the Nativity.} \red{9 lect. duplex.}
			\item \textbf{\red{Nativity of the Lord.}} Solemnity. \textit{With octave.}
			\item \red{Saint Stephen, Protomartyr.} Major feast.
			\item \red{Saint John, Apostle and Evangelist.} Major feast.
			\item \red{Holy Innocents, Martyrs.} Major feast.
			\item \red{Saint Thomas Becket, Martyr.} Major feast.
			\item Of the Octave. \red{9 lect. triplex.} % 30
			\item \red{Holy Forefathers of the Lord.} Principle feast.
		\end{enumerate}
	\vesectionp{January}
		\begin{enumerate}[noitemsep, nolistsep]
			\item \red{Circumcision of the Lord.} Major feast.
			\item Octave of Saint Stephen. \red{9 lect. triplex.}
			\item Octave of Saint John. \red{9 lect. triplex.}
			\item Octave of the Innocents. \red{9 lect. triplex.}
		\end{enumerate}
	
	
	\gresetinitiallines{1}
	
	\vepart{Ordinary}
	
		\vechapter{The Order of Vespers}
			\skiplines{-1}\vesectionp{Introduction}\label{vespers:introduction}
			
				\instruct{Before the service begins, kneel and make these prayers quietly.}
				
				Our Father, who art in heaven, hallowed be thy name. Thy kingdom come. Thy will be done on earth as it is in heaven. Give us this day our daily bread and forgive us our trespasses as we forgive those who trespass against us. And lead us not into temptation but deliver us from evil.
					
				Hail Mary, full of grace, the Lord is with thee. Blessed art thou among women and blessed is the fruit of thy womb, Jesus. Holy Mary, Mother of God, pray for us sinners now and at the hour of our death. Amen.
				
				\instruct{Stand as the minister begins from his place:}
				
				\gresetinitiallines{0}
				\skiplines{0.5}\gabcsnippet{(c3) <sp>v</sp> O(fv) God,(hv) make(hv) speed(iv) to(h) save(g) me.(hv) <sp>r</sp>(::) O(hv) Lord,(hv) make(hv) haste(hv) to(hv) help(g) me.(hv)}
				
				\instruct{The congregation continues together, turning toward the altar, making a profound bow, and crossing themselves at the invocation of the Holy Trinity here and whenever else:}
				
				\skiplines{0.5}\gabcsnippet{(c3) Glor(h)y(h) be(h) to(h) the(h) Fa(h)ther,(h) and(h) to(h) the(h) Son,(h) (,) and(h) to(h) the(h) Ho(h)ly(h) Spir(g)it.(hv) (:) As(h) it(h) was(h) in(h) the(h) be(h)gin(h)ning,(h) is(h) now(h) and(h) ev(h)er(h) shall(g) be:(hv) (,) world(hv) with(hv)out(hv) end.(hv) A(g)men.(hv) (::) A(hv)le(hi)lu(h)ia.(hg)}
				
			\vesectionp{Psalmody}\label{vespers:psalmody}
			
				\instruct{Turn to the \greenribbon and sing the psalms with their antiphons prescribed there in this manner.}
				
				\instruct{The minister intones all antiphons until the asterisk, when he is then joined by the whole choir. Once the antiphon has been intoned, all psalms are sung to its tone until another antiphon is began.}
				
				\instruct{Psalms themselves are sung in this manner: the precentor or succentor sings the verse until the asterisk, when he is then joined by that entire side of the congregation. The other then begins the next verse, and that side of the congregation joins after the asterisk.}
				
				\instruct{Once a psalm is complete, if another antiphon is to be sung, first the previous antiphon is repeated. This is usually after every psalm. that is, each antiphon has one psalm under it, but on occasions multiple psalms are sung to the same antiphon. In this case, the antiphon is only repeated once all its psalms have been sung.}
				
			\vesection{Chapter}\label{vespers:chapter}
			
				\instruct{Once the psalmody is finished, the lector will read the chapter from his place. This is a short reading of Scripture, no more than three verses in length. The congregation responds:}
				
				\skiplines{0.5}\gabcsnippet{(c3) Thanks(h) be(h) to(h) God.(h)}
				
				\instruct{There is usually a responsory to accompany the chapter, which is sung in this manner.}
				
				\instruct{The lector will sing the responsory until the asterisk, when he is joined by the whole choir. Finishing this, the lector will sing any verses to which the whole congregation sings the prescribed response.}
				
				\instruct{Even when the \enquote{Glory be} is sung, the lector sings it alone, but all the congregation turns toward the altar, makes a profound bow, and crosses themselves as it is sung.}
				
				\instruct{After the responsory is the hymn. One of the cantors will intone the first line of the hymn when he is joined by the whole congregation.}
				
				\instruct{After the hymn is a versicle which the minister proclaims the verse and the whole congregation gives the response.}
				
				\instruct{Turn to the \redribbon and sing the responsory, if any, hymn, and versicle.}
				
			\vesection{Magnificat}\label{vespers:magnificat}
			
				\instruct{After the versicle has been sung, the minister proceeds to stand before the altar and intones the antiphon to the Canticle of Mary until the asterisk, when he is joined by the whole choir. The precentor, succentor, and congregation sings the verses of the canticle in like manner as the psalms.}
				
				\instruct{While the canticle is sung, where possible, the minister incenses the altar.}
				
			\vesectionp{Preces}\label{vespers:preces}
			
				\instruct{The preces are said on feria, that is, days besides Sunday when no feast day occurs. But they are not said from December 17 until after the Epiphany. The whole congregation kneels.}
				
				\instruct{The precentor and succentor alternate singing \enquote{Kyrie, eleison} to this melody. Each invocation is sung thrice.}
				
				\skiplines{0.5}\gregorioscore{common/preces/kyrie.gabc}
				
				\instruct{The first invocation and the last are sung by the precentor.}
				
				\instruct{Make this prayer quietly:}
				
				Our Father, who art in heaven, hallowed be thy name. Thy kingdom come. Thy will be done on earth as it is in heaven. Give us this day our daily bread and forgive us our trespasses as we forgive those who trespass against us. And lead us not into temptation but deliver us from evil.
				
				\instruct{Then the petitions are said:}
				
				\skiplines{0.5}\gregorioscore{common/preces/petitions.gabc}
				
				\instruct{This psalm is then recited by the congregation in a low voice:}
				
				\veheading{Psalm l: Miserere mei Deus}
				Have mercy upon me, O God, after thy great mercy * according to the multitude of thy mercies do away mine offenses.
	
Wash me thoroughly from my wickedness * and cleanse me from my sin.
	
For I acknowledge my faults * and my sin is ever before me.
	
Against thee only have I sinned, and done this evil in thy sight * that thou mightest be justified in thy saying, and clear when thou art judged.

Behold, I was shapen in wickedness * and with sin hath my mother conceived me.
	
But lo, thou requirest truth in the inward parts * and shalt make me to understand wisdom secretly.
	
Thou shalt purge me with hyssop, and I shall be clean * thou shalt wash me and I shall be whiter than the snow.
	
Thou shalt make me hear of joy and gladness * that the bones which thou hast broken may rejoice.
	
Turn thy face from my sins * and put out all my misdeeds.
	
Make me a clean heart, O God * and renew a right spirit within me.
	
Cast me not away from thy presence * and take not thy Holy Spirit from me.
	
O give me the comfort of thy help again * and establish me with thy free Spirit.
	
Then shall I teach thy ways unto the wicked * and sinners shall be converted unto thee.
	
Deliver me from blood-guiltiness, O God, thou that art the God of my salvation * and my tongue shall sing of thy righteousness.
	
Thou shalt open my lips, O Lord * and my mouth shall shew thy praise.
	
For thou desirest no sacrifice, else would I give it thee * but thou delightest not in burnt-offerings.
	
The sacrifice of God is a troubled spirit * a broken and contrite heart, O God, shalt thou not despise.
	
O be favorable and gracious unto Zion * build thou the walls of Jerusalem.
	
Then shalt thou be pleased with the sacrifice of righteousness, with the burnt-offerings and oblations * then shall they offer young bullocks upon thine altar.

				
				\instruct{The priest alone stands and proceeds to stand before the altar, where he proclaims this peition:}
				
				\skiplines{0.5}\gregorioscore{common/preces/conclusion.gabc}
				
			\vesection{Conclusion}\label{vespers:conclusion}
			
				\instruct{The minister proceeds again to stand before the altar and says this versicle:}
				
				\skiplines{0.5}\gabcsnippet{(c3) <sp>v</sp> O(h) Lord,(h) hear(h) my(f) prayer.(g) <sp>r</sp>(::) And(h) let(h) my(h) cry(h) come(h) un(h)to(f) thee.(f)}
				
				\instruct{He will then proclaim the invocation \enquote{Let us pray}, after which he will genuflect. If the preces have been said, the congregation kneels during the collect; otherwise they remain standing.}
				
				\instruct{The minister will then pray the collect of the day. The congregation responds:}
				
				\skiplines{0.5}\gabcsnippet{(c3) A(gh)men.(h)}
				
				\instruct{This petition concludes the office:}
				
				\skiplines{0.5}\gregorioscore{common/conclusion.gabc}
				
				\instruct{Then kneel and make this prayer quietly:}
				
				Our Father, who art in heaven, hallowed be thy name. Thy kingdom come. Thy will be done on earth as it is in heaven. Give us this day our daily bread and forgive us our trespasses as we forgive those who trespass against us. And lead us not into temptation but deliver us from evil.
			
		\vechapter{The Order of Compline}
	
	
	\gresetinitiallines{1}
	
	\vepart{Psalter}
			
		\vechapter{Sunday at Vespers}
		
			\veant{1}{i. v}
\gregorioscore{ant/sede_a_dextris.gabc}

\greannotation{\veformat{Ps. 109}}
\greannotation{\veformat{i. v.}}
\skiplines{1}\gabcsnippet{
	(f3) The(hv) Lord(iv) (:?) said(jvr) unto(jv) <b>my</b>(i) Lord(jv) <sp>*</sp>(:) Sit(jrv) thou() on() my() right() hand,() until() I() make() thine() en(jv)e(jv)<b>mies</b>(i) thy(h) foot(irv)stool.(gf)
}

\begin{multicols}{2}
	The Lord shall send the rod of thy power out of \textbf{Zi}on * be thou ruler, even in the midst a\textbf{mong} thine enemies.
	
	In the day of thy power shall the people offer thee free-will offerings with a holy \textbf{wor}ship * the dew of thy birth is of the womb \textbf{of} the morning.
	
	The Lord sware, and will not \textbf{re}pent: * Thou art a priest forever after the order \textbf{of} Melchizedek.
	
	The Lord upon thy \textbf{right} hand * shall wound even kings in the \textbf{day} of his wrath.
	
	He shall judge among the heathen; he shall fill the places with the dead \textbf{bod}ies * and smite in sunder the heads over \textbf{di}verse countries.
	
	He shall drink of the brook in \textbf{the} way * therefore shall \textbf{he} lift up his head.
	
	Glory be to the Father, and to \textbf{the} Son * and to the \textbf{Ho}ly Spirit.
	
	As it was in the beginning is now and ever \textbf{shall} be: * world with\textbf{out} end. Amen.
\end{multicols}

\skiplines{1}\repant{ant/sede_a_dextris.gabc}



\veant{2}{iv. vi}
\skiplines{1}\gregorioscore{ant/fidelia_omnia.gabc}

\greannotation{\veformat{Ps. 110}}
\greannotation{\veformat{iv. vi}}
\skiplines{1}\gabcsnippet{
	(f3) I(jv) will(i) (:?) give(jrv) thanks(jv) unto() the() Lord(jv) <b>with</b>(i) my(jv) whole(kv) heart(jr) <sp>*</sp>(:) se(jr)cretly(j) among() the() faithful() and() in() the() congrega(j)<b>tion.</b>(i)
}

\begin{multicols}{2}
	The works \textbf{of} the Lord are great * sought out of all them that \textbf{have} pleasure therein.
	
	His work is worthy to be praised, and \textbf{had} in honor * and his righteousness en\textbf{dur}eth forever.
	
	The merciful and gracious Lord hath so done his \textbf{mar}velous works * that they ought to be \textbf{had} in remembrance.
	
	He hath given meat unto \textbf{them} who fear him * he shall ever be mind\textbf{ful} of his covenant.
	
	He hath shewed his people the \textbf{pow}er of his works * that he may give them the herit\textbf{age} of the heathen.
	
	The works of his hands are veri\textbf{ty} and judgment * all his \textbf{com}mandments are true.
	
	They stand fast fore\textbf{ver} and ever * and are done \textbf{in} truth and equity.
	
	He sent redemption un\textbf{to} his people * he hath commanded his covenant forever; holy \textbf{and} reverend is his Name.
	
	The fear of the Lord is the begin\textbf{ning} of wisdom * a good understanding have all they who do thereafter; his praise en\textbf{dur}eth forever.
	
	Glory be to the Fath\textbf{er}, and to the Son * and to \textbf{the} Holy Spirit.
	
	As it was in the beginning, is now and \textbf{ev}er shall be * world \textbf{with}out end. Amen.
\end{multicols}

\skiplines{1}\repant{ant/fidelia_omnia.gabc}\clearpage




\veant{3}{iv. vi}
\skiplines{1}\gregorioscore{ant/in_mandatis_ejus.gabc}

\greannotation{\veformat{Ps. 111}}
\greannotation{\veformat{iv. vi}}
\skiplines{1}\gabcsnippet{
	(f3) Bles(jv)ed(i) (:?) is(jrv) the(jv) man() that(jv) <b>fear</b>(i)eth(jv) the(kv) Lord(jr) <sp>*</sp>(:) he(jr) hath() great() delight() in() his() com(j)mand(j)<b>ments.</b>(i)
}

\begin{multicols}{2}
	His seed shall be migh\textbf{ty} upon earth * the generation of the faithful shall be bles\textbf{sed}.
	
	Riches and plenteousness \textbf{shall} be in his house * and his righteousness endureth fore\textbf{ver}.
	
	Unto the godly there ariseth up light \textbf{in} the darkness * he is merciful, loving, and righ\textbf{teous}.
	
	A good man is merci\textbf{ful}, and lendeth * and will guide his words with discre\textbf{tion}.
	
	For he \textbf{shall} never be moved * and the righteous shall be had in everlasting remem\textbf{brance}.
	
	He will not be afraid of any \textbf{ev}il tidings * for his heart standeth fast, and believeth in the \textbf{Lord}.
	
	His heart is esta\textbf{blished}, and will not shrink * until he sees his desire upon his enem\textbf{ies}.
	
	He hath dispersed abroad, and \textbf{giv}en to the poor * and his rightouesness remaineth forever; his horn shall be exalted with hon\textbf{or}.
	
	The ungodly shall see it, and \textbf{it} shall grieve him * he shall gnash with his teeth, and consume away; the desire of the ungodly shall per\textbf{ish}.
	
	Glory be to the Fath\textbf{er}, and to the Son * and to the Holy Spir\textbf{it}.
	
	As it was in the beginning, is now and \textbf{ev}er shall be * world without end. A\textbf{men}.
\end{multicols}

\skiplines{1}\repant{ant/in_mandatis_ejus.gabc}




\veant{4}{vii. ii}
\skiplines{1}\gregorioscore{ant/sit_nomen_domini.gabc}

\greannotation{\veformat{Ps. 112}}
\greannotation{\veformat{vii. ii.}}
\skiplines{1}\gabcsnippet{
	(c3) Praise(gv) ye(hv) (:?) the(irv) <b>Lord,</b>(kv) ye(jr) <b>ser</b>(i)vants(jv) <sp>*</sp>(:) O(ir) praise() the(i) <b>Name</b>(jv ir) <b>of</b>(h) the(hr) Lord.(gh)
}

\begin{multicols}{2}
	Blessed be the \textbf{Name} of \textbf{the} Lord * from this time \textbf{forth} for\textbf{ev}ermore.
	
	The \textbf{Lord's} Name \textbf{is} praised * from the rising up of the sun unto the \textbf{go}ing down \textbf{of} the same.
	
	The Lord is high a\textbf{bove} all \textbf{ hea}then * and his glory a\textbf{bove} the \textbf{heav}ens.
	
	Who is like unto the Lord our God, that hath his \textbf{dwel}ling \textbf{so} high? * and yet humbleth himself to behold the things that are in \textbf{hea}ven \textbf{and} earth?
	
	He taketh up the simple \textbf{out} of \textbf{the} dust * and lifteth the poor \textbf{out} \textbf{of} the mire.
	
	That he may sit him \textbf{with} the \textbf{prin}ces * even with the princes \textbf{of} his \textbf{peo}ple.
	
	He maketh the barren \textbf{wo}man to \textbf{keep} house * and to be the joyful \textbf{mo}ther of \textbf{child}ren.
	
	Glory be to the \textbf{Fa}ther, and to \textbf{the} Son * and to the \textbf{Ho}ly \textbf{Spir}it.
	
	As it was in the beginning is now, \textbf{and} ever \textbf{shall} be * world \textbf{with}out end. \textbf{A}men.
\end{multicols}

\skiplines{1}\repant{ant/sit_nomen_domini.gabc}




\veant{5}{T. Per}
\skiplines{1}\gregorioscore{ant/nos_qui_vivimus.gabc}

\greannotation{\veformat{Ps. 113}}
\greannotation{\veformat{T. Per.}}
\skiplines{1}\gabcsnippet{
	(c4)
}

\red{\Huge FIX THIS}

\skiplines{1}\repant{ant/nos_qui_vivimus.gabc}
			
		\vechapter{Monday at Vespers}
		
		\vechapter{Tuesday at Vespers}
		
		\vechapter{Wednesday at Vespers}
		
		\vechapter{Thursday at Vespers}
		
		\vechapter{Friday at Vespers}
			
		\vechapter{Saturday at Vespers}
		
			\veant{1}{vi}
\gregorioscore{ant/benedictus_dominus.gabc}

\greannotation{\veformat{Ps. 143}}
\greannotation{\veformat{vi. i.}}
\skiplines{1}\gabcsnippet{
	(f3) Bles(hv)ed(iv) (:?) be(jrv) the() Lord(jv) <b>my</b>(i) strength(jv) <sp>*</sp>(:) who(jrv) teacheth() my() hands() to() war,() and() my(jv) <b>fin</b>(h)gers(ij) to(i) (hr) fight.(h)
}

\begin{multicols}{2}
	My hope and my fortress, my castle and deliverer, my defender in whom \textbf{I} trust * who subdueth my people \textbf{that} is under me.
	
	Lord, what is man that thou hast such respect un\textbf{to} him? * or the son of man, that thou \textbf{so} regardest him?
	
	Man is like a thing \textbf{of} nought * his time passeth away \textbf{like} a shadow.
	
	Bow thy heavens, O Lord, and \textbf{come} down * touch the moun\textbf{tains}, and they shall smoke.
	
	Cast forth thy lightning, and \textbf{tear} them * shoot out thine arrows, \textbf{and} consume them.
	
	Send down thine hand from \textbf{a}bove * deliver me, and take me out of the great waters, from the hand \textbf{of} strange children.
	
	Whose mouth talketh of van\textbf{i}ty * and their right hand is a right \textbf{hand} of wickedness.
	
	I will sing a new song unto thee, \textbf{O} God * and sing praises unto thee upon \textbf{a} ten-stringed lute.
	
	Thou hast given victory un\textbf{to} kings * and hast delivered David thy servant from the \textbf{pe}ril of the sword.
	
	Save me, and deliver me from the hand of strange \textbf{child}ren * whose mouth talketh of vanity, and their right hand is a right hand \textbf{of} iniquity.
	
	That our sons may grow up as the \textbf{young} plants * and that our daughters may be as the polished corners \textbf{of} the temple.
	
	That our garners may be full and plenteous with all manner \textbf{of} store * that our sheep may bring forth thousands and ten \textbf{thou}sands in our streets.
	
	That our oxen may be strong to labor, that there be no \textbf{de}cay * no leading into captivity, and no com\textbf{plain}ing in our streets.
	
	Happy are the people that are in such \textbf{a} case * yea, blessed are the people who have \textbf{the} Lord for their God.
	
	Glory be to the Father, and to \textbf{the} Son * and to \textbf{the} Holy Spirit.
	
	As it was in the beginning is now, and ever \textbf{shall} be * world with\textbf{out} end. Amen.
\end{multicols}

\skiplines{1}\repant{ant/benedictus_dominus.gabc}  \clearpage

\veant{2}{viii. iii}
\gregorioscore{ant/in_eternum.gabc}

\greannotation{\veformat{Ps. 144}}
\greannotation{\veformat{viii. iii.}}
\skiplines{1}\gabcsnippet{
	(c4) I(g) will(hv) (:?) mag(jrv)nify() thee,() O(jv) <b>God</b>,(kv) my(jr) King() <sp>*</sp>(:) and(jr) I() will() praise() thy() Name() for(j)ev(j)<b>er</b>(i) and(h) ev(jv)er.(jr)
}

\begin{multicols}{2}
	Every day will I give thanks \textbf{un}to thee * and praise thy Name forev\textbf{er} and ever.
	
	Great is the Lord, and marvelous, worthy \textbf{to} be praised * there is no end \textbf{of} his greatness.
	
	One generation shall praise thy works unto a\textbf{noth}er * and de\textbf{clare} thy power.
	
	As for me, I will be talking of thy \textbf{wor}ship * thy glory, thy \textbf{praise}, and wondrous works.
	
	So that men shall speak of the might of thy marve\textbf{lous} acts * and I will also tell \textbf{of} thy greatness.
	
	The memorial of thine abundant kindness \textbf{shall} be shewed * and men shall sing \textbf{of} thy righteousness.
	
	The Lord is gracious and \textbf{mer}ciful * long-suffering, and \textbf{of} great goodness.
	
	The Lord is loving unto \textbf{ev}ery man * and his mercy is \textbf{ov}er all his works.
	
	All thy works praise \textbf{thee}, O Lord * and thy saints \textbf{give} thanks unto thee.
	
	They shew the glory of thy \textbf{king}dom * and talk \textbf{of} thy power.
	
	That thy power, thy glory, and mightiness of thy \textbf{king}dom * might \textbf{be} known unto men.
	
	Thy kingdom is an everlasting \textbf{king}dom * and thy dominion endureth through\textbf{out} all ages.
	
	The Lord upholdeth all \textbf{such} as fall * and lifteth up all \textbf{those} that are down.
	
	The eyes of all wait upon \textbf{thee}, O Lord * and thou givest them their meat \textbf{in} due season.
	
	Thou openest \textbf{thine} hand * and fillest all things liv\textbf{ing} with plenteousness.
	
	The Lord is righteous in \textbf{all} his ways * and ho\textbf{ly} in all his works.
	
	The Lord is nigh unto all them that call up\textbf{on} him * yea, all such as call up\textbf{on} him faithfully.
	
	He will fulfill the desire of them that \textbf{fear} him * he also will hear their cry, \textbf{and} will help them.
	
	The Lord preserveth all that \textbf{love} him * but scattereth abroad all \textbf{the} ungodly.
	
	My mouth shall speak the praise \textbf{of} the Lord * and let all flesh give thanks unto his holy Name forev\textbf{er} and ever.
	
	Glory be to the Father, and \textbf{to} the Son * and to the \textbf{Ho}ly Spirit.
	
	As it was in the beginning is now, and ever \textbf{shall} be * world with\textbf{out} end. Amen.
\end{multicols}

\skiplines{1}\repant{ant/in_eternum.gabc} \clearpage

\veant{3}{iv. i}
\gregorioscore{ant/laudabo.gabc}

\greannotation{\veformat{Ps. 145}}
\greannotation{\veformat{iv. i.}}
\skiplines{1}\gabcsnippet{
	(c4) Praise(hv) the(g) (:?) Lord,(hrv) O() my() soul;() while() I() live,(hv) <b>will</b>(g) I(hv) praise(iv) the(hr) Lord <sp>*</sp>(:) yea,(hr) as() long() as() I() have() any() being,() I() will() sing() prai(h)<b>ses</b>(g) un(hv)to(ih) (gr) <b>my</b>(gvFE) God.(e)
}

\begin{multicols}{2}
	O put not your trust in princes, nor in \textbf{an}y son of man * for there \textbf{is} no help in them.
	
	For when the breath of man goeth forth he shall turn \textbf{a}gain to his earth * and then \textbf{all} his thoughts perish.
	
	Blessed is he that hath the God of \textbf{Ja}cob for his help * and whose hope \textbf{is} in the Lord his God.
	
	Who made heaven and earth, the sea, and \textbf{all} that therein is * who keepeth his \textbf{pro}mise forever.
	
	Who helpeth them to \textbf{right} that suffer wrong * who \textbf{feed}eth the hungry.
	
	The Lord looseth men \textbf{out} of prison * the Lord \textbf{giv}eth sight to the blind.
	
	The Lord helpeth them \textbf{that} are fallen * the Lord car\textbf{eth} for the righteous.
	
	The Lord careth for the strangers; he defendeth the father\textbf{less} and widow * as for the way of the ungodly, he turn\textbf{eth} it upside down.
	
	The Lord thy God, O Zion, shall be \textbf{King} forevermore * and throughout \textbf{all} generations.
	
	Glory be to the Fath\textbf{er}, and to the Son * and to \textbf{the} Holy \textbf{Spir}it.
	
	As it was in the beginning is now, and \textbf{ev}er shall be * world \textbf{with}out end. \textbf{A}men.
\end{multicols}

\skiplines{0.5}\repant{ant/laudabo.gabc}

\veant{4}{viii. iii}
\skiplines{1}\gregorioscore{ant/deo_nostro.gabc}

\greannotation{\veformat{Ps. 146}}
\greannotation{\veformat{viii. iii.}}
\skiplines{1}\gabcsnippet{
	(c3) O(e) praise(fv) (:?) the(hrv) Lord,() for() it() is() a() good() thing() to() sing() praises() un(hv)<b>to</b>(iv) our(hr) God(h) <sp>*</sp>(:) yea,(hr) a() joyful() and() pleasant() thing() it() is(h) <b>to</b>(g) be(f) thank(hrv)ful.(hv)
}

\begin{multicols}{2}
	The Lord doth build up Je\textbf{ru}salem * and gather together the out\textbf{casts} of Israel.
	
	He healeth those that are broken \textbf{in} heart * and giveth medicine to \textbf{heal} their sickness.
	
	He telleth the number \textbf{of} the stars * and calleth \textbf{them} by all their names.
	
	Great is our Lord, and great is his \textbf{pow}er * yea, and his wis\textbf{dom} is infinite.
	
	The Lord setteth \textbf{up} the meek * and bringeth down to the ground \textbf{the} ungodly.
	
	O sing unto the Lord with thanks\textbf{giv}ing * sing praises upon the harp un\textbf{to} our God.
	
	Who covereth the heavens with clouds, and prepareth rain \textbf{for} the earth * and maketh the grass to grow upon the mountains, and herb \textbf{for} the use of men.
	
	Who giveth fodder unto the \textbf{cat}tle * and feedeth the young ravens that \textbf{call} upon him.
	
	He hath no pleasure in the strength \textbf{of} a horse * neither delighteth he in \textbf{an}y man's legs.
	
	But the Lord's delight is in them that \textbf{fear} him * and put their trust \textbf{in} his mercy.
	
	Glory be to the Father, and \textbf{to} the Son * and to the \textbf{Ho}ly Spirit.
	
	As it was in the beginning is now, and ever \textbf{shall} be * world with\textbf{out} end. Amen.
\end{multicols}

\skiplines{1}\repant{ant/deo_nostro.gabc} \clearpage

\veant{5}{iv. vii}
\gregorioscore{ant/lauda_hierusalem.gabc}

\greannotation{\veformat{Ps. 147}}
\greannotation{\veformat{iv. vii.}}
\skiplines{1}\gabcsnippet{
	(c4) Praise(hv) the(g) (:?) Lord,(hrv) <b>O</b>(g) Je(hv)ru(iv)sa(hr)lem(h) <sp>*</sp>(:) praise(hr) thy() God,() O() <b>Zi</b>(g)on.(hv)
}

\begin{multicols}{2}
	For he hath made fast \textbf{the} bars of thy gates * and hath blest thy children with\textbf{in} thee.
	
	He maketh peace \textbf{in} thy borders * and filleth thee with the flour \textbf{of} wheat.
	
	He sendeth forth his command\textbf{ment} upon earth * and his word runneth very \textbf{swift}ly.
	
	He \textbf{giv}eth snow like wool * and scattereth the hoarfrost like \textbf{ash}es.
	
	He casteth forth his \textbf{ice} like morsels * who is able to abide \textbf{his} frost?
	
	He sendeth out his \textbf{word}, and melteth them * he bloweth his wind, and the wat\textbf{ers} flow.
	
	He sheweth his word \textbf{un}to Jacob * his statutes and ordinances unto Is\textbf{ra}el.
	
	He hath not dealt so with \textbf{an}y nation * neither have the heathen knowledge of \textbf{his} laws.
	
	Glory be to the Fath\textbf{er}, and to the Son * and to the Holy \textbf{Spir}it.
	
	As it was in the beginning, is now and \textbf{ev}er shall be * world without end. \textbf{A}men.
\end{multicols}

\repant{ant/lauda_hierusalem_2.gabc}

\instruct{Turn to the \whiteribbon at the chapter rite, pp. \pageref{vespers:chapter}.}

\skiplines{1}\gregorioscore{kyrie/xvii.gabc}
			
		\vechapter{Advent at Compline}
		
			\greannotation{\veformat{Ant.}}
\greannotation{\veformat{viii. i.}}
\gregorioscore{ant/miserere_michi_domine.gabc}

\greannotation{\veformat{Ps. 4}}
\greannotation{\veformat{viii. i.}}
\skiplines{1}\gabcsnippet{
	(c4) Hear(gv) me(hv) (:?) when(jvr) I() call,() O() God() of() my(jv) <b>right</b>(kv)eous(jr)ness(j) <sp>*</sp>(:) thou(jr) hast() set() me() at() liberty() when() I() was() in() trouble;() have() mercy() upon() me,() and() hear(j)<b>ken</b>(i) un(jv)to(h) my(gr) prayer.(g)
}

\begin{multicols}{2}
	O ye sons of men, how long will ye blaspheme mine \textbf{hon}or? * and have such pleasure in vanity, and seek \textbf{af}ter lying?
	
	Know this also, that the Lord hath chosen to himself the man that is \textbf{god}ly * when I call upon the Lord, \textbf{he} will hear me.
	
	Stand in awe, and \textbf{sin} not * commune with your own heart, and in your \textbf{cham}ber, and be still.
	
	Offer the sacrifice of \textbf{right}eousness * and put your \textbf{trust} in the Lord.
	
	There be many \textbf{that} say * Who will \textbf{shew} us any good?
	
	Lord, \textbf{lift} thou up * the light of thy counte\textbf{nance} upon us.
	
	Thou hast put gladness \textbf{in} my heart * since the time that their corn, and wine, and \textbf{o}il increased.
	
	I will lay me down in peace, and \textbf{take} my rest * for it is thou, Lord, only, who makest me \textbf{dwell} in safety.
	
	Glory be to the Father, and \textbf{to} the Son * and to the \textbf{Ho}ly Spirit.
	
	As it was in the beginning is now, and ever \textbf{shall} be * world with\textbf{out} end. Amen.
\end{multicols}



\greannotation{\veformat{Ps. 90}}
\greannotation{\veformat{viii. i.}}
\skiplines{1}\gabcsnippet{
	(c4) Who(gv)so(hv) (:?) dwell(jrv)eth() under() the() defense() of() the(jv) <b>most</b>(kv jr) High(j) <sp>*</sp>(:) shall(jr) abide() under() the() shadow() of(j) <b>the</b>(i) Al(jv)migh(h gr)ty.(g)
}

\begin{multicols}{2}
	I will say unto the Lord, Thou art my hope, and my \textbf{strong}hold * my God, \textbf{in} him will I trust.
	
	For he shall deliver thee from the snare of the \textbf{hunt}er * and from the \textbf{noi}some pestilence.
	
	He shall defend thee under his wings, and thou shalt be safe under his \textbf{fea}thers * his faithfulness and truth shall be thy \textbf{shield} and buckler.
	
	Thou shalt not be afraid for any terror \textbf{by} night * nor for the arrow that \textbf{fli}eth by day.
	
	For the pestilence that walketh in \textbf{dark}ness * nor for the sickness that destroyeth \textbf{in} the noonday.
	
	A thousand shall fall beside thee, and ten thousand at thy \textbf{right} hand * but it shall \textbf{not} come nigh thee.
	
	Yea, with thine eyes shalt thou \textbf{be}hold * and see the reward of \textbf{the} ungodly.
	
	For thou, Lord, \textbf{art} my hope * thou hast set thine house of \textbf{de}fense very high.
	
	There shall no evil happen \textbf{un}to thee * neither shall any plague come \textbf{nigh} thy dwelling.
	
	For he shall give his angels charge \textbf{ov}er thee * to keep \textbf{thee} in all thy ways.
	
	They shall bear thee \textbf{in} their hands * that thou hurt not thy \textbf{foot} against a stone.
	
	Thou shalt go upon the lion and \textbf{ad}der * the young lion and the dragon shalt thou tread \textbf{un}der thy feet.
	
	Because he hath set his love upon me, therefore will I de\textbf{liv}er him * I will set him up, because \textbf{he} hath known my Name.
	
	He shall call upon me, and I will \textbf{hear} him * yea, I am with him in trouble; I will deliver him, and bring \textbf{him} to honor.
	
	With long life will I satis\textbf{fy} him * and shew him \textbf{my} salvation.
	
	Glory be to the Father, and \textbf{to} the Son * and to the \textbf{Ho}ly Spirit.
	
	As it was in the beginning is now, and ever \textbf{shall} be * world with\textbf{out} end. Amen.
\end{multicols}



\clearpage\greannotation{\veformat{Ps. 133}}
\greannotation{\veformat{viii. i.}}
\skiplines{1}\gabcsnippet{
	(c4) Be(gv)hold(hv) (:?) now,(jvr) <b>praise</b>(kv) the(jr) Lord(j) <sp>*</sp>(:) all(jr) ye(j) <b>ser</b>(i)vants(jv) of(h) the(gr) Lord.(g)
}

\begin{multicols}{2}
	Ye that by night stand in the house \textbf{of} the Lord * even in the courts of the \textbf{house} of our God.
	
	Lift up your hands in the sanctu\textbf{ar}y * \textbf{and} praise the Lord.
	
	The Lord who made heaven \textbf{and} earth * give thee blessing \textbf{out} of Zion.
	
	Glory be to the Father, and \textbf{to} the Son * and to the \textbf{Ho}ly Spirit.
	
	As it was in the beginning is now, and ever \textbf{shall} be * world with\textbf{out} end. Amen.
\end{multicols}

\skiplines{1}\repant{ant/miserere_michi_domine.gabc}
			
		\vechapter{Christmastide at Vespers}
	
	\vepart{Propers}
	
		\vechapter{First Sunday in Advent}
			
			\skiplines{-1}\vesectionp{At First Vespers}
				
				\skiplines{-1}\resp{Ecce dies veniunt}{viii}
				\gregorioscore{resp/ecce_dies_veniunt.gabc}
					
				\instruct{Turn to the \blueribbon and sing the hymn and versicle.}
				
				\skiplines{1}
				\veant{M}{i}
				\gregorioscore{ant/ecce_nomen_domini.gabc}
				
				\greannotation{\veformat{Cant.}}
				\greannotation{\veformat{i.}}
				\skiplines{1}\gabcsnippet{
					(cb4) My(f) soul(gh :? hr) <b>doth</b>(hg) mag(gi)ni(hr)fy(h) <b>the</b>(hg) Lord(gh) <sp>*</sp>(:) and(hr) my() <b>spir</b>(hi)it(hr) hath() rejoiced() in(h) <b>God</b>(g) my(f) Sav(gh)ior.(g)
				}
				
				\begin{multicols}{2}
	For he hath re\textbf{gard}ed * the lowliness of \textbf{his} handmaiden.
	
	For behold, from \textbf{hence}forth * all  generations shall \textbf{call} me blessed.
	
	For he that is mighty hath magni\textbf{fied} me * and \textbf{ho}ly is his Name.
	
	And his mercy is on them who \textbf{fear} him * throughout all \textbf{gen}erations.
	
	He hath shewed strength with \textbf{his} arm * he hath scattered the proud in the imagi\textbf{na}tion of their hearts.
	
	He hath put down the mighty from \textbf{their} seat * and hath exalted the \textbf{hum}ble and meek.
	
	He hath filled the hungry with \textbf{good} things * and the rich he hath sent \textbf{emp}ty away.
	
	He, remembering his mercy, hath holpen his servant Is\textbf{ra}el * as he promised to our forefathers, to Abraham and his \textbf{seed} forever.
	
	Glory be to the Father, and to \textbf{the} Son * and to the \textbf{Ho}ly Spirit.
	
	As it was in the beginning, is now, and ever \textbf{shall} be * world with\textbf{out} end. Amen.
\end{multicols}
				
				\repant{ant/ecce_nomen_domini.gabc}
			
				\instruct{Turn to the \whiteribbon at the conclusion rite, pp. \pageref{vespers:conclusion}.}
			
			\vesection{At Second Vespers}
			
				\instruct{Turn to the \blueribbon and sing the responsory, hymn, and versicle.}
				
				\skiplines{1}
				\veant{M}{viii}
				\gregorioscore{ant/ne_timeas_maria.gabc}
				
				\greannotation{\veformat{Cant.}}
				\greannotation{\veformat{viii.}}
				\skiplines{1}\gabcsnippet{
					(c4) My(gh) soul(gj :? jr) <b>doth</b>(ji) mag(jk)ni(kr)<b>fy</b>(jk) the(jr) Lord(j) <sp>*</sp>(:) and(jr) my(j) <b>spir</b>(jk)it(jr) hath() rejoiced() in(jr) <b>God</b>(i) my(j) Sav(h)ior.(g)
				}
			
				\begin{multicols}{2}
	For he hath re\textbf{gard}ed * the lowliness of \textbf{his} handmaiden.
	
	For behold, from \textbf{hence}forth * all  generations shall \textbf{call} me blessed.
	
	For he that is mighty hath magni\textbf{fied} me * and \textbf{ho}ly is his Name.
	
	And his mercy is on them who \textbf{fear} him * throughout all \textbf{gen}erations.
	
	He hath shewed strength \textbf{with} his arm * he hath scattered the proud in the imagi\textbf{na}tion of their hearts.
	
	He hath put down the mighty \textbf{from} their seat * and hath exalted the \textbf{hum}ble and meek.
	
	He hath filled the hungry with \textbf{good} things * and the rich he hath sent \textbf{emp}ty away.
	
	He, remembering his mercy, hath holpen his servant\textbf{Is}rael * as he promised to our forefathers, to Abraham and his \textbf{seed} forever.
	
	Glory be to the Father, and to \textbf{the} Son * and to the \textbf{Ho}ly Spirit.
	
	As it was in the beginning, is now, and ever \textbf{shall} be * world with\textbf{out} end. Amen.
\end{multicols}
			
				\repant{ant/ne_timeas_maria.gabc}
				
				\instruct{Turn to the \whiteribbon at the conclusion rite, pp. \pageref{vespers:conclusion}.}
			
			\vesection{Monday at Vespers}
			
				\instruct{Turn to the \blueribbon and sing the responsory, hymn, and versicle.}
				
				\skiplines{1}
				\veant{M}{vii}
				\gregorioscore{ant/hierusalem_respice.gabc}
				
				\greannotation{\veformat{Cant.}}
				\greannotation{\veformat{vii.}}
				\skiplines{1}\gabcsnippet{
					(c4) My(fe) soul(fg) (:?) doth(gr) <b>mag</b>(gi)ni(hr)fy(h) the(g) Lord(h) <sp>*</sp>(:) and(gr) my() spirit() hath() rejoiced() in() <b>God</b>(h) my(g) Sav(f)ior.(ed)
				}
				
				\begin{multicols}{2}
	For he hath regarded * the lowliness of his handmaiden.
	
	For behold, from henceforth * all generations shall call me blessed.
	
	For he that is mighty hath magnified me * and holy is his Name.
	
	And his mercy is on them that fear him * throughout all generations.
	
	He hath shewed strength with his arm * he hath scattered the proud in the imagination of their hearts.
	
	He hath put down the mighty from their seat * and hath exalted the humble and meek.
	
	He hath filled the hungry with good things * and the rich he hath sent empty away.
	
	He, remembering his mercy, hath holpen his servant Israel * as he promised to our forefathers, to Abraham and his seed forever.
	
	Glory be to the Father, and to the Son * and to the Holy Spirit.
	
	As it was in the beginning, is now, and ever shall be * world without end. Amen.
\end{multicols}
				
				\repant{ant/hierusalem_respice.gabc}
				
				\instruct{Turn to the \whiteribbon at the conclusion rite, pp. \pageref{vespers:conclusion}.}
			
			\vesection{Tuesday at Vespers}
			
			\vesection{Wednesday at Vespers}
			
			\vesection{Thursday at Vespers}
			
			\vesection{Friday at Vespers}
			
		\vechapter{Second Sunday in Advent}
			\skiplines{-1}\vesectionp{At First Vespers}
			
			\vesection{At Second Vespers}
			
			\vesection{Monday at Vespers}
			
			\vesection{Tuesday at Vespers}
			
			\vesection{Wednesday at Vespers}
			
			\vesection{Thursday at Vespers}
			
			\vesection{Friday at Vespers}
			
		\vechapter{Third Sunday in Advent}
			\skiplines{-1}\vesectionp{At First Vespers}
			
			\vesection{At Second Vespers}
			
			\vesection{Monday at Vespers}
			
			\vesection{Tuesday at Vespers}
			
			\vesection{Wednesday at Vespers}
			
			\vesection{Thursday at Vespers}
			
			\vesection{Friday at Vespers}
			
		\vechapter{Fourth Sunday in Advent}
			\skiplines{-1}\vesectionp{At First Vespers}
			
			\vesection{At Second Vespers}
			
			\vesection{Monday at Vespers}
			
			\vesection{Tuesday at Vespers}
			
			\vesection{Wednesday at Vespers}
			
			\vesection{Thursday at Vespers}
			
			\vesection{Friday at Vespers}
			
		\vechapter{Nativity of the Lord}
		
			\skiplines{-1}\vesectionp{At First Vespers}
			
			\vesection{Eve at Compline}
			
			\vesection{At Second Vespers}
			
			\vesection{Day at Compline}
			
		\vechapter{Saint Stephen, Protomartyr}
		
		\vechapter{Saint John, Evangelist and Apostle}
		
		\vechapter{Holy Innocents}
		
		\vechapter{Within the Octave}
			\skiplines{-1}\vesectionp{Sixth Day at Vespers}
			
			\vesection{Seventh Day at Vespers}
			
		\vechapter{Circumcision of the Lord}
		
		\vechapter{Epiphany of the Lord}
			\skiplines{-1}\vesectionp{At First Vespers}
			
			\vesection{At Compline}
			
			\vesection{At Second Vespers}
	
	\vepart{Commons}
	
		\vechapter{Advent 1 at Vespers}
		
			\skiplines{-1}\resp{Tu exurgens Domine}{iv}
			\gregorioscore{resp/tu_exurgens_domine.gabc}
			
			\greannotation{\veformat{Hymn.}}
\greannotation{\veformat{iv.}}

\vecomment{Conditor alme siderum}

\gregorioscore{hymn/conditor_alme_siderum.gabc}

\begin{multicols}{1}
	Thou, grieving that the ancient curse,\\
	Should doom to death the universe,\\
	Hast found the med'cine, full of grace,\\
	To save and heal a ruined race.
	
	Thou camst, the Bridegroom of the Bride,\\
	As drew the world to eveningtide;\\
	Proceeding from a virgin shrine,\\
	The spotless Victim all divine.
	
	At whose dread name, majestic mow,\\
	All knees must bend, all hearts must bow;\\
	And things celestial thee shall own,\\
	And things terrestial, Lord alone.
	
	O thou whose coming is with dread,\\
	To judge and doom the quick and dead,\\
	Preserve us, while we dwell below,\\
	From every insult of the foe.
	
	To God the Father, God the Son,\\
	And God the Spirit, Three in one,\\
	Laud, honor, might, and glory be,\\
	From age to age eternally.
	
	\skiplines{1}\gresetinitiallines{0}\gabcsnippet{(f3) A(fg/hg)men.(gfg)}\gresetinitiallines{1}
\end{multicols}
			
			\gresetinitiallines{0}\gabcsnippet{
				<sp>v</sp> Drop(hv) down,(hv) ye(hv) hea(hv)vens,(hv) from(hv) a(hv)bove.(ghGFEfhGF) <sp>r</sp>(::) And(hv) let(hv) the(hv) skies(hv) pour(hv) down(hv) right(hv)eous(f)ness:(gv) (;) let(hv) the(hv) earth(hv) o(hv)pen,(hv) and(hv) let(hv) it(hv) bring(hv) forth(hv) sal(hv)va(f)tion.(f)
			}\gresetinitiallines{1}
				
			\instruct{The Magnificat is always proper on all days in Advent; turn to the \redribbon and sing as appointed there.}
			
		\vechapter{Advent 2 at Vespers}
		
			\skiplines{-1}\resp{Festina ne taraveris}{ii}
			\gregorioscore{resp/festina_ne_taraveris.gabc}
		
			\clearpage\greannotation{\veformat{Hymn.}}
\greannotation{\veformat{iv.}}

\vecomment{Conditor alme siderum}

\gregorioscore{hymn/conditor_alme_siderum.gabc}

\begin{multicols}{1}
	Thou, grieving that the ancient curse,\\
	Should doom to death the universe,\\
	Hast found the med'cine, full of grace,\\
	To save and heal a ruined race.
	
	Thou camst, the Bridegroom of the Bride,\\
	As drew the world to eveningtide;\\
	Proceeding from a virgin shrine,\\
	The spotless Victim all divine.
	
	At whose dread name, majestic mow,\\
	All knees must bend, all hearts must bow;\\
	And things celestial thee shall own,\\
	And things terrestial, Lord alone.
	
	O thou whose coming is with dread,\\
	To judge and doom the quick and dead,\\
	Preserve us, while we dwell below,\\
	From every insult of the foe.
	
	To God the Father, God the Son,\\
	And God the Spirit, Three in one,\\
	Laud, honor, might, and glory be,\\
	From age to age eternally.
	
	\skiplines{1}\gresetinitiallines{0}\gabcsnippet{(f3) A(fg/hg)men.(gfg)}\gresetinitiallines{1}
\end{multicols}
			
			\gresetinitiallines{0}\gabcsnippet{
				<sp>v</sp> Drop(hv) down,(hv) ye(hv) hea(hv)vens,(hv) from(hv) a(hv)bove.(ghGFEfhGF) <sp>r</sp>(::) And(hv) let(hv) the(hv) skies(hv) pour(hv) down(hv) right(hv)eous(f)ness:(gv) (;) let(hv) the(hv) earth(hv) o(hv)pen,(hv) and(hv) let(hv) it(hv) bring(hv) forth(hv) sal(hv)va(f)tion.(f)
			}\gresetinitiallines{1}
			
			\instruct{The Magnificat is always proper on all days in Advent; turn to the \redribbon and sing as appointed there.}
			
		\vechapter{Advent at Compline}
\end{document}