\documentclass[14pt,twoside]{extarticle}
\usepackage{moresize}
\usepackage{fontspec}
\usepackage{aeguill}
\usepackage{titlesec}
\usepackage[maxlevel=3]{csquotes}
\usepackage{lettrine}
\usepackage{xifthen}
\usepackage{yfonts}
\usepackage{parskip}
\usepackage[dvipsnames]{xcolor}
\usepackage{fancyhdr}
\usepackage{hanging}
\usepackage[autocompile,allowdeprecated=false]{gregoriotex}
\usepackage[twoside]{geometry}
\usepackage{multicol}
\usepackage{paracol}
\usepackage[latin,english]{babel}
\usepackage{enumitem}
\usepackage{tabto}
\usepackage{titlesec}
\usepackage[savepos]{zref}
%\usepackage{libertine}

\setlength{\columnseprule}{0.025pt}
\setlength{\columnsep}{1.5em}
\setlength{\topskip}{7pt}

\def\vehymn{-0.50}

\makeatletter
\patchcmd{\pcol@buildcolseprule}%
{\hrule\@height\@tempdima\@width\columnseprule}%
{\hrule\@height\dimexpr\@tempdima-0.5ex\@width\columnseprule\@depth-1pt}{}{}
\makeatother

\gredefsizedsymbol{GreCross}{greextra}{Cross}
\gredefsizedsymbol{GreDagger}{greextra}{Dagger}
\gredefsizedsymbol{GreCrossAlt}{greextra}{Cross.alt}
\gredefsizedsymbol{GreStarHeight}{greextra}{StarHeight}
\gredefsizedsymbol{greABar}{greextra}{ABar}
\gredefsizedsymbol{greVBar}{greextra}{VBar}
\gredefsizedsymbol{greRBar}{greextra}{RBar}

\gresetspecial{r}{\textcolor{red}{\Rbar{}.}}
\gresetspecial{v}{\textcolor{red}{\Vbar{}.}}
\gresetspecial{a}{\textcolor{red}{\Abar{}.}}
\gresetspecial{+}{\textcolor{red}{\GreCross{17}}}
\gresetspecial{1}{\textcolor{red}{\GreDagger{17}}}
\gresetspecial{2}{\textcolor{red}{\GreCrossAlt{17}}}
\gresetspecial{*}{\textcolor{red}{\GreStarHeight{17}}}
%%% for spacing in chant %%%
\gresetspecial{ }{\textcolor{white}{a}}
\gresetspecial{  }{\textcolor{white}{be}}
\gresetspecial{   }{\textcolor{white}{God}}
\gresetspecial{    }{\textcolor{white}{text}}

\grechangestaffsize{30}
%\grechangedim{abovelinestextheight}{0.25cm}{fixed}
%\grechangedim{spacebeneathtext}{0.5cm}{fixed}
\grechangedim{annotationseparation}{0.25cm}{fixed}

\newcommand{\skiplines}[1]{\vspace*{#1\baselineskip}}

\newcommand{\veformat}[1]{\textcolor{red}{\textbf{#1}}}
\newcommand{\veant}[2]{\greannotation{#1. \veformat{Ant.}}\greannotation{\veformat{#2.}}}
\newcommand{\instruct}[1]{\textcolor{red}{\textit{#1}}}

\makeatletter
\newcommand{\whiteribbon}[0]{\textcolor{black}{\textbf{white ribbon}} }%
\newcommand{\greenribbon}[0]{\textcolor{ForestGreen}{\textbf{green ribbon}} }%
\newcommand{\redribbon}[0]{\textcolor{red}{\textbf{red ribbon}} }%
\newcommand{\blueribbon}[0]{\textcolor{blue}{\textbf{blue ribbon}} }%
\makeatother

%%% HEADERS %%%
\newcommand{\vepart}[1]{%
	\cleardoublepage%
	\vspace*{\fill}%
	\begin{center}%
		{\Huge \selectfont \uppercase{ \textbf{#1}}}%
	\end{center}%
	\vspace*{\fill}%
}%
\makeatother

\newcommand{\vechapter}[1]{%
	\cleardoublepage%
	\begin{center}
		{\LARGE \selectfont \color{red} \uppercase{ \textbf{#1}}}\par%
	\end{center}
	\skiplines{1}
}%
\makeatother

\newcommand{\vesectionp}[1]{%
	\skiplines{1}
	\begin{center}
		{\Large \selectfont \uppercase{ \textbf{#1}}}%
	\end{center}
	\skiplines{1}
}%
\makeatother

\newcommand{\vesection}[1]{%
	\clearpage\skiplines{-2}\vesectionp{#1}%
}%
\makeatother

\makeatletter
\newcommand{\veheading}[1]{%
	\skiplines{0.5}%
	\begin{center}
		{\selectfont \textit{#1}}%
	\end{center}
	\skiplines{0.5}
}%
\makeatother

\makeatletter
\newcommand{\vecomment}[1]{%
	\hfill{\small\selectfont \textit{#1}.}%
}%
\makeatother

%%% OTHER FORMATS %%%

\makeatletter
\newcommand{\red}[1]{%
	{\color{red} #1}%
}%
\makeatother

\makeatletter
\newcommand{\chapter}[1]{%
	\gresetinitiallines{1}%
	\greannotation{\veformat{Chap.}}%
	\greannotation{\veformat{Dir.}}%
	\gabcsnippet{#1}%
}%
\makeatother

\makeatletter
\newcommand{\resp}[2]{%
	\vecomment{#1}
	
	\gresetinitiallines{1}%
	\greannotation{\veformat{Resp.}}%
	\greannotation{\veformat{#2.}}%
}%
\makeatother

\makeatletter
\newcommand{\tone}[2]{%
	\greannotation{\veformat{Ps.}}
	\greannotation{\veformat{#1.}}
	
	\gabcsnippet{#2}
}%
\makeatother

\makeatletter
\newcommand{\repant}[1]{%
	\gresetspecial{*}{}
	\nopagebreak[4]\skiplines{1}\nopagebreak[4]\gresetinitiallines{0}\gregorioscore{#1}\gresetinitiallines{1}
	\gresetspecial{*}{\textcolor{red}{\GreStarHeight{17}}}
}%
\makeatother



%%% FANCY %%%
\makeatletter  
\newcounter{score}
\newcounter{tabstop}[score]
\newcommand{\grealign}{%
	\@bsphack%
	\ifgre@boxing\else%
	\kern\gre@dimen@begindifference%
	\stepcounter{tabstop}%
	\expandafter\zsavepos{stop-\thescore-\thetabstop}%
	\kern-\gre@dimen@begindifference%
	\fi%
	\@esphack%
}

\newcommand{\setstops}{%
	\gdef\nstabbing@stops{%
		\hspace*{-\oddsidemargin}\hspace{-1in}%
		\hspace*{\zposx{stop-\thescore-1} sp}\=%
	}%
	\count@=\@ne
	\loop\ifnum\count@<\value{tabstop}%
	\begingroup\edef\x{\endgroup
		\noexpand\g@addto@macro\noexpand\nstabbing@stops{%
			\noexpand\hspace{-\noexpand\zposx{stop-\thescore-\the\count@} sp}%
			\noexpand\hspace{\noexpand\zposx{stop-\thescore-\the\numexpr\count@+1} sp}\noexpand\=%
		}%
	}\x
	\advance\count@\@ne
	\repeat
	\nstabbing@stops\kill
}
\makeatother

\newenvironment{nstabbing}
{\setlength{\topsep}{0pt}%
	\setlength{\partopsep}{0pt}%
	\tabbing%
	\setstops}
{\endtabbing\stepcounter{score}}

\begin{document}
	\newgeometry{margin=0pt}
	\begin{titlepage}
		\vspace*{\fill}
		
		\begin{center}
			
			\textbf{\Huge\color{red} VESPERALE}
			
			ACCORDING TO THE RITE OF THE
			
			\textbf{\LARGE HOLY ROMAN CHURCH}
			
			\vspace*{\fill}
			
			\textbf{\Large\color{red} VOLUME I}
			
			Seasons of Advent and Christmas\\Until the Epiphany of the Lord
			
		\end{center}
		
		\vspace*{\fill}
	\end{titlepage}
	
	\newgeometry{top=0.75in,bottom=0.75in,inner=0.75in,outer=0.75in,includefoot}
	
	\pagenumbering{arabic}
	\gresetinitiallines{0}
	
	\raggedbottom
	
	%\tableofcontents
	
	
	%%% COPY ABOVE %%%
	
	\vechapter{Liturgical Calendar}
	\skiplines{-1}\vesectionp{November}
		\begin{enumerate}[noitemsep, nolistsep]
			\setcounter{enumi}{26}
			\item {\textit{\small Earliest day for Advent.}}
			\item % 28
			\item \footnote{Outside of Advent, here is celebrated the Eve of Saint Andrew.}
			\item \red{Saint Andrew, Apostle.} Minor feast.
		\end{enumerate}
	\vesectionp{December}
		\begin{enumerate}[noitemsep, nolistsep]
			\item %\textit{\small Saint Nahum, Prophet.} % 1
			\item %\textit{\small Saint Habakkuk, Prophet.}  % 2
			\item \textit{\small Latest day for Advent.} %\textit{\small Saint Zephaniah, Prophet.}  % 3
			\item  % 4
			\item % 5
			\item Saint Nicholas, Bishop. \red{9 lect. duplex.}
			\item Saint Ambrose, Bishop and Doctor. \red{9 lect. triplex.}
			\item \red{Conception of the Blessed Virgin Mary.} Major feast.
			\item % 9
			\item % 10
			\item % 11
			\item  % 12
			\item Saint Lucy, Virgin and Martyr. \red{9 lect. duplex.}
			\item  % 14
			\item % 15
			\item %\textit{\small Saint Haggai, Prophet.}  % 16
			\item {\textit{\small O Wisdom.}} % 17
			\item  % 18
			\item  % 19
			\item  % 20
			\item %\textit{\small Saint Micah, Prophet.}  % 21
			\item  % 22
			\item  % 23
			\item \textcolor{purple}{\textbf{Eve of the Nativity.}}
			\item \textbf{\red{Nativity of the Lord.}} Solemnity. \textit{With octave.}
			\item \red{Saint Stephen, Protomartyr.} Major feast.
			\item \red{Saint John, Apostle and Evangelist.} Major feast.
			\item \red{Holy Innocents, Martyrs.} Major feast.
			\item \red{Saint Thomas Becket, Martyr.} Major feast.
			\item Of the Octave of the Nativity. % 30
			\item Of the Octave of the Nativity.
		\end{enumerate}
	\vesectionp{January}
		\begin{enumerate}[noitemsep, nolistsep]
			\item \red{Circumcision of the Lord.} Principle feast.
			\item Octave of Saint Stephen. \red{9 lect. triplex.}
			\item Octave of Saint John. \red{9 lect. triplex.}
			\item Octave of the Innocents. \red{9 lect. triplex.}
			\item \textcolor{purple}{\textbf{Eve of the Epiphany.}}
			\item \red{\textbf{Epiphany of the Lord.}} Solemnity.\footnote{On the following days, until the Sunday after Epiphany, its office is continued.}
		\end{enumerate}
	
	
	\gresetinitiallines{1}
	
	\vepart{Ordinary}
	
		\vechapter{The Order of Vespers}
			\skiplines{-1}\vesectionp{Introduction}\label{vespers:introduction}
			
				\instruct{Before the service begins, kneel and make these prayers quietly.}
				
				Our Father, who art in heaven, hallowed be thy name. Thy kingdom come. Thy will be done on earth as it is in heaven. Give us this day our daily bread and forgive us our trespasses as we forgive those who trespass against us. And lead us not into temptation but deliver us from evil.
					
				Hail Mary, full of grace, the Lord is with thee. Blessed art thou among women and blessed is the fruit of thy womb, Jesus. Holy Mary, Mother of God, pray for us sinners now and at the hour of our death. Amen.
				
				\instruct{Stand as the minister begins from his place:}
				
				\gresetinitiallines{0}
				\skiplines{0.5}\gabcsnippet{(c3) <sp>v</sp> O(fv) God,(hv) make(hv) speed(iv) to(h) save(g) me.(hv) <sp>r</sp>(::) O(hv) Lord,(hv) make(hv) haste(hv) to(hv) help(g) me.(hv)}
				
				\instruct{The congregation continues together, turning toward the altar, making a profound bow, and crossing themselves at the invocation of the Holy Trinity here and whenever else:}
				
				\skiplines{0.5}\gabcsnippet{(c3) Glor(h)y(h) be(h) to(h) the(h) Fa(h)ther,(h) and(h) to(h) the(h) Son,(h) (,) and(h) to(h) the(h) Ho(h)ly(h) Spir(g)it.(hv) (:) As(h) it(h) was(h) in(h) the(h) be(h)gin(h)ning,(h) is(h) now(h) and(h) ev(h)er(h) shall(g) be:(hv) (,) world(hv) with(hv)out(hv) end.(hv) A(g)men.(hv) (::) A(hv)le(hi)lu(h)ia.(hg)}
				
			\vesectionp{Psalmody}\label{vespers:psalmody}
			
				\instruct{Turn to the \greenribbon and sing the psalms with their antiphons prescribed there in this manner.}
				
				\instruct{The minister intones all antiphons until the asterisk, when he is then joined by the whole choir. Once the antiphon has been intoned, all psalms are sung to its tone until another antiphon is began.}
				
				\instruct{Psalms themselves are sung in this manner: the precentor or succentor sings the verse until the asterisk, when he is then joined by that entire side of the congregation. The other then begins the next verse, and that side of the congregation joins after the asterisk.}
				
				\instruct{Once a psalm is complete, if another antiphon is to be sung, first the previous antiphon is repeated. This is usually after every psalm. that is, each antiphon has one psalm under it, but on occasions multiple psalms are sung to the same antiphon. In this case, the antiphon is only repeated once all its psalms have been sung.}
				
			\vesection{Chapter}\label{vespers:chapter}
			
				\instruct{Once the psalmody is finished, the lector will read the chapter from his place. This is a short reading of Scripture, no more than three verses in length. The congregation responds:}
				
				\skiplines{0.5}\gabcsnippet{(c3) Thanks(h) be(h) to(h) God.(h)}
				
				\instruct{There is usually a responsory to accompany the chapter, which is sung in this manner.}
				
				\instruct{The lector will sing the responsory until the asterisk, when he is joined by the whole choir. Finishing this, the lector will sing any verses to which the whole congregation sings the prescribed response.}
				
				\instruct{Even when the \enquote{Glory be} is sung, the lector sings it alone, but all the congregation turns toward the altar, makes a profound bow, and crosses themselves as it is sung.}
				
				\instruct{After the responsory is the hymn. One of the cantors will intone the first line of the hymn when he is joined by the whole congregation.}
				
				\instruct{After the hymn is a versicle which the minister proclaims the verse and the whole congregation gives the response.}
				
				\instruct{Turn to the \redribbon and sing the responsory, if any, hymn, and versicle.}
				
			\vesection{Magnificat}\label{vespers:magnificat}
			
				\instruct{After the versicle has been sung, the minister proceeds to stand before the altar and intones the antiphon to the Canticle of Mary until the asterisk, when he is joined by the whole choir. The precentor, succentor, and congregation sings the verses of the canticle in like manner as the psalms.}
				
				\instruct{While the canticle is sung, where possible, the minister incenses the altar.}
				
			\vesectionp{Preces}\label{vespers:preces}
			
				\instruct{The preces are said on feria, that is, days besides Sunday when no feast day occurs. But they are not said from December 17 until after the Epiphany. The whole congregation kneels.}
				
				\instruct{The precentor and succentor alternate singing \enquote{Kyrie, eleison} to this melody. Each invocation is sung thrice.}
				
				\skiplines{0.5}\gregorioscore{common/preces/kyrie.gabc}
				
				\instruct{The first invocation and the last are sung by the precentor.}
				
				\instruct{Make this prayer quietly:}
				
				Our Father, who art in heaven, hallowed be thy name. Thy kingdom come. Thy will be done on earth as it is in heaven. Give us this day our daily bread and forgive us our trespasses as we forgive those who trespass against us. And lead us not into temptation but deliver us from evil.
				
				\instruct{Then the petitions are said:}
				
				\skiplines{0.5}\gregorioscore{common/preces/petitions.gabc}
				
				\instruct{This psalm is then recited by the congregation in a low voice:}
				
				\veheading{Psalm l: Miserere mei Deus}
				The King shall rejoice in thy strength, O Lord * exceedingly glad shall he be of thy salvation.

Thou hast given him his heart's desire * and hast not denied him the request of his lips.

For thou shall prevent him with the blessings of goodness * and shall set a crown of pure gold upon his head.

He asked life of thee, and thou gavest him a long life * even forever and ever.

His honor is great in thy salvation * glory and great worship shall thou lay upon him.

For thou shalt given him everlasting felicity * and make him glad with the joy of thy countenance.

And why? because the King putteth his trust in the Lord * and in the mercy of the most High he shall not miscarry.

All thine enemies shall feel thy hand * thy right and shall find out them who hate thee.

Thou shall make them like a fiery oven in the time of thy wrath * the Lord shall destroy them in his displeasure, and the fire shall consume them.

Their fruit shall thou root out of the earth * and their seed from among the children of men.

For they intended mischief against thee * and imagined such a device as they are not able to perform.

Therefore shall thou put them to flight * and the strings of thy bow shall thou make ready against the face of them.

Be thou exalted, O Lord, in thine own strength * so will we sing, and praise thy power.
				
				\instruct{The priest alone stands and proceeds to stand before the altar, where he proclaims this peition:}
				
				\skiplines{0.5}\gregorioscore{common/preces/conclusion.gabc}
				
			\vesection{Conclusion}\label{vespers:conclusion}
			
				\instruct{The minister proceeds again to stand before the altar and says this versicle:}
				
				\skiplines{0.5}\gabcsnippet{(c3) <sp>v</sp> O(h) Lord,(h) hear(h) my(f) prayer.(g) <sp>r</sp>(::) And(h) let(h) my(h) cry(h) come(h) un(h)to(f) thee.(f)}
				
				\instruct{He will then proclaim the invocation \enquote{Let us pray}, after which he will genuflect. If the preces have been said, the congregation kneels during the collect; otherwise they remain standing.}
				
				\instruct{The minister will then pray the collect of the day. The congregation responds:}
				
				\skiplines{0.5}\gabcsnippet{(c3) A(gh)men.(h)}
				
				\instruct{This petition concludes the office:}
				
				\skiplines{0.5}\gregorioscore{common/conclusion.gabc}
				
				\instruct{Then kneel and make this prayer quietly:}
				
				Our Father, who art in heaven, hallowed be thy name. Thy kingdom come. Thy will be done on earth as it is in heaven. Give us this day our daily bread and forgive us our trespasses as we forgive those who trespass against us. And lead us not into temptation but deliver us from evil.
			
		\vechapter{The Order of Compline}
	
	
	\gresetinitiallines{1}
	
	\vepart{Psalter}
	
		\vechapter{Sunday at First Vespers}
		
			\veant{1}{vi}
			\skiplines{1}\gregorioscore{ant/benedictus_dominus.gabc}
			
			\greannotation{\veformat{Ps. 143}}
			\greannotation{\veformat{vi. i.}}
			\skiplines{1}\gabcsnippet{(c3) Bles(hv)ed(iv) (:?) be(jrv) the() Lord(jv) my(i) strength(jv) <sp>*</sp>(:) who(jrv) teacheth() my() hands() to() war,() and() my(jr) <b>fin</b>(h)gers(ij) to(i) (hr) fight.(h)}
			
			\begin{multicols}{2}
	My hope and my fortress, my castle and deliverer, my defender in whom \textbf{I} trust * who subdueth my people \textbf{that} is under me.
	
	Lord, what is man that thou hast such respect un\textbf{to} him? * or the son of man, that thou \textbf{so} regardest him?
	
	Man is like a thing \textbf{of} nought * his time passeth away \textbf{like} a shadow.
	
	Bow thy heavens, O Lord, and \textbf{come} down * touch the moun\textbf{tains}, and they shall smoke.
	
	Cast forth thy lightning, and \textbf{tear} them * shoot out thine arrows, \textbf{and} consume them.
	
	Send down thine hand from \textbf{a}bove * deliver me, and take me out of the great waters, from the hand \textbf{of} strange children.
	
	Whose mouth talketh of van\textbf{i}ty * and their right hand is a right \textbf{hand} of wickedness.
	
	I will sing a new song unto thee, \textbf{O} God * and sing praises unto thee u\textbf{pon} a ten-stringed lute.
	
	Thou hast given victory un\textbf{to} kings * and hast delivered David thy servant from the \textbf{pe}ril of the sword.
	
	Save me, and deliver me from the hand of strange \textbf{child}ren * whose mouth talketh of vanity, and their right hand is a right hand \textbf{of} iniquity.
	
	That our sons may grow up as the \textbf{young} plants * and that our daughters may be as the polished corners \textbf{of} the temple.
	
	That our garners may be full and plenteous with all manner \textbf{of} store * that our sheep may bring forth thousands and ten \textbf{thou}sands in our streets.
	
	That our oxen may be strong to labor, that there be no \textbf{de}cay * no leading into captivity, and no com\textbf{plain}ing in our streets.
	
	Happy are the people that are in such \textbf{a} case * yea, blessed are the people who have \textbf{the} Lord for their God.
	
	Glory be to the Father, and to \textbf{the} Son * and to the \textbf{Ho}ly Spirit.
	
	As it was in the beginning is now, and ever \textbf{shall} be * world with\textbf{out} end. Amen.
\end{multicols} % tone 6 has only the one ending
		
			\instruct{Turn to the \whiteribbon at the chapter rite, pp. \pageref{vespers:chapter}.}
	
	\vepart{Propers}
	
		\vechapter{First Sunday in Advent}
			
			\skiplines{-1}\vesectionp{At First Vespers}
				
				\resp{Ecce dies veniunt}{viii}
				\gregorioscore{resp/ecce_dies_veniunt.gabc}
					
				\instruct{Turn to the \blueribbon and sing the hymn and versicle.}
				
				\skiplines{1}
				\veant{M}{I. i}
				\gregorioscore{ant/ecce_nomen_domini.gabc}
				
				\begin{multicols}{2}
	For \textbf{he} hath re\textbf{gard}ed * the \textbf{low}liness of \textbf{his} handmaiden.
	
	For \textbf{be}hold, from \textbf{hence}forth * all  ge\textbf{ner}ations shall \textbf{call} me blessed.
	
	For he that is mighty \textbf{hath} magni\textbf{fied} me * --- and ho\textbf{ly} is his Name.
	
	And his mercy is \textbf{on} them that \textbf{fear} him * --- throughout all \textbf{gen}erations.
	
	He hath \textbf{shewed} strength with \textbf{his} arm * he hath \textbf{scat}tered the proud in the imagina\textbf{tion} of their hearts.
	
	He hath put down \textbf{the} mighty from \textbf{their} seat * and hath ex\textbf{al}ted the \textbf{hum}ble and meek.
	
	He hath filled \textbf{the} hungry with \textbf{good} things * and the \textbf{rich} he hath sent \textbf{em}pty away.
	
	He, remembering his mercy, hath holpen \textbf{his} servant Is\textbf{ra}el * as he \textbf{prom}ised to our forefathers, to Abraham and his \textbf{seed} forever.
	
	Glory be to \textbf{the} Father, and to \textbf{the} Son * \textbf{---} and to the \textbf{Ho}ly Spirit.
	
	As it was in the beginning, is now, \textbf{and} ever \textbf{shall} be * --- world with\textbf{out} end. Amen.
\end{multicols}
			
				\instruct{Turn to the \whiteribbon at the conclusion rite, pp. \pageref{vespers:conclusion}.}
			
			\vesection{At Compline}
			
			\vesection{At Second Vespers}
	
	\vepart{Commons}
	
		\vechapter{Common of Advent I}
		
			\skiplines{-1}\resp{Tu exurgens Domine}{iv}
			\gregorioscore{resp/tu_exurgens_domine.gabc}
			
			\gresetinitiallines{0}
\gresetlastline{justified}

\vecomment{Conditor alme siderum}

\gregorioscore{hymn/conditor_alme_siderum/1.gabc}
\skiplines{\vehymn}\begin{nstabbing}
	\>2. Thou, \>griev-\>ing \>that \>the \>an-\>cient \>curse\\
	\>3. Thou \>cam'st \>the \>Bride-\>groom \>of \>the \>Bride,\\
	\>4. At \>whose \>dred \>name, \>ma-\>jes-\>tic \>now,\\
	\>5. O \>thou \>whose \>com-\>ing \>is \>with \>dread,\\
	\>6. To \>God \>the \>Fa-\>ther, \>God \>the \>Son,
\end{nstabbing}

\skiplines{0.5}\gregorioscore{hymn/conditor_alme_siderum/2.gabc}
\skiplines{\vehymn}\begin{nstabbing}
	\>2. Should \>doom \>to \>death \>the \>u-\>ni-\>verse,\\
	\>3. As \>drew \>the \>world \>to \>eve-\>ning-\>tide;\\
	\>4. All \>knees \>must \>bend, \>all \>hearts \>must \>bow;\\
	\>5. To \>judge \>and \>doom \>the \>quick \>and \>dead,\\
	\>6. And \>God \>the \>Spir-\>it, \>Three \>in \>one
\end{nstabbing}

\skiplines{0.5}\gregorioscore{hymn/conditor_alme_siderum/3.gabc}
\skiplines{\vehymn}\begin{nstabbing}
	\>2. Hast \>found \>the \>med'-\>cine, \>full \>of \>grace,\\
	\>3. Pro-\>cee-\>ding \>from \>a \>vir-\>gin \>shrine,\\
	\>4. And \>things \>cel-\>es-\>tial \>thee \>shall \>own,\\
	\>5. Pre-\>serve \>us, \>while \>we \>dwell \>be-\>low,\\
	\>6. Laud, \>ho-\>nor, \>might, \>and \>glo-\>ry \>be,
\end{nstabbing}

\skiplines{0.5}\gregorioscore{hymn/conditor_alme_siderum/4.gabc}
\skiplines{\vehymn}\begin{nstabbing}
	\>2. To \>save \>and \>heal \>a \>ru-\>ined \>race.\\
	\>3. The \>spot-\>less \>Vic-\>tim \>all \>di-\>vine.\\
	\>4. And \>things \>terr-\>es-\>tial, \>Lord \>a-\>lone.\\
	\>5. From \>ev-\>ery \>in-\>sult \>of \>the \>foe.\\
	\>6. From \>age \>to \>age \>e-\>ter-\>na-\>ly. \>A-\>men.
\end{nstabbing}

\gresetinitiallines{1}
			
			\gresetinitiallines{0}\gabcsnippet{
				<sp>v</sp> Drop(hv) down,(hv) ye(hv) hea(hv)vens,(hv) from(hv) a(hv)bove.(ghGFEfhGF) <sp>r</sp>(::) And(hv) let(hv) the(hv) skies(hv) pour(hv) down(hv) right(hv)eous(f)ness:(gv) (;) let(hv) the(hv) earth(hv) o(hv)pen,(hv) and(hv) let(hv) it(hv) bring(hv) forth(hv) sal(hv)va(f)tion.(f)
			}\gresetinitiallines{1}
				
			\instruct{The Magnificat is always proper on all days in Advent; turn to the \redribbon and sing as appointed there.}
			
		\vechapter{Common of Advent II}
		
			\skiplines{-1}\resp{Festina ne taraveris}{ii}
			\gregorioscore{resp/festina_ne_taraveris.gabc}
		
			\clearpage\gresetinitiallines{0}
\gresetlastline{justified}

\vecomment{Conditor alme siderum}

\gregorioscore{hymn/conditor_alme_siderum/1.gabc}
\skiplines{\vehymn}\begin{nstabbing}
	\>2. Thou, \>griev-\>ing \>that \>the \>an-\>cient \>curse\\
	\>3. Thou \>cam'st \>the \>Bride-\>groom \>of \>the \>Bride,\\
	\>4. At \>whose \>dred \>name, \>ma-\>jes-\>tic \>now,\\
	\>5. O \>thou \>whose \>com-\>ing \>is \>with \>dread,\\
	\>6. To \>God \>the \>Fa-\>ther, \>God \>the \>Son,
\end{nstabbing}

\skiplines{0.5}\gregorioscore{hymn/conditor_alme_siderum/2.gabc}
\skiplines{\vehymn}\begin{nstabbing}
	\>2. Should \>doom \>to \>death \>the \>u-\>ni-\>verse,\\
	\>3. As \>drew \>the \>world \>to \>eve-\>ning-\>tide;\\
	\>4. All \>knees \>must \>bend, \>all \>hearts \>must \>bow;\\
	\>5. To \>judge \>and \>doom \>the \>quick \>and \>dead,\\
	\>6. And \>God \>the \>Spir-\>it, \>Three \>in \>one
\end{nstabbing}

\skiplines{0.5}\gregorioscore{hymn/conditor_alme_siderum/3.gabc}
\skiplines{\vehymn}\begin{nstabbing}
	\>2. Hast \>found \>the \>med'-\>cine, \>full \>of \>grace,\\
	\>3. Pro-\>cee-\>ding \>from \>a \>vir-\>gin \>shrine,\\
	\>4. And \>things \>cel-\>es-\>tial \>thee \>shall \>own,\\
	\>5. Pre-\>serve \>us, \>while \>we \>dwell \>be-\>low,\\
	\>6. Laud, \>ho-\>nor, \>might, \>and \>glo-\>ry \>be,
\end{nstabbing}

\skiplines{0.5}\gregorioscore{hymn/conditor_alme_siderum/4.gabc}
\skiplines{\vehymn}\begin{nstabbing}
	\>2. To \>save \>and \>heal \>a \>ru-\>ined \>race.\\
	\>3. The \>spot-\>less \>Vic-\>tim \>all \>di-\>vine.\\
	\>4. And \>things \>terr-\>es-\>tial, \>Lord \>a-\>lone.\\
	\>5. From \>ev-\>ery \>in-\>sult \>of \>the \>foe.\\
	\>6. From \>age \>to \>age \>e-\>ter-\>na-\>ly. \>A-\>men.
\end{nstabbing}

\gresetinitiallines{1}
			
			\gresetinitiallines{0}\gabcsnippet{
				<sp>v</sp> Drop(hv) down,(hv) ye(hv) hea(hv)vens,(hv) from(hv) a(hv)bove.(ghGFEfhGF) <sp>r</sp>(::) And(hv) let(hv) the(hv) skies(hv) pour(hv) down(hv) right(hv)eous(f)ness:(gv) (;) let(hv) the(hv) earth(hv) o(hv)pen,(hv) and(hv) let(hv) it(hv) bring(hv) forth(hv) sal(hv)va(f)tion.(f)
			}\gresetinitiallines{1}
			
			\instruct{The Magnificat is always proper on all days in Advent; turn to the \redribbon and sing as appointed there.}
			
\end{document}