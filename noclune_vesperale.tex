\documentclass[14pt,twoside]{extarticle}
\usepackage{moresize}
\usepackage{fontspec}
\usepackage{aeguill}
\usepackage{titlesec}
\usepackage[maxlevel=3]{csquotes}
\usepackage{lettrine}
\usepackage{xifthen}
\usepackage{yfonts}
\usepackage{parskip}
\usepackage[dvipsnames]{xcolor}
\usepackage{fancyhdr}
\usepackage{hanging}
\usepackage[autocompile,allowdeprecated=false]{gregoriotex}
\usepackage[twoside]{geometry}
\usepackage{multicol}
\usepackage{paracol}
\usepackage[latin,english]{babel}
\usepackage{enumitem}
\usepackage{tabto}
\usepackage{titlesec}
\usepackage[savepos]{zref}
\usepackage{setspace}
\usepackage{xstring}
\def\ReplaceStr#1{%
	\IfSubStr{#1}{\newline}{%
		\StrSubstitute{#1}{\newline}{ }}{#1}}
%\usepackage{libertine}

\setlength{\columnseprule}{0.025pt}
\setlength{\columnsep}{1.5em}
\setlength{\topskip}{7pt}

\def\vehymn{-0.50}

\makeatletter
\patchcmd{\pcol@buildcolseprule}%
{\hrule\@height\@tempdima\@width\columnseprule}%
{\hrule\@height\dimexpr\@tempdima-0.5ex\@width\columnseprule\@depth-1pt}{}{}
\makeatother

\gredefsizedsymbol{GreCross}{greextra}{Cross}
\gredefsizedsymbol{GreDagger}{greextra}{Dagger}
\gredefsizedsymbol{GreCrossAlt}{greextra}{Cross.alt}
\gredefsizedsymbol{GreStarHeight}{greextra}{StarHeight}
\gredefsizedsymbol{greABar}{greextra}{ABar}
\gredefsizedsymbol{greVBar}{greextra}{VBar}
\gredefsizedsymbol{greRBar}{greextra}{RBar}

\gresetspecial{r}{\textcolor{red}{\Rbar{}.}}
\gresetspecial{v}{\textcolor{red}{\Vbar{}.}}
\gresetspecial{a}{\textcolor{red}{\Abar{}.}}
\gresetspecial{+}{\textcolor{red}{\GreCross{17}}}
\gresetspecial{1}{\textcolor{red}{\GreDagger{17}}}
\gresetspecial{2}{\textcolor{red}{\GreCrossAlt{17}}}
\gresetspecial{*}{\textcolor{red}{\GreStarHeight{17}}}
%%% for spacing in chant %%%
\gresetspecial{ }{\textcolor{white}{x}}
\gresetspecial{  }{\textcolor{white}{xx}}
\gresetspecial{   }{\textcolor{white}{xxx}}
\gresetspecial{    }{\textcolor{white}{xxxx}}

\grechangestaffsize{30}
%\grechangedim{abovelinestextheight}{0.25cm}{fixed}
%\grechangedim{spacebeneathtext}{0.5cm}{fixed}
%\grechangedim{annotationseparation}{0.25cm}{fixed}


\newcommand{\skiplines}[1]{\pagebreak[1]\vspace*{#1\baselineskip}}

\makeatletter
\newcommand{\veformat}[1]{\textcolor{red}{\textbf{#1}}}
\newcommand{\veant}[2]{\greannotation{#1. \veformat{Ant.}}\greannotation{\veformat{#2.}}}
\newcommand{\instruct}[1]{\textcolor{red}{\textit{#1}}}

\newcommand{\whiteribbon}[0]{\textcolor{black}{\textbf{white ribbon}} }%
\newcommand{\greenribbon}[0]{\textcolor{ForestGreen}{\textbf{green ribbon}} }%
\newcommand{\redribbon}[0]{\textcolor{red}{\textbf{red ribbon}} }%
\newcommand{\blueribbon}[0]{\textcolor{blue}{\textbf{blue ribbon}} }%
\makeatother

%%% HEADERS %%%
\makeatletter
\newcommand{\vepart}[1]{%
	\cleardoublepage%
	\vspace*{\fill}%
	\begin{center}%
		\setstretch{2}
		{\Huge \selectfont \uppercase{ \textbf{#1}}}\par%
	\end{center}%
	\vspace*{\fill}%
	\fancyhead{}
}%
\makeatother

\newcommand{\vechapter}[1]{%
	\cleardoublepage%
	\fancyhead[RO,RE]{text}
	\StrSubstitute{#1}{\\}{\space}[\vetempmacrolol]%
	\fancyhead[LO,LE]{\vetempmacrolol}%
	\begin{center}
		\setstretch{1.5}
		{\LARGE \selectfont \color{red} \uppercase{ \textbf{#1}}}\par%
	\end{center}
	\skiplines{1}
}%
\makeatother

\newcommand{\vesectionp}[1]{%
	\skiplines{1}
	\fancyhead[RO,RE]{#1}%
	\begin{center}
		\setstretch{1.5}
		{\Large \selectfont \uppercase{ \textbf{#1}}}\par%
	\end{center}
	\skiplines{1}
}%
\makeatother

\newcommand{\vesection}[1]{%
	\clearpage\skiplines{-2}\vesectionp{#1}%
}%
\makeatother

\makeatletter
\newcommand{\veheading}[1]{%
	\skiplines{1}%
	\begin{center}
		\setstretch{1.5}
		{\selectfont \textit{#1}}\par%
	\end{center}
	\skiplines{1}
}%
\makeatother

\makeatletter
\newcommand{\vecomment}[1]{%
	\hfill{\small\selectfont \textit{#1}.}%
}%
\makeatother

%%% OTHER FORMATS %%%

\makeatletter
\newcommand{\red}[1]{%
	{\color{red} #1}%
}%
\makeatother

\makeatletter
\newcommand{\chapter}[1]{%
	\gresetinitiallines{1}%
	\greannotation{\veformat{Chap.}}%
	\greannotation{\veformat{Dir.}}%
	\gabcsnippet{#1}%
}%
\makeatother

\makeatletter
\newcommand{\resp}[2]{%
	\vecomment{#1}
	
	\gresetinitiallines{1}%
	\greannotation{\veformat{Resp.}}%
	\greannotation{\veformat{#2.}}%
}%
\makeatother

\makeatletter
\newcommand{\tone}[2]{%
	\greannotation{\veformat{Ps.}}
	\greannotation{\veformat{#1.}}
	
	\gabcsnippet{#2}
}%
\makeatother

\makeatletter
\newcommand{\repant}[1]{%
	\gresetspecial{*}{}
	\nopagebreak[4]\skiplines{1}\nopagebreak[4]\gresetinitiallines{0}\gregorioscore{#1}\gresetinitiallines{1}
	\gresetspecial{*}{\textcolor{red}{\GreStarHeight{17}}}
}%
\makeatother



%%% FANCY %%%
\makeatletter  
\newcounter{score}
\newcounter{tabstop}[score]
\newcommand{\grealign}{%
	\@bsphack%
	\ifgre@boxing\else%
	\kern\gre@dimen@begindifference%
	\stepcounter{tabstop}%
	\expandafter\zsavepos{stop-\thescore-\thetabstop}%
	\kern-\gre@dimen@begindifference%
	\fi%
	\@esphack%
}

\newcommand{\setstops}{%
	\gdef\nstabbing@stops{%
		\hspace*{-\oddsidemargin}\hspace{-1in}%
		\hspace*{\zposx{stop-\thescore-1} sp}\=%
	}%
	\count@=\@ne
	\loop\ifnum\count@<\value{tabstop}%
	\begingroup\edef\x{\endgroup
		\noexpand\g@addto@macro\noexpand\nstabbing@stops{%
			\noexpand\hspace{-\noexpand\zposx{stop-\thescore-\the\count@} sp}%
			\noexpand\hspace{\noexpand\zposx{stop-\thescore-\the\numexpr\count@+1} sp}\noexpand\=%
		}%
	}\x
	\advance\count@\@ne
	\repeat
	\nstabbing@stops\kill
}
\makeatother

\newenvironment{nstabbing}
{\setlength{\topsep}{0pt}%
	\setlength{\partopsep}{0pt}%
	\tabbing%
	\setstops}
{\endtabbing\stepcounter{score}}

\begin{document}
	\newgeometry{margin=0pt}
	\begin{titlepage}
		\vspace*{\fill}
		
		\begin{center}
			
			\textbf{\Huge\color{red} VESPERALE}
			
			ACCORDING TO THE RITE OF THE
			
			\textbf{\LARGE HOLY ROMAN CHURCH}
			
			\vspace*{\fill}
			
			\textbf{\Large\color{red} VOLUME I}
			
			Seasons of Advent and Christmas\\Until the Epiphany of the Lord
			
		\end{center}
		
		\vspace*{\fill}
	\end{titlepage}
	
	\newgeometry{top=0.75in,bottom=0.75in,inner=0.75in,outer=0.75in,includefoot}
	
	\pagenumbering{arabic}
	\gresetinitiallines{0}
	
	\raggedbottom
	
	%\tableofcontents
	
	
	%%% COPY ABOVE %%%
	
	\vechapter{Perpetual Calendar}
	\skiplines{-1}\vesectionp{November}
		\begin{enumerate}[noitemsep, nolistsep]
			\setcounter{enumi}{26}
			\item {\textit{\small Earliest day for Advent.}}
			\item % 28
			\item Saints Saturnius and Sisinnius, Martyrs.\footnote{Outside of Advent, here is celebrated the Eve of Saint Andrew.} \red{3 lect. simplex.}
			\item \red{Saint Andrew, Apostle.} Minor feast.
		\end{enumerate}
	\vesectionp{December}
		\begin{enumerate}[noitemsep, nolistsep]
			\item \textit{\small Saint Nahum, Prophet.} % 1
			\item \textit{\small Saint Habakkuk, Prophet.}  % 2
			\item \textit{\small Saint Zephaniah, Prophet.}  % 3
			\item  % 4
			\item % 5
			\item Saint Nicholas, Bishop and Confessor. \red{9 lect. duplex.}
			\item Saint Ambrose, Bishop and Doctor. \red{9 lect. triplex.}
			\item \red{Conception of the Blessed Virgin Mary.} Major feast.
			\item % 9
			\item % 10
			\item Saint Damasus, Pope and Confessor. \red{3 lect. duplex.} % 11
			\item  % 12
			\item Saint Lucy, Virgin and Martyr. \red{9 lect. duplex.}
			\item  % 14
			\item % 15
			\item \textit{\small Saint Haggai, Prophet.}  % 16
			\item {\textit{\small O Wisdom.}} % 17
			\item  % 18
			\item  % 19
			\item  % 20
			\item \textit{\small Saint Micah, Prophet.}  % 21
			\item  % 22
			\item  % 23
			\item \textcolor{purple}{Eve of the Nativity.} \red{9 lect. duplex.}
			\item \textbf{\red{Nativity of the Lord.}} Solemnity. \textit{With octave.}
			\item \red{Saint Stephen, Protomartyr.} Major feast.
			\item \red{Saint John, Apostle and Evangelist.} Major feast.
			\item \red{Holy Innocents, Martyrs.} Major feast.
			\item \red{Saint Thomas Becket, Martyr.} Major feast.
			\item Of the Octave. \red{9 lect. triplex.} % 30
			\item \red{Holy Forefathers of the Lord.} Principle feast.
		\end{enumerate}
	\vesectionp{January}
		\begin{enumerate}[noitemsep, nolistsep]
			\item \red{Circumcision of the Lord.} Major feast.
			\item Octave of Saint Stephen. \red{9 lect. triplex.}
			\item Octave of Saint John. \red{9 lect. triplex.}
			\item Octave of the Innocents. \red{9 lect. triplex.}
		\end{enumerate}
	
	
	\gresetinitiallines{1}
	
	\vepart{Ordinary}
	
		\vechapter{The Order of Vespers}
			\skiplines{-1}\vesectionp{Introduction}
			
				\instruct{Before the service begins, kneel and make these prayers quietly.}
				
				Our Father, who art in heaven, hallowed be thy name. Thy kingdom come. Thy will be done on earth as it is in heaven. Give us this day our daily bread and forgive us our trespasses as we forgive those who trespass against us. And lead us not into temptation but deliver us from evil.
					
				Hail Mary, full of grace, the Lord is with thee. Blessed art thou among women and blessed is the fruit of thy womb, Jesus. Holy Mary, Mother of God, pray for us sinners now and at the hour of our death. Amen.
				
				\instruct{Stand as the minister begins from his place:}
				
				\gresetinitiallines{0}
				\skiplines{0.5}\gabcsnippet{(c3) <sp>v</sp> O(fv) God,(hv) make(hv) speed(iv) to(h) save(g) me.(hv) <sp>r</sp>(::) O(hv) Lord,(hv) make(hv) haste(hv) to(hv) help(g) me.(hv)}
				
				\instruct{The congregation continues together, turning toward the altar, making a profound bow, and crossing themselves at the invocation of the Holy Trinity here and whenever else:}
				
				\skiplines{0.5}\gabcsnippet{(c3) Glor(h)y(h) be(h) to(h) the(h) Fa(h)ther,(h) and(h) to(h) the(h) Son,(h) (,) and(h) to(h) the(h) Ho(h)ly(h) Spir(g)it.(hv) (:) As(h) it(h) was(h) in(h) the(h) be(h)gin(h)ning,(h) is(h) now(h) and(h) ev(h)er(h) shall(g) be:(hv) (,) world(hv) with(hv)out(hv) end.(hv) A(g)men.(hv) (::) A(hv)le(hi)lu(h)ia.(hg)}
				
			\vesectionp{Psalmody}
			
				\instruct{Turn to the \greenribbon and sing the psalms with their antiphons prescribed there in this manner.}
				
				\instruct{The minister intones all antiphons until the asterisk, when he is then joined by the whole choir. Once the antiphon has been intoned, all psalms are sung to its tone until another antiphon is began.}
				
				\instruct{Psalms themselves are sung in this manner: the precentor or succentor sings the verse until the asterisk, when he is then joined by that entire side of the congregation. The other then begins the next verse, and that side of the congregation joins after the asterisk.}
				
				\instruct{Once a psalm is complete, if another antiphon is to be sung, first the previous antiphon is repeated. This is usually after every psalm. that is, each antiphon has one psalm under it, but on occasions multiple psalms are sung to the same antiphon. In this case, the antiphon is only repeated once all its psalms have been sung.}
				
			\vesection{Chapter}
			
				\instruct{Once the psalmody is finished, the lector will read the chapter from his place. This is a short reading of Scripture, no more than three verses in length. The congregation responds:}
				
				\skiplines{0.5}\gabcsnippet{(c3) Thanks(h) be(h) to(h) God.(h)}
				
				\instruct{There is usually a responsory to accompany the chapter, which is sung in this manner.}
				
				\instruct{The lector will sing the responsory until the asterisk, when he is joined by the whole choir. Finishing this, the lector will sing any verses to which the whole congregation sings the prescribed response.}
				
				\instruct{Even when the \enquote{Glory be} is sung, the lector sings it alone, but all the congregation turns toward the altar, makes a profound bow, and crosses themselves as it is sung.}
				
				\instruct{After the responsory is the hymn. One of the cantors will intone the first line of the hymn when he is joined by the whole congregation.}
				
				\instruct{After the hymn is a versicle which the minister proclaims the verse and the whole congregation gives the response.}
				
				\instruct{Turn to the \redribbon and sing the responsory, if any, hymn, and versicle.}
				
			\vesection{Magnificat}
			
				\instruct{After the versicle has been sung, the minister proceeds to stand before the altar and intones the antiphon to the Canticle of Mary until the asterisk, when he is joined by the whole choir. The precentor, succentor, and congregation sings the verses of the canticle in like manner as the psalms.}
				
				\instruct{While the canticle is sung, where possible, the minister incenses the altar.}
				
			\vesectionp{Preces}
			
				\instruct{The preces are said on feria, that is, days besides Sunday when no feast day occurs. But they are not said from December 17 until after the Epiphany. The whole congregation kneels.}
				
				\instruct{The precentor and succentor alternate singing \enquote{Kyrie, eleison} to this melody. Each invocation is sung thrice.}
				
				\skiplines{0.5}\gregorioscore{common/preces/kyrie.gabc}
				
				\instruct{The first invocation and the last are sung by the precentor.}
				
				\instruct{Make this prayer quietly:}
				
				Our Father, who art in heaven, hallowed be thy name. Thy kingdom come. Thy will be done on earth as it is in heaven. Give us this day our daily bread and forgive us our trespasses as we forgive those who trespass against us. And lead us not into temptation but deliver us from evil.
				
				\instruct{Then the petitions are said:}
				
				\skiplines{0.5}\gregorioscore{common/preces/petitions.gabc}
				
				\instruct{This psalm is then recited by the congregation in a low voice:}
				
				\veheading{Psalm l: Miserere mei Deus}
				Have mercy upon me, O God, after thy great mercy * according to the multitude of thy mercies do away mine offenses.
	
Wash me thoroughly from my wickedness * and cleanse me from my sin.
	
For I acknowledge my faults * and my sin is ever before me.
	
Against thee only have I sinned, and done this evil in thy sight * that thou mightest be justified in thy saying, and clear when thou art judged.

Behold, I was shapen in wickedness * and with sin hath my mother conceived me.
	
But lo, thou requirest truth in the inward parts * and shalt make me to understand wisdom secretly.
	
Thou shalt purge me with hyssop, and I shall be clean * thou shalt wash me and I shall be whiter than the snow.
	
Thou shalt make me hear of joy and gladness * that the bones which thou hast broken may rejoice.
	
Turn thy face from my sins * and put out all my misdeeds.
	
Make me a clean heart, O God * and renew a right spirit within me.
	
Cast me not away from thy presence * and take not thy Holy Spirit from me.
	
O give me the comfort of thy help again * and establish me with thy free Spirit.
	
Then shall I teach thy ways unto the wicked * and sinners shall be converted unto thee.
	
Deliver me from blood-guiltiness, O God, thou that art the God of my salvation * and my tongue shall sing of thy righteousness.
	
Thou shalt open my lips, O Lord * and my mouth shall shew thy praise.
	
For thou desirest no sacrifice, else would I give it thee * but thou delightest not in burnt-offerings.
	
The sacrifice of God is a troubled spirit * a broken and contrite heart, O God, shalt thou not despise.
	
O be favorable and gracious unto Zion * build thou the walls of Jerusalem.
	
Then shalt thou be pleased with the sacrifice of righteousness, with the burnt-offerings and oblations * then shall they offer young bullocks upon thine altar.

				
				\instruct{The priest alone stands and proceeds to stand before the altar, where he proclaims this peition:}
				
				\skiplines{0.5}\gregorioscore{common/preces/conclusion.gabc}
				
			\vesection{Conclusion}
			
				\instruct{The minister proceeds again to stand before the altar and says this versicle:}
				
				\skiplines{0.5}\gabcsnippet{(c3) <sp>v</sp> O(h) Lord,(h) hear(h) my(f) prayer.(g) <sp>r</sp>(::) And(h) let(h) my(h) cry(h) come(h) un(h)to(f) thee.(f)}
				
				\instruct{He will then proclaim the invocation \enquote{Let us pray}, after which he will genuflect. If the preces have been said, the congregation kneels during the collect; otherwise they remain standing.}
				
				\instruct{The minister will then pray the collect of the day. The congregation responds:}
				
				\skiplines{0.5}\gabcsnippet{(c3) A(gh)men.(h)}
				
				\instruct{This petition concludes the office:}
				
				\skiplines{0.5}\gregorioscore{common/conclusion.gabc}
				
				\instruct{Then kneel and make this prayer quietly:}
				
				Our Father, who art in heaven, hallowed be thy name. Thy kingdom come. Thy will be done on earth as it is in heaven. Give us this day our daily bread and forgive us our trespasses as we forgive those who trespass against us. And lead us not into temptation but deliver us from evil.
			
		\vechapter{The Order of Compline}
	
	
	\gresetinitiallines{1}
	
	\vepart{Psalter}
	
		\vechapter{Sunday at First Vespers}
		
			\instruct{Turn to the \whiteribbon at the chapter rite.}
	
	\vepart{Propers}
	
		\vechapter{First Sunday in Advent}
			
			\skiplines{-1}\vesectionp{At First Vespers}
				
				\resp{Ecce dies veniunt}{viii}
				\gregorioscore{resp/ecce_dies_veniunt.gabc}
					
				\instruct{Turn to the \blueribbon and sing the hymn and versicle.}
				
				\skiplines{1}
				\veant{M}{I. i}
				\gregorioscore{ant/ecce_nomen_domini.gabc}
				
				\begin{multicols}{2}
	For he hath re\textbf{gard}ed * the lowliness of \textbf{his} handmaiden.
	
	For behold, from \textbf{hence}forth * all  generations shall \textbf{call} me blessed.
	
	For he that is mighty hath magni\textbf{fied} me * and \textbf{ho}ly is his Name.
	
	And his mercy is on them who \textbf{fear} him * throughout all \textbf{gen}erations.
	
	He hath shewed strength with \textbf{his} arm * he hath scattered the proud in the imagi\textbf{na}tion of their hearts.
	
	He hath put down the mighty from \textbf{their} seat * and hath exalted the \textbf{hum}ble and meek.
	
	He hath filled the hungry with \textbf{good} things * and the rich he hath sent \textbf{emp}ty away.
	
	He, remembering his mercy, hath holpen his servant Is\textbf{ra}el * as he promised to our forefathers, to Abraham and his \textbf{seed} forever.
	
	Glory be to the Father, and to \textbf{the} Son * and to the \textbf{Ho}ly Spirit.
	
	As it was in the beginning, is now, and ever \textbf{shall} be * world with\textbf{out} end. Amen.
\end{multicols}
			
				\instruct{Turn to the \whiteribbon at the conclusion rite.}
			
			\vesection{At Compline}
			
			\vesection{At Second Vespers}
	
	\vepart{Commons}
	
		\vechapter{Common of Advent I}
		
			\skiplines{-1}\resp{Tu exurgens Domine}{iv}
			\gregorioscore{resp/tu_exurgens_domine.gabc}
			
			\greannotation{\veformat{Hymn.}}
\greannotation{\veformat{iv.}}

\vecomment{Conditor alme siderum}

\gregorioscore{hymn/conditor_alme_siderum.gabc}

\begin{multicols}{1}
	Thou, grieving that the ancient curse,\\
	Should doom to death the universe,\\
	Hast found the med'cine, full of grace,\\
	To save and heal a ruined race.
	
	Thou camst, the Bridegroom of the Bride,\\
	As drew the world to eveningtide;\\
	Proceeding from a virgin shrine,\\
	The spotless Victim all divine.
	
	At whose dread name, majestic mow,\\
	All knees must bend, all hearts must bow;\\
	And things celestial thee shall own,\\
	And things terrestial, Lord alone.
	
	O thou whose coming is with dread,\\
	To judge and doom the quick and dead,\\
	Preserve us, while we dwell below,\\
	From every insult of the foe.
	
	To God the Father, God the Son,\\
	And God the Spirit, Three in one,\\
	Laud, honor, might, and glory be,\\
	From age to age eternally.
	
	\skiplines{1}\gresetinitiallines{0}\gabcsnippet{(f3) A(fg/hg)men.(gfg)}\gresetinitiallines{1}
\end{multicols}
			
			\gresetinitiallines{0}\gabcsnippet{
				<sp>v</sp> Drop(hv) down,(hv) ye(hv) hea(hv)vens,(hv) from(hv) a(hv)bove.(ghGFEfhGF) <sp>r</sp>(::) And(hv) let(hv) the(hv) skies(hv) pour(hv) down(hv) right(hv)eous(f)ness:(gv) (;) let(hv) the(hv) earth(hv) o(hv)pen,(hv) and(hv) let(hv) it(hv) bring(hv) forth(hv) sal(hv)va(f)tion.(f)
			}\gresetinitiallines{1}
				
			\instruct{The Magnificat is always proper on all days in Advent; turn to the \redribbon and sing as appointed there.}
			
		\vechapter{Common of Advent II}
		
			\skiplines{-1}\resp{Festina ne taraveris}{ii}
			\gregorioscore{resp/festina_ne_taraveris.gabc}
		
			\clearpage\greannotation{\veformat{Hymn.}}
\greannotation{\veformat{iv.}}

\vecomment{Conditor alme siderum}

\gregorioscore{hymn/conditor_alme_siderum.gabc}

\begin{multicols}{1}
	Thou, grieving that the ancient curse,\\
	Should doom to death the universe,\\
	Hast found the med'cine, full of grace,\\
	To save and heal a ruined race.
	
	Thou camst, the Bridegroom of the Bride,\\
	As drew the world to eveningtide;\\
	Proceeding from a virgin shrine,\\
	The spotless Victim all divine.
	
	At whose dread name, majestic mow,\\
	All knees must bend, all hearts must bow;\\
	And things celestial thee shall own,\\
	And things terrestial, Lord alone.
	
	O thou whose coming is with dread,\\
	To judge and doom the quick and dead,\\
	Preserve us, while we dwell below,\\
	From every insult of the foe.
	
	To God the Father, God the Son,\\
	And God the Spirit, Three in one,\\
	Laud, honor, might, and glory be,\\
	From age to age eternally.
	
	\skiplines{1}\gresetinitiallines{0}\gabcsnippet{(f3) A(fg/hg)men.(gfg)}\gresetinitiallines{1}
\end{multicols}
			
			\gresetinitiallines{0}\gabcsnippet{
				<sp>v</sp> Drop(hv) down,(hv) ye(hv) hea(hv)vens,(hv) from(hv) a(hv)bove.(ghGFEfhGF) <sp>r</sp>(::) And(hv) let(hv) the(hv) skies(hv) pour(hv) down(hv) right(hv)eous(f)ness:(gv) (;) let(hv) the(hv) earth(hv) o(hv)pen,(hv) and(hv) let(hv) it(hv) bring(hv) forth(hv) sal(hv)va(f)tion.(f)
			}\gresetinitiallines{1}
			
			\instruct{The Magnificat is always proper on all days in Advent; turn to the \redribbon and sing as appointed there.}
			
\end{document}