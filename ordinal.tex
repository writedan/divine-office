\documentclass[14pt,twoside]{extarticle}
\usepackage{moresize}
\usepackage{fontspec}
\usepackage{aeguill}
\usepackage{titlesec}
\usepackage[maxlevel=3]{csquotes}
\usepackage{lettrine}
\usepackage{xifthen}
\usepackage{yfonts}
\usepackage{parskip}
\usepackage[dvipsnames]{xcolor}
\usepackage{fancyhdr}
\usepackage{hanging}
\usepackage[autocompile,allowdeprecated=false]{gregoriotex}
\usepackage[twoside]{geometry}
\usepackage{multicol}
\usepackage{paracol}
\usepackage[latin,english]{babel}
\usepackage{enumitem}
\usepackage{tabto}
\usepackage{titlesec}
\usepackage[savepos]{zref}
\usepackage{setspace}
\usepackage{xstring}
\def\ReplaceStr#1{%
	\IfSubStr{#1}{\newline}{%
		\StrSubstitute{#1}{\newline}{ }}{#1}}
%\usepackage{libertine}

\setlength{\columnseprule}{0.025pt}
\setlength{\columnsep}{1.5em}
\setlength{\topskip}{7pt}

\def\vehymn{-0.50}

\makeatletter
\patchcmd{\pcol@buildcolseprule}%
{\hrule\@height\@tempdima\@width\columnseprule}%
{\hrule\@height\dimexpr\@tempdima-0.5ex\@width\columnseprule\@depth-1pt}{}{}
\makeatother

\gredefsizedsymbol{GreCross}{greextra}{Cross}
\gredefsizedsymbol{GreDagger}{greextra}{Dagger}
\gredefsizedsymbol{GreCrossAlt}{greextra}{Cross.alt}
\gredefsizedsymbol{GreStarHeight}{greextra}{StarHeight}
\gredefsizedsymbol{greABar}{greextra}{ABar}
\gredefsizedsymbol{greVBar}{greextra}{VBar}
\gredefsizedsymbol{greRBar}{greextra}{RBar}

\gresetspecial{r}{\textcolor{red}{\Rbar{}.}}
\gresetspecial{v}{\textcolor{red}{\Vbar{}.}}
\gresetspecial{a}{\textcolor{red}{\Abar{}.}}
\gresetspecial{+}{\textcolor{red}{\GreCross{17}}}
\gresetspecial{1}{\textcolor{red}{\GreDagger{17}}}
\gresetspecial{2}{\textcolor{red}{\GreCrossAlt{17}}}
\gresetspecial{*}{\textcolor{red}{\GreStarHeight{17}}}
%%% for spacing in chant %%%
\gresetspecial{ }{\textcolor{white}{x}}
\gresetspecial{  }{\textcolor{white}{xx}}
\gresetspecial{   }{\textcolor{white}{xxx}}
\gresetspecial{    }{\textcolor{white}{xxxx}}

\grechangestaffsize{30}
%\grechangedim{abovelinestextheight}{0.25cm}{fixed}
%\grechangedim{spacebeneathtext}{0.5cm}{fixed}
%\grechangedim{annotationseparation}{0.25cm}{fixed}


\newcommand{\skiplines}[1]{\pagebreak[1]\vspace*{#1\baselineskip}}

\makeatletter
\newcommand{\veformat}[1]{\textcolor{red}{\textbf{#1}}}
\newcommand{\veant}[2]{\greannotation{#1. \veformat{Ant.}}\greannotation{\veformat{#2.}}}
\newcommand{\instruct}[1]{\textcolor{red}{\textit{#1}}}

\newcommand{\whiteribbon}[0]{\textcolor{black}{\textbf{white ribbon}} }%
\newcommand{\greenribbon}[0]{\textcolor{ForestGreen}{\textbf{green ribbon}} }%
\newcommand{\redribbon}[0]{\textcolor{red}{\textbf{red ribbon}} }%
\newcommand{\blueribbon}[0]{\textcolor{blue}{\textbf{blue ribbon}} }%
\makeatother

%%% HEADERS %%%
\makeatletter
\newcommand{\vepart}[1]{%
	\cleardoublepage%
	\vspace*{\fill}%
	\begin{center}%
		\setstretch{2}
		{\Huge \selectfont \uppercase{ \textbf{#1}}}\par%
	\end{center}%
	\vspace*{\fill}%
	\fancyhead{}
}%
\makeatother

\newcommand{\vechapter}[1]{%
	\cleardoublepage%
	\fancyhead[RO,RE]{text}
	\StrSubstitute{#1}{\\}{\space}[\vetempmacrolol]%
	\fancyhead[LO,LE]{\vetempmacrolol}%
	\begin{center}
		\setstretch{1.5}
		{\LARGE \selectfont \color{red} \uppercase{ \textbf{#1}}}\par%
	\end{center}
	\skiplines{1}
}%
\makeatother

\newcommand{\vesectionp}[1]{%
	\skiplines{1}
	\fancyhead[RO,RE]{#1}%
	\begin{center}
		\setstretch{1.5}
		{\Large \selectfont \uppercase{ \textbf{#1}}}\par%
	\end{center}
	\skiplines{1}
}%
\makeatother

\newcommand{\vesection}[1]{%
	\clearpage\skiplines{-2}\vesectionp{#1}%
}%
\makeatother

\makeatletter
\newcommand{\veheading}[1]{%
	\skiplines{1}%
	\begin{center}
		\setstretch{1.5}
		{\selectfont \textit{#1}}\par%
	\end{center}
	\skiplines{1}
}%
\makeatother

\makeatletter
\newcommand{\vecomment}[1]{%
	\hfill{\small\selectfont \textit{#1}.}%
}%
\makeatother

%%% OTHER FORMATS %%%

\makeatletter
\newcommand{\red}[1]{%
	{\color{red} #1}%
}%
\makeatother

\makeatletter
\newcommand{\chapter}[1]{%
	\gresetinitiallines{1}%
	\greannotation{\veformat{Chap.}}%
	\greannotation{\veformat{Dir.}}%
	\gabcsnippet{#1}%
}%
\makeatother

\makeatletter
\newcommand{\resp}[2]{%
	\vecomment{#1}
	
	\gresetinitiallines{1}%
	\greannotation{\veformat{Resp.}}%
	\greannotation{\veformat{#2.}}%
}%
\makeatother

\makeatletter
\newcommand{\tone}[2]{%
	\greannotation{\veformat{Ps.}}
	\greannotation{\veformat{#1.}}
	
	\gabcsnippet{#2}
}%
\makeatother

\makeatletter
\newcommand{\repant}[1]{%
	\gresetspecial{*}{}
	\nopagebreak[4]\skiplines{1}\nopagebreak[4]\gresetinitiallines{0}\gregorioscore{#1}\gresetinitiallines{1}
	\gresetspecial{*}{\textcolor{red}{\GreStarHeight{17}}}
}%
\makeatother



%%% FANCY %%%
\makeatletter  
\newcounter{score}
\newcounter{tabstop}[score]
\newcommand{\grealign}{%
	\@bsphack%
	\ifgre@boxing\else%
	\kern\gre@dimen@begindifference%
	\stepcounter{tabstop}%
	\expandafter\zsavepos{stop-\thescore-\thetabstop}%
	\kern-\gre@dimen@begindifference%
	\fi%
	\@esphack%
}

\newcommand{\setstops}{%
	\gdef\nstabbing@stops{%
		\hspace*{-\oddsidemargin}\hspace{-1in}%
		\hspace*{\zposx{stop-\thescore-1} sp}\=%
	}%
	\count@=\@ne
	\loop\ifnum\count@<\value{tabstop}%
	\begingroup\edef\x{\endgroup
		\noexpand\g@addto@macro\noexpand\nstabbing@stops{%
			\noexpand\hspace{-\noexpand\zposx{stop-\thescore-\the\count@} sp}%
			\noexpand\hspace{\noexpand\zposx{stop-\thescore-\the\numexpr\count@+1} sp}\noexpand\=%
		}%
	}\x
	\advance\count@\@ne
	\repeat
	\nstabbing@stops\kill
}
\makeatother

\newenvironment{nstabbing}
{\setlength{\topsep}{0pt}%
	\setlength{\partopsep}{0pt}%
	\tabbing%
	\setstops}
{\endtabbing\stepcounter{score}}

\begin{document}
	\newgeometry{margin=0pt}
	\begin{titlepage}
		\vspace*{\fill}
		
		\begin{center}
			
			\textbf{\Huge\color{black} ORDINAL}
			
		\end{center}
		
		\vspace*{\fill}
	\end{titlepage}
	
	\newgeometry{top=0.75in,bottom=0.75in,inner=0.75in,outer=0.75in,includefoot}
	
	\pagenumbering{arabic}
	\gresetinitiallines{0}
	
	\raggedbottom
	
	%\tableofcontents
	
	
	%%% COPY ABOVE %%%
	
	\vechapter{Order of the Chapel}
	
		\skiplines{-1}\vesectionp{Of the Liturgical Day}
		
			\begin{enumerate}
				\item The ferial liturgical day runs from the Vigil hour at midnight until the midnight of the following day, commencing with its Vigil hour. Sundays and the greater feasts begin sooner, running from Vespers the evening prior until the same next midnight; the lesser festivals begin at the same time but run only through the ninth hour of the day, when at Vespers is resumed the ferial order.
				
				\item The times of the hours are arranged as this: that Vigils ought be celebrated at midnight, but given the frailty of our nature, is actually celebrated at the eighth or ninth hour of the night. Lauds, or Matins, ought be celebrated precisely at the break of dawn, but during the summer months is appended to the end of Vigils at midnight. Prime is ordained for the first or second hour of the day, Terce for the third, Sext for the sixth, None for the ninth, and Vespers for the twelfth at sunset. Compline for the first or second hour of the night, such that a suitable interval exists both from Vespers and before Vigils.
				
				\item The hours of the day or night are calculated as taking the time between sunrise and sunset (or sunset and the following sunrise for the night hours) and dividing it into twelve equal parts such that the twelfth corresponds to the next astronomical event, whether sunrise or sunset. Thus the first hour is reckoned as the first after sunrise or sunset, and so forth.
				
			\vesectionp{Of the Opening and Closing}
			
				\item Except during the Bright Week of Pascha, all hours begin with all the ministers and congregation kneeling and quietly praying the \textit{Pater noste} and \textit{Ave Maria}. Each hour likewise ends with the \textit{Pater noster} in the same secret tone. Vigils, however, after these have been said to begin the hour then also adds \textit{Credo in Deum} while Compline ends with all these prayers being said.
				
				\item All hours, beside Vigils and Compline, then begin properly with the ringing of the annunciation bell, at which signal all the faithful stand as the presbyter, deacon, or other officiant begins from his place in the choir \enquote{O God, make speed to save me.} At this, all the faithful cross themselves and give the response: \enquote{O Lord, make haste to help me.} All then continue together \enquote{Glory be to the Father...}
				
				\item At Compline, however, while all the faithful are still kneeling, the officiant says instead \enquote{Turn us then, O God our Savior} to which the people reply \enquote{And let thine anger cease from us.} Then is the above order followed.
				
				\item Vigils begins, as with Compline, when the people are still kneeling with the versicle \enquote{O Lord, open thou my lips. } At this, everyone traces the sign of the cross on their lips and then gives the response: \enquote{And my mouth shall shew forth thy praise.} Then, from the back of the church some number of canons, corresponding to the rank of the liturgical day, sing the inviatatory antiphon, which the people then repeat. They process towards the choir while singing the verses of the inviatatory, stopping for these canons to sing the antiphon again, until they have entered the choir. Let the officiant always be among them, and after reverencing the sanctuary and taking their place, let the above order now be followed.
				
				\item All hours likewise terminate with the kneeling of the congregation while the choir sings the Kyrie appropriate to the liturgical day while, if he be a presbyter or deacon, the officiant incenses the whole church and all the people. Then, let everyone quietly pray a \textit{Pater noster}. At all the hours besides Vigils, Lauds, and Vespers, the \textit{Confiteor} is also said with different forms for a service presided by clergy or laity. This is the preces rite.
				
				\item After the preces, the people again stand as the officiant says either \enquote{The Lord be with you} if he be a cleric or \enquote{O Lord, hear my prayer} if he not be. Then an acolyte shall present him with the collectary from whence he shall pray the appropriate oration for the liturgical day. The congregation responds \enquote{Amen}. Then follows a short petition and the above conclusion with secret prayers.
				
				\item At Compline, however, after the oration is sung an anthem in honor of the Blessed Virgin Mary, varying upon the liturgical season: in Advent and Christmastide until the Purification is sung the \textit{Alma Redemptoris}, after the Purification until Pascha the \textit{Ave Regina Celorum}, during Paschaltide the \textit{Regin Caeli}, and then from Pentecost until Advent again the \textit{Salve Regina}. After the anthem is a versicle and second oration proper to the liturgical season, to which the people respond \enquote{Amen}. Thereafter follows the petition and conclusion as above.
				
				\item On the greater festivals of the Blessed Virgin Mary, the seasonal anthem is substituted for the \textit{Sub tuum praesidium} with its own versicle and oration.
			\end{enumerate}
\end{document}