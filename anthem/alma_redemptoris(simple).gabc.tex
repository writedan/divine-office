name:Alma Redemptoris (simple tone);
office-part:Antiphona;
mode:5;
book:The Liber Usualis, 1961, p. 277 & Chants of the Church, 1956, p. 83 & Liber antiphonarius, 1960, p. 69;
transcriber:Andrew Hinkley;
%%
(c4) Al(ef/gh)ma(g.) <sp>*</sp> Red(g)em(g)ptó(h)ris(i) Ma(j)ter,(g.) (;) quae(e) pér(e)vi(e)a(f) cae(e)li(d) por(e)ta(f) ma(h)nes,(g.) (;) Et(h) stel(j)la(i) ma(h)ris,(g.) (;) suc(e)cúr(f)re(e) ca(d)dén(e)ti(g.) (;) súr(e)ge(f)re(g) qui(h) cu(j)rat(i) pó(k)pu(j)lo :(j.) (:) Tu(j) quae(i) ge(j)nu(j)í(k)sti,(g) (;) na(j)tú(i)ra(h) mi(g)rán(f)te,(e.) (;) tu(e)um(e) san(h)ctum(g) Ge(f)ni(e)tó(de)rem :(fe..) (:) Vir(j)go(i) pri(h)us(g) ac(h) po(g)sté(e)ri(f)us,(g.) (;) Ga(h)bri(h)é(j)lis(h) ab(g) o(f)re(e) (;) su(f)mens(e) il(g)lud(g) A(h)ve,(g.) (;) pec(j)ca(i)tó(j)rum(hg) mi(h)se(f)ré(ed)re.(c.)
