name: Alma Redemptoris;
office-part: Antiphona;
mode: 5;
book: The Liber Usualis, 1961, p. 273 & Liber antiphonarius, 1960, p. 65;
transcriber: Andrew Hinkley;
initial-style: 0;
%%
(cb3) Al(d.f/gwhhv//ikkvJIHGhihh)ma(fef.) <sp>*</sp>(,) Re(h)dem(h)ptó(d)ris(ef) Ma(gvED)ter,(d.) (;) quae(h) pér(jk)vi(ih)a(gh) cae(h)li(h.) (,) por(de/fg)ta(f) ma(h.d/ewf/gv)nes,(f.) (::)

Et(k) stel(jkiih~)la(g) ma(hif)ris,(f.) (;) suc(d)cúr(eg)re(g) ca(g)dén(fgeed~)ti(d.) (,) súr(h)ge(h)re(h) qui(g) cu(hih)rat(gf) pó(e)pu(f)lo:(f.) (::)

Tu(kvIH) quae(k) ge(j)nu(kl)í(kj)sti,(ih) (,) na(gh)tú(hvGF)ra(f) mi(g)rán(fgeed~)te,(d.) (;) tu(d)um(ef) san(f)ctum(fe~) Ge(gf)ni(ed)tó(e)rem:(d.) (::) Vir(kvIGh)go(h.) pri(hfgvED)us(d.) (,) ac(h) po(jk)sté(ih)ri(gh)us,(h.) (;) Ga(h)bri(h)é(hg)lis(f) ab(gh) o(hg)re(ed) (,) su(d)mens(ef) il(g)lud(ed) A(ef)ve,(f.) (,) pec(h)ca(g)tó(h)rum(de) mi(f)se(e)ré(d.)re.(d.) (::)