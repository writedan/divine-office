Unto thee, O Lord, will I lift up my soul; my God, I have put my trust in thee * O let me not be confounded, neither let mine enemies triumph over me.

For all they who hope in thee shall not be ashamed * but such as transgress without a cause shall be put to confusion.

Shew me thy ways, O Lord * and teach me thy paths.

Lead me forth in thy truth and learn me * for thou art the God of my salvation, in thee hath been my hope all the day long.

Call to remembrance, O Lord, thy tender mercies * and thy loving-kindness, which have been ever of old.

O remember not the sins and offenses of my youth * but according to thy mercy think thou upon me, O Lord, for thy goodness.

Gracious and righteous is the Lord * therefore will he teach sinners in the way.

Them who are meek shall he guide in judgment * and such as are gentle, them shall he learn his way.

All the paths of the Lord are mercy and truth * unto such as keep his covenant and his testimonies.

For thy Name's sake, O Lord * be merciful unto my sin for it is great.

What man is he who feareth the Lord * him shall he teach in the way that he shall choose.

His soul shall swell at ease * and his seed shall inherit the land.

The secret of the Lord is among them who fear him * and he will shew them his covenant.

Mine eyes are ever looking unto the Lord * for he shall pluck my feet out of the net.

Turn thee unto me, and have mercy upon me * for I am desolate and in misery.

The sorrows of my heart are enlarged * O bring thou me out of my troubles.

Look upon my adversity and misery * and forgive me all my sin.

Consider mine enemies, how many they are * let me not be confounded, for I have put my trust in thee.

Let perfectness and righteous dealing wait upon me * for my hope hath been in thee.

Deliver Israel, O God * out of all his troubles.